\chapter{Cubed-sphere grids}
\label{chp-cs-grids}

The cubed-sphere grid was originally proposed by \citet{sadourny:1972}
and was reinvestigated by \citet{ronchi:1996} and \citet{rancic:1996}.
As it is usual to Planotic grids, we start with a cube circumscribed
in a sphere and project its faces on the sphere.
This chapter aims to review and investigated geometrical properties
of the cubed-spheres available in the literature.
This chapter is under development.

%\citep{taylor:1997}
%\citep{nair:2005}
%\citep{lauritzen:2011}

\section{Cubed-sphere mappings}
\label{cs-mappings}

\subsection{Equidistant cubed-sphere}
\label{equidistant-cs}

We consider a sphere radius $R>0$ and
$a = \frac{R}{\sqrt{3}}$ representing the half-length of 
the cube, and the family of maps
$\Psi_{p}: [-a,a] \times [-a,a] \to \mathbb{S}^2_R$, $p=1, \cdots, 6$,
where:
\begin{equation}
	\label{chp3-eqdistant-psi1}
	\Psi_{1}(x,y) = \frac{R}{\sqrt{a^2 + x^2 + y^2}}(a, x, y), 
\end{equation}

\begin{equation}
	\label{chp3-eqdistant-psi2}
	\Psi_{2}(x,y) = \frac{R}{\sqrt{a^2 + x^2 + y^2}}(-x, a, y), 
\end{equation}

\begin{equation}
	\label{chp3-eqdistant-psi3}
	\Psi_{3}(x,y) = \frac{R}{\sqrt{a^2 + x^2 + y^2}}(-a, -x, y), 
\end{equation}

\begin{equation}
	\label{chp3-eqdistant-psi4}
	\Psi_{4}(x,y) = \frac{R}{\sqrt{a^2 + x^2 + y^2}}(x, -a, y), 
\end{equation}

\begin{equation}
	\label{chp3-eqdistant-psi5}
	\Psi_{5}(x,y) = \frac{R}{\sqrt{a^2 + x^2 + y^2}}(-y, x, a), 
\end{equation}

\begin{equation}
	\label{chp3-eqdistant-psi6}
	\Psi_{6}(x,y) = \frac{R}{\sqrt{a^2 + x^2 + y^2}}(y, x, -a).
\end{equation}

The family of maps $\{\Psi_{p}, p = 1, \cdots, 6\}$ allow us to cover the sphere, and 
by creating an uniform partition of the square $[-a,a]\times[-a,a]$,
we generate the equidistant cubed-sphere grid. 
we show an example of an equidistante cubed-sphere grid.

The derivative of the maps $\Psi_p$ are given by:
\begin{equation}
	\label{chp3-eqdistant-dpsi1}
	D\Psi_{1}(x,y) = \frac{R}{{(a^2 + x^2 + y^2)}^{3/2}}
	\begin{bmatrix}
		-ax & -ay \\
	 	 a^2+y^2  & -xy \\
		 -xy  & a^2+x^2
	\end{bmatrix},
\end{equation}

\begin{equation}
	\label{chp3-eqdistant-dpsi2}
	D\Psi_{2}(x,y) = \frac{R}{{(a^2 + x^2 + y^2)}^{3/2}}
	\begin{bmatrix}
		-(a^2+y^2) & xy \\
		 -ax &  -ay \\
		 -xy &  a^2+x^2
	\end{bmatrix},
\end{equation}

\begin{equation}
	\label{chp3-eqdistant-dpsi3}
	D\Psi_{3}(x,y) = \frac{R}{{(a^2 + x^2 + y^2)}^{3/2}}
	\begin{bmatrix}
		 ax &  ay \\
		-(a^2+y^2) & xy \\
		 -xy &  a^2+x^2
	\end{bmatrix},
\end{equation}

\begin{equation}
	\label{chp3-eqdistant-dpsi4}
	D\Psi_{4}(x,y) = \frac{R}{{(a^2 + x^2 + y^2)}^{3/2}}	
	\begin{bmatrix}
		 a^2+y^2 &  -xy \\
		 ax & ay \\
		 -xy &  a^2+x^2
	\end{bmatrix},
\end{equation}

\begin{equation}
	\label{chp3-eqdistant-dpsi5}
	D\Psi_{5}(x,y) = \frac{R}{{(a^2 + x^2 + y^2)}^{3/2}}	
	\begin{bmatrix}
		 xy  & -(a^2+x^2) \\
	 	 a^2+y^2  &  -xy \\
		-ax & -ay
	\end{bmatrix},
\end{equation}

\begin{equation}
	\label{chp3-eqdistant-dpsi6}
	D\Psi_{6}(x,y) = \frac{R}{{(a^2 + x^2 + y^2)}^{3/2}}
	\begin{bmatrix}
		 -xy  &  a^2+x^2 \\
		 a^2+y^2  &  -xy \\
		 ax &  ay
	\end{bmatrix}.
\end{equation}

With the aid of the derivative, we may define a basis of tangent vectors 
$\{ \boldsymbol{g}_{1}, \boldsymbol{g}_{2} \}$ on each point on the sphere by:
\begin{equation}
	\boldsymbol{g}_{1}(x,y;p) = D\Psi_{p}(x,y)
	\begin{bmatrix}
		 1 \\
		 0
	\end{bmatrix},
	\boldsymbol{g}_{2}(x,y;p) = D\Psi_{p}(x,y)
	\begin{bmatrix}
		 0 \\
		 1
	\end{bmatrix},
\end{equation}

In other words, we have $\{\boldsymbol{g}_{1}(x,y;p),\boldsymbol{g}_{2}(x,y;p)\} \subset T_{\Psi_p(x,y)}
\mathbb{S}_{R}^2$, $\forall (x,y) \in [-a,a]\times[-a,a]$.
Notice that
\begin{equation}
	\label{chp3-eqdistant-psitensor}
	[D\Psi_{p}(x,y)]^TD\Psi_{p}(x,y)
	= \frac{R^2}{(a^2 + x^2 + y^2)^2}
	\begin{bmatrix}
		 a^2 + x^2 &  -xy \\
		 -xy & a^2 + y^2
	\end{bmatrix},
\end{equation}

does not depend on $p$.
Hence, it makes sense to define the matrix 
$G_{\Psi}(x,y) = [D\Psi_{p}(x,y)]^TD\Psi_{p}(x,y)$ 
which is known as metric tensor.
It is easy to see that:
\begin{equation}
	\label{chp3-eqdistant-psi-metric-tensor}
	G_{\Psi}(x,y) = 
	\begin{bmatrix}
		\langle \boldsymbol{g}_{1}(x,y;p), \boldsymbol{g}_{1}(x,y;p) \rangle & 
		\langle \boldsymbol{g}_{1}(x,y;p), \boldsymbol{g}_{2}(x,y;p) \rangle \\
		\langle \boldsymbol{g}_{1}(x,y;p), \boldsymbol{g}_{2}(x,y;p) \rangle  &
		\langle \boldsymbol{g}_{2}(x,y;p), \boldsymbol{g}_{2}(x,y;p) \rangle 
	\end{bmatrix}
\end{equation}
and that $G_{\Psi}(x,y)$ is positive-definite, 
$\forall (x,y) \in [-a,a]\times[-a,a]$.
The Jacobian of the metric tensor $G_{\Psi}(x,y)$ is then given by:
\begin{equation}
	\sqrt{|\det{G_{\Psi}(x,y)}|} = \frac{R^2}{(a^2+x^2+y^2)^{3/2}}a 
\end{equation}

\subsection{Equiangular cubed-sphere}
\label{equiangular-cs}
We consider again $a=\frac{R}{\sqrt{3}}$
and we define the family of maps
$\Phi_{p}: [-\frac{\pi}{4},\frac{\pi}{4}] 
\times [-\frac{\pi}{4},\frac{\pi}{4}] 
\to \mathbb{S}^2_R$, $p=1, \cdots, 6$,
given by $\Phi_{p}(x,y) = \Psi_{p}(a\tan{x}, a\tan{y})$.
The coordinates $(a\tan{x}, a\tan{y})$ are called as angular coordinates.
By the chain rule:
\begin{equation}
	D\Phi_{p}(x,y) = a
	D\Psi_{p}(a\tan{x}, a\tan{y})
	\begin{bmatrix}
		\frac{1}{\cos^2 x} & 0 \\ 
		0 & \frac{1}{\cos^2 y} 
	\end{bmatrix}	
\end{equation}

and therefore we can define the following tangent vectors
\begin{align}
	\boldsymbol{r}_{1}(x,y;p) = D\Phi_{p}(x,y)
	\begin{bmatrix}
		 1 \\
		 0
	\end{bmatrix}
	= \frac{a}{\cos^2 x}
	\boldsymbol{g}_{1}(\tan{x}, \tan{y}; p)
	,\\
	\boldsymbol{r}_{2}(x, y ;p) = D\Phi_{p}(x,y)
	\begin{bmatrix}
		 0 \\
		 1
	\end{bmatrix}
	= \frac{a}{\cos^2 y}
	\boldsymbol{g}_{2}(\tan{x}, \tan{y}; p),
\end{align}
that is, $\{\boldsymbol{r}_{1}(x,y;p),\boldsymbol{r}_{2,p}(x,y;p)\} \subset T_{\Phi_p(x,y)}
\mathbb{S}_{R}^2$, $\forall (x,y) \in 
[-\frac{\pi}{4},\frac{\pi}{4}] 
\times [-\frac{\pi}{4},\frac{\pi}{4}]$.


Again, it makes sense to define the matrix 
\begin{align}
	G_{\Phi}(x,y) &= [D\Phi_{p}(x,y)]^TD\Phi_{p}(x,y) \\
	&= a^2
	[D\Psi_{p}(a\tan{x},a\tan{y})]^T
	\begin{bmatrix}
		\frac{1}{\cos^4 x} & 0 \\ 
		0 & \frac{1}{\cos^4 y} 
	\end{bmatrix}
	D\Psi_{p}(a\tan{x}, a\tan{y})
\end{align}
that does not depend on $p$ and is the  metric tensor.
It is easy to see that:
\begin{equation}
	\label{chp3-eqangle-phi-metric-tensor}
	G_{\Phi}(x,y) = 
	\begin{bmatrix}
		\langle \boldsymbol{r}_{1}(x,y;p), \boldsymbol{r}_{1}(x,y;p) \rangle & 
		\langle \boldsymbol{r}_{1}(x,y;p), \boldsymbol{r}_{2}(x,y;p) \rangle \\
		\langle \boldsymbol{r}_{1}(x,y;p), \boldsymbol{r}_{2}(x,y;p) \rangle  &
		\langle \boldsymbol{r}_{2}(x,y;p), \boldsymbol{r}_{2}(x,y;p) \rangle 
	\end{bmatrix}
\end{equation}
and that $G_{\Phi}(x,y)$ is positive-definite, 
$\forall (x,y) \in [-\frac{\pi}{4},\frac{\pi}{4}] 
\times [-\frac{\pi}{4},\frac{\pi}{4}]$.
The Jacobian of the metric tensor $G_{\Phi}(x,y)$ is then given by:
\begin{align}
	\sqrt{|\det{G_{\Phi}(x,y)}|} &= \frac{a}{\cos^2 x \cos^2 y}
	\frac{R^2}{(a^2 + a^2\tan^2x + a^2\tan^2y)^{3/2}}a\\
	&= \frac{R^2}{\cos^2 x \cos^2 y}
	\frac{1}{(1 + \tan^2x + \tan^2y)^{3/2}}
\end{align}

