\chapter{Two-dimensional finite-volume methods}
\label{chp-2d-fv}
In Chapter \ref{chp-1d-fv}, we tackled the problem of solving the one-dimensional linear
advection equation using the finite-volume method based on PPM.
In this Chapter, we are interested in solving the two-dimensional linear
advection equation using the finite-volume method. This step plays a key role in this work,
since as we shall see in Chapter  \ref{chp-cs-fv}, the solution of the linear
advection equation on the cubed-sphere boils down to solving one two-dimensional linear
advection equations at each cube face with interpolation between the adjacent panels.

A natural way to derive a finite-volume method for the two-dimensional linear
advection equation would be to extend the PPM for two dimensions. Indeed, a piecewise bi-parabolic extension of PPM
was proposed by \citet{rancic:1992} using a semi-lagrangian temporal discretization. 
However, the major problem of this method is its quite expensive computational cost.
A popular alternative, and usually computationally cheaper, is to use dimension-splitting methods.
These schemes replace two-dimensional problem with a sequence of one-dimensional problems.
For instance, we can solve the two-dimensional linear advection equation by solving a sequence
one-dimensional linear advection equations using the PPM  from Chapter \ref{chp-1d-fv}.
Further, we could, in principle, use any numerical method that solves the one-dimensional linear advection. 
A comparison between two-dimension and dimension splitting semi-lagrangian schemes on the plane has been investigated by \citet{chen:2017}
using the PPM as the one-dimensional solver and distorted two-dimensional grids.
Their major conclusion is that the dimension splitting schemes are more sensitive to grid distortions
but it is computationally cheaper and more accurate than their two-dimensional methods, 
specially considering large CFL numbers.

The main aim of this chapter is to give a detailed description of the dimension
splitting method proposed by \citet{lin:1996}, which is currently employed in the FV3 dynamical core,
applied to the two-dimensional linear advection equation using the one-dimensional
finite-volume schemes from Chapter \ref{chp-1d-fv}.
Similarly to Chapter \ref{chp-1d-fv}, we start this Chapter 
start with a review of the two-dimensional advection equation in
the integral form in Section \ref{sec-adv2d}, 
and in Section \ref{sec:fv-2d} we set the framework of
general two-dimensional finite-volumes schemes.
Section \ref{sec-dsplit} presents the dimension splitting method and 
numerical experiments are shown in Section \ref{sec-ds-exp}.

\section{Two-dimensional advection equation in integral form}
\label{sec-adv2d}
Let us consider a $\mathcal{C}^1$ velocity field given by
$\boldsymbol{u}: \mathbb{R}^2 \to \mathbb{R}^2$,  $\boldsymbol{u}=(u,v)$, where
$u$ is the velocity in $x$-direction and $v$ is the velocity in $x$ and $y$ direction. 
The two-dimensional advection equation in the differential form in 
a domain $\Omega=[a,b]\times[c,d] \subset \mathbb{R}^2$
associated to the velocity field $\boldsymbol{u}$ is given by:
\begin{equation}
\label{sec-adv2d:eq1}
\frac{\partial q}{\partial t}(x, y, t) +
\nabla \cdot (q\boldsymbol{u})(x, y, t)
= 0, \quad \forall (x, y, t) \in \Omega^{\mathrm{o}}
 \times ]0, +\infty[,
\footnote{$\Omega^{\mathrm{o}}$ denotes the interior of $\Omega$. 
	Namely, $\Omega^{\mathrm{o}} = ]a,b[ \times ]c,d[$.}
\end{equation}
where $\nabla \cdot (q\boldsymbol{u})$ is the spatial divergence
\begin{equation}
	\label{sec-adv2d:eqdiv}
	\nabla \cdot (q\boldsymbol{u})(x, y, t) =  
	\frac{\partial (uq)}{\partial x}(x, y, t) + \frac{\partial (vq)}{\partial y}(x, y, t).
\end{equation}
A classical or strong solution to the two-dimensional advection equation is a 
$\mathcal{C}^1$ function ${q}$ satisfying Equation \eqref{sec-adv2d:eq1}.
As we did in Section \ref{chp2-sec1}, our goal is to deduce an
integral form of Equation \eqref{sec-adv2d:eq1}.
Thus, let us consider  $[x_1,x_2] \times [y_1, y_2]
\subset \Omega^{\mathrm{o}}$ and $[t_1,t_2] \subset [0, +\infty[$.
Integrating Equation $\eqref{sec-adv2d:eq1}$ over 
$[x_1,x_2] \times [y_1, y_2]$ yields:
\begin{align}
	\label{sec-adv2d:eq2}
	\frac{d}{d t} \bigg(\int_{x_1}^{x_2} \int_{y_1}^{y_2}
	{q}(x, y, t) \,dx \,dy \bigg)=
	&-\int_{y_1}^{y_2} \bigg({(uq)}(x_2, y, t)
	-{(uq)}(x_1, y, t) \bigg) \,dy \\ \nonumber
	&-\int_{x_1}^{x_2} \bigg({(vq)}(x, y_2, t)
	-{(vq)}(x, y_1, t) \bigg) \,dx.
\end{align}
Integrating Equation \eqref{sec-adv2d:eq2} over the time interval $[t_1,t_2]$, 
we have:
\begin{align}
	\label{sec-adv2d:eq3}
	\int_{x_1}^{x_2} \int_{y_1}^{y_2}
	{q}(x, y, t_{n+1}) \,dx \,dy = &\int_{x_1}^{x_2} \int_{y_1}^{y_2}
	{q}(x, y, t_n) \,dx \,dy \\ \nonumber
	&-\int_{t_1}^{t_2} \int_{y_1}^{y_2} \bigg({(uq)}(x_2, y, t)
	-{(uq)}(x_1, y, t) \bigg) \,dy \,dt\\ \nonumber
	&-\int_{t_1}^{t_2} \int_{x_1}^{x_2} \bigg({(vq)}(x, y_2, t)
	-{(vq)}(x, y_1, t) \bigg) \,dx \,dt.
\end{align}
Equation \eqref{sec-adv2d:eq3} is the integral form of Equation 
\eqref{sec-adv2d:eq1}. We say that ${q} \in 
L^{\infty}{(\Omega \times [0, +\infty[ )}$ is a weak
solution to the advection equation \eqref{sec-adv2d:eq1} if ${q}$
satisfies the integral form \eqref{sec-adv2d:eq3}, 
$\forall [x_1,x_2]\times[y_1,y_2] \subset \Omega^{\mathrm{o}}$ and 
$\forall [t_1,t_2] \subset [0,+\infty[$.
Similarly to Section \ref{chp2-sec1}, these problems are equivalent
when  ${q}$ is a $\mathcal{C}^1$ function.

We consider an initial condition ${q_0} \in L^{\infty}{(\Omega)}$,
${q}(x,y,0) =  {q_0}(x,y)$, $\forall (x,y) \in \Omega$.
Boundary conditions will be assumed bi-periodic.
%At last, the matrix $\alpha D{f}(q) + \beta D{g}(q)$ is assumed to have
%real eigenvalues and be diagonalizable
%$\forall q \in \mathbb{R}^m , \forall \alpha, \beta \in \mathbb{R}$
%\citep{leveque:1990}, so that we have a hyperbolic conservation law.
Therefore, we are again dealing with a Cauchy problem. Besides that, the 
two-dimensional advection equation is a hyperbolic PDE.

To move in the direction of a discrete version of Equation \eqref{sec-adv2d:eq3},
let us discretize the domain $D = \Omega \times [0,T]$ following 
the notations of Section \ref{chp2-sec1}.
Given a positive integer $N_T$, we define the time step 
$\Delta t = \frac{T}{N_T}$, $t_n = n \Delta t$, for $n = 0, 1 ,\cdots, N_T$.
The spatial discretization is constructed through an uniformly spaced partition of $\Omega$ given by:
\begin{align}
	[a,b] &= \bigcup_{i=1}^N X_i, 
	\text{ where } X_i= [x_{i-\frac{1}{2}}, x_{i+\frac{1}{2}}] \text{ and } 
	a = x_{\frac{1}{2}} < x_{\frac{3}{2}} < \cdots < x_{N-\frac{1}{2}} < x_{N+\frac{1}{2}} = b, \\
	[c,d] &= \bigcup_{j=1}^M Y_j, 
\text{ where } Y_j= [y_{j-\frac{1}{2}}, y_{j+\frac{1}{2}}] \text{ and } 
	c = y_{\frac{1}{2}} < y_{\frac{3}{2}} < \cdots < y_{M-\frac{1}{2}} < y_{M+\frac{1}{2}} = d, \\
    \Omega &=  \bigcup_{i=1}^N \bigcup_{j=1}^M \Omega_{ij}, \text{ where } \Omega_{ij} = X_i \times Y_j.
\end{align}
The regions $\Omega_{ij}$ are known as control volumes or cells. \index{Control volume}
Similarly to  Chapter \ref{chp-1d-fv} we employ the notations
$\Delta x = x_{i+\frac{1}{2}} - x_{i-\frac{1}{2}}$,
$\Delta y = y_{j+\frac{1}{2}} - y_{j-\frac{1}{2}}$ 
and $x_i = \frac{1}{2}(x_{i+\frac{1}{2}} + x_{i-\frac{1}{2}})$
$y_j = \frac{1}{2}(y_{j+\frac{1}{2}} + y_{j-\frac{1}{2}})$, $\forall i = 1, \cdots, N$, 
$\forall j = 1, \cdots, M$,
to define the control volume lengths and centroids, respectively.
Finally, we denote by ${Q}_{ij}(t)$ as the
average values of state variable at time $t$
in the control volume $\Omega_{ij}$, that is:
\begin{equation}
	\label{sec-adv2d:eq4}
	{Q}_{ij}(t) = \frac{1}{\Delta x \Delta y}
	\int_{x_{i-\frac{1}{2}}}^{x_{i+\frac{1}{2}}} \int_{y_{j-\frac{1}{2}}}^{y_{j-\frac{1}{2}}} {q}(x,y,t) \,dx.
\end{equation}
Substituting $t_1, t_2, x_1, x_2, y_1$ and $y_2$ by 
$t_{n}, t_{n+1}, x_{i-\frac{1}{2}}, x_{i+\frac{1}{2}}, y_{j-\frac{1}{2}}, y_{j+\frac{1}{2}}$,
respectively, in Equation \eqref{sec-adv2d:eq3}, we obtain:
\begin{align}
	\label{sec-adv2d:eq5}
	{Q}_{ij}(t_{n+1})  = {Q}_{ij}(t_{n})
	&- \frac{\Delta t}{\Delta x \Delta y}
	\delta _x \bigg( \frac{1}{\Delta t}
	\int_{t_1}^{t_2} \int_{y_{j-\frac{1}{2}}}^{y_{j+\frac{1}{2}}} 
	{(uq)}(x_{i}, y, t)
	\,dy \,dt \bigg) \\ \nonumber
	&- \frac{\Delta t}{\Delta x \Delta y}
	\delta _y \bigg( \frac{1}{\Delta t}
	\int_{t_1}^{t_2} \int_{x_{i-\frac{1}{2}}}^{x_{i+\frac{1}{2}}} 
	{(vq)}(x, y_{j}, t)
	\,dx \,dt \bigg),
\end{align}
where we are using the centered finite-difference notation:
\begin{align}
	\label{sec-adv2d:eq6}
	\delta_x {h}(x_i,y, t) = 
	{h}(x_{i+\frac{1}{2}}, y, t) - 
	{h}(x_{i-\frac{1}{2}}, y, t), \\
	\delta_y {h}(x, y_j,t) = 
    {h}(x, y_{j+\frac{1}{2}},t) - 
    {h}(x, y_{j-\frac{1}{2}},t),
\end{align}
for any function ${h}$. The Equation \eqref{sec-adv2d:eq5} is useful to
motivate two-dimensional finite-volume schemes, as we shall see in the next section.

\section{The finite-volume approach}
\label{sec:fv-2d}
This Section is basically an extension to two dimensions of the concepts presented in Section \ref{chp2-sec2}.
We introduce the following spaces of bi-periodic functions:
\begin{align*}
	\mathcal{F}(\mathbb{T}^2) &= \{q:\mathbb{R}^2 \to \mathbb{R};
	\quad q(x+b-a,y) = q(x,y), \quad q(x,y+d-c) = q(x,y),\quad \forall (x,y) \in \mathbb{R}^2\},\\
	\mathcal{F}(\mathbb{T}^2_{T}) &= \{q:\mathbb{R}^2\times[0,T]\to \mathbb{R};\quad q(\cdot,\cdot,t) \in \mathcal{F}(\mathbb{T}^2), \quad \forall t \in [0,T]\},\\
	\mathcal{C}^k(\mathbb{T}^2_{T}) &= \{q\in \mathcal{C}^k(\mathbb{R}^2\times[0,T]): q \in \mathcal{F}(\mathbb{T}^2_{T})\}.
\end{align*}
where we are using the notation $\mathbb{T}^2_{T} = \mathbb{T}^2\times[0,T]$ and $\mathbb{T}^2 = [a,b] \times [c,d]$ denotes the torus of lengths $b-a$ and $d-c$.
We are using the notation $\mathbb{T}^2$ since we may think of bi-periodic functions as functions defined on the torus of lengths $b-a$ and $d-c$.
Whenever we use the notation $\mathbb{T}^2$, $a, b, c$ and $d$ will be implicitly defined.
We also introduce the following the locally integrable periodic functions:
\begin{align*}
	L^p_{\text{loc}}(\Omega) &= \{q:\Omega\to \mathbb{R};\quad \int_{K} |q(x,y)|^p\,dx\,dy  < +\infty, \quad \text{for all compact sets } K \subset \Omega\},\\
	{L}^{p}_{\text{loc}}(\mathbb{T}^2) &= \{q\in \mathcal{F}(\mathbb{T}^2): \quad  q \in L^p_{\text{loc}}(\mathbb{R}^2)\},\\
	{L}^{p,x,y}_{\text{loc}}(\mathbb{T}^2_{T}) &= \{q\in \mathcal{F}(\mathbb{T}^2_{T}): \forall t \in [0,T], \quad q(\cdot,\cdot,t) \in L^p_{\text{loc}}(\mathbb{R}^2)\},\\
	{L}^{p,y,t}_{\text{loc}}(\mathbb{T}^2_{T}) &= \{q\in \mathcal{F}(\mathbb{T}^2_{T}): \forall x \in \mathbb{R},\quad q(x,\cdot,\cdot) \in L^p_{\text{loc}}(\mathbb{R}\times [0,T])\},\\
	{L}^{p,x,t}_{\text{loc}}(\mathbb{T}^2_{T}) &= \{q\in \mathcal{F}(\mathbb{T}^2_{T}): \forall y \in \mathbb{R},\quad q(\cdot,y,\cdot) \in L^p_{\text{loc}}(\mathbb{R}\times [0,T])\},\\
\end{align*}

\subsection{Discretization of the problem}
The problem of two-dimensional advection equation in the integral form 
presented Section \ref{sec-adv2d} is written in a concise way in Problem \ref{chp3-sec2-prob1}.
\begin{prob}
	\label{chp3-sec2-prob1}
	Given an initial condition ${q}_0\in {L}^{1}_{\text{loc}}(\mathbb{T}^2) \cap {L}^{2}_{\text{loc}}(\mathbb{T}^2)$ and
	a velocity function $u \in {L}^{2,y,t}_{\text{loc}}(\mathbb{T}^2_{T})$ in the $x$-direction, 
	a velocity function $v \in {L}^{2,x,t}_{\text{loc}}(\mathbb{T}^2_{T})$ in the $y$-direction, 
 	we would like to find a weak solution 
 	${q} \in {L}^{1,x,y}_{\text{loc}}(\mathbb{T}^1_{T}) \cap {L}^{2,x,t}_{\text{loc}}(\mathbb{T}^2_{T}) \cap {L}^{2,y,t}_{\text{loc}}(\mathbb{T}^2_{T})$
	of the two-dimensional advection equation in the integral form:
	\begin{align*}
		\int_{x_1}^{x_2} \int_{y_1}^{y_2}
		{q}(x, y, t) \,dx \,dy = &\int_{x_1}^{x_2} \int_{y_1}^{y_2}
		{q}(x, y, t) \,dx \,dy \\ \nonumber
		&-\int_{t_1}^{t_2} \int_{y_1}^{y_2} \bigg({(uq)}(x_2, y, t)
		-{(uq)}(x_1, y, t) \bigg) \,dy \,dt\\ \nonumber
		&-\int_{t_1}^{t_2} \int_{x_1}^{x_2} \bigg({(vq)}(x, y_2, t)
		-{(vq)}(x, y_1, t) \bigg) \,dx \,dt.
	\end{align*}
	$\forall [x_1, x_2]\times [y_1, y_2] \times[t_1, t_2] \subset [a,b] \times [c,d] \times[0,T]$, 
	and ${q}(x, y, 0) = {q}_0(x, y)$, $\forall (x, y) \in [a,b] \times [c,d]$.
\end{prob}
For Problem \ref{chp3-sec2-prob1}, the total mass in $\Omega$ is defined by: 
\begin{equation}
	{M}_{\Omega}(t) = \int_{\Omega} {q}(x,y,t) \,dx \,dy , \quad \forall t \in [0,T],
\end{equation}
and is conserved within time: 
\begin{equation}
	{M}_{\Omega}(t) = {M}_{\Omega}(0), \quad \forall t \in [0,T].
\end{equation}
considering a discretization of the domain $[a,b] \times [0,T]$. 
Similar to Definitions \ref{chp2-def-2dgrid} and \ref{chp2-def-dxtimegrid},
we introduce the concepts of $(\Delta x,\Delta y)$-uniform grid and $(\Delta x, \Delta y,\Delta t, \lambda_x,\lambda_y)$ discretization.
\begin{definition}[$(\Delta x,\Delta y)$-uniform grid]
	\label{chp2-def-2dgrid}
	Given $[a,b]\times [c,d]$ and positive real numbers $\Delta x$ and $\Delta y$ such that $\Delta x = (b-a)/N$, 
	$\Delta y = (d-c)/M$, for positive integers $N$ and $M$,
	we say that a $(N,M)$-tuple $\mathcal{D}=(\Omega_{ij})_{i=1,\cdots,N,j=1,\cdots,M}$ is a $(\Delta x, \Delta y)$-uniform grid for $[a,b]\times [c,d]$ if
	$\Omega_{ij} = [x_{i-\frac{1}{2}}, x_{i+\frac{1}{2}}]\times [y_{j-\frac{1}{2}}, y_{j+\frac{1}{2}}]$, 
	$a = x_{\frac{1}{2}} < x_{\frac{3}{2}} < \cdots < x_{N-\frac{1}{2}} < x_{N+\frac{1}{2}} = b$,
	$c = y_{\frac{1}{2}} < y_{\frac{3}{2}} < \cdots < y_{M-\frac{1}{2}} < y_{M+\frac{1}{2}} = d$,
	$\Delta x = x_{i+\frac{1}{2}}-x_{i-\frac{1}{2}}$, 	$\Delta y = y_{j+\frac{1}{2}}-y_{j-\frac{1}{2}}$.
	Each $\Omega_{ij}$ is called control volume or cell.
\end{definition}
\begin{remark}
	We may define the cells $\Omega_{ij}$ for $i$ outside of the range $1,\cdots, N$ or $j$ outside of the range $1,\cdots, M$  by 
	$\Omega_{ij} = [a+(i-1)\Delta x,a+i\Delta x]\times [c+(j-1)\Delta x,c+j\Delta x]$.  These cells are called ghost cells.
\end{remark}
\begin{definition}[$(\Delta x,\Delta y,\Delta t, \lambda_x, \lambda_y)$-discretization]
	\label{chp2-def-dxdytimegrid}
	Given $[a,b]\times[c,d]\times [0,T]$ and positive real numbers $\Delta x$ $\Delta y$ and $\Delta t$, we say that $(\mathcal{D},\mathcal{T})$ is a
	$(\Delta x,\Delta y,\Delta t, \lambda_x, \lambda_y)-$discretization of $[a,b]\times[c,d]\times [0,T]$ if $\Omega$ is a $(\Delta x,\Delta y)$-uniform 
	grid for $[a,b]\times[c,d]$ and $\mathcal{T}$ is a $\Delta t$-temporal grid for $[0,T]$, $\frac{\Delta t}{\Delta x} = \lambda_x$
	and  $\frac{\Delta t}{\Delta y} = \lambda_y$.
\end{definition}
\begin{remark}
	Whenever we mention a $(\Delta x,\Delta y)-$uniform grid, or a $(\Delta x,\Delta y,\Delta t, \lambda_x, \lambda_y)$-discretization,
	then $\Omega_{ij}$, $N$ and $M$ are implicitly defined.
\end{remark}
Section \ref{sec-adv2d} introduced a version of Problem \ref{chp3-sec2-prob1}
considering a discretization of the domain $[a,b] \times [c,d] \times[0,T]$. 
This version is also summarized in Problem \ref{chp3-sec2-prob2}.	
\begin{prob}
	\label{chp3-sec2-prob2}
	Assume the framework of Problem \ref{chp3-sec2-prob1}
	and that $(\mathcal{D},\mathcal{T})$ is a $(\Delta x, \Delta y, \Delta t, \lambda_x,\lambda_y)$-discretization of $[a,b]\times [c,d]\times [0,T]$.
	Since we are in the framework of Problem \ref{chp3-sec2-prob1}, it follows that:
	\begin{align*}
		{Q}_{ij}(t_{n+1})  = {Q}_{ij}(t_{n})
		&- {\lambda_x}
		\delta _x \bigg( \frac{1}{\Delta t}
		\int_{t^n}^{t^{n+1}} \int_{y_{j-\frac{1}{2}}}^{y_{j+\frac{1}{2}}} 
		{(uq)}(x_{i}, y, t)
		\,dy \,dt \bigg) \\ \nonumber
		&- {\lambda_y}
		\delta _y \bigg( \frac{1}{\Delta t}
		\int_{t^n}^{t^{n+1}} \int_{x_{i-\frac{1}{2}}}^{x_{i+\frac{1}{2}}} 
		{(vq)}(x, y_{j}, t)
		\,dx \,dt \bigg),
	\end{align*}
	where ${Q}_{ij}(t) = \frac{1}{\Delta x \Delta y}
	\int_{x_{i-\frac{1}{2}}}^{x_{i+\frac{1}{2}}} 
	\int_{y_{j-\frac{1}{2}}}^{y_{j+\frac{1}{2}}} {q}(x,y,t) \,dx \,dy$.
	
	Our problem now consists of finding the values ${Q}_{ij}(t_{n})$, 
	$\forall i = 1, \cdots, N$, $\forall j = 1, \cdots, M$, $\forall n = 1, \cdots, N_T$,
    given the initial values ${Q}_{ij}(0)$, $\forall i = 1, \cdots N$, $\forall j = 1, \cdots, M$.
	In other words, we would like to find the average values of ${q}$
	in each control volume $\Omega_{ij}$ at the considered time instants.
\end{prob}
Next, we introduce the definitions of grid functions at cell centroids and C-grid functions. 
\begin{definition}[$(\Delta x,\Delta y)$-grid function]
	\label{chp2-rmk-2d-gridfunction1}
	For a $(\Delta x,\Delta y)$-uniform grid $\mathcal{D}$, we say that $Q \in \mathbb{R}^{N\times M}$ is a 
	$(\Delta x,\Delta y)$-grid function, where we assume that $Q$ is ordered by
	the lines in the $x$ direction of the grid, \textit{i.e.},
	\begin{equation*}
		Q = (Q_{11}, Q_{21} \cdots, Q_{M1}, Q_{12}, Q_{22} \cdots, Q_{M2}, \cdots,  Q_{1N}, Q_{2N} \cdots, Q_{NM}).
	\end{equation*}
	We denote the space of $(\Delta x,\Delta y)$-grid functions by $\mathbb{R}^{\Delta x \times \Delta y}$.
\end{definition}
	
\begin{definition}[$(\Delta x,\Delta y)$-C grid wind]
	\label{chp2-rmk-2d-gridfunction2}
	For a $(\Delta x,\Delta y)$-uniform grid $\mathcal{D}$, we say that $(u,v)$ is a $(\Delta x,\Delta y)$-C grid wind if 
	$u \in \mathbb{R}^{(N+1) \times M}$, $v \in \mathbb{R}^{N \times (M+1)}$, where we assume that $u$ is ordered by
	the lines in the $x$ direction of the grid, \textit{i.e.},
	\begin{align*}
	u = (u_{\frac{1}{2},1}, u_{\frac{3}{2},1}, \cdots, u_{N+\frac{1}{2},1}, u_{\frac{1}{2},2}, u_{\frac{3}{2},2}, \cdots, u_{N+\frac{1}{2},2}, \cdots ,
	u_{\frac{1}{2},M}, u_{\frac{3}{2},M}, \cdots, u_{N+\frac{1}{2},M}),
	\end{align*}
	$v$ is ordered by the lines in the $y$ direction of the grid, \textit{i.e.},
	\begin{align*}
	v = (v_{1,\frac{1}{2}}, v_{1,\frac{3}{2}}, \cdots, v_{1,M+\frac{1}{2}}, v_{2,\frac{1}{2}}, v_{2,\frac{3}{2}}, \cdots, v_{2,M+\frac{1}{2}}, \cdots ,
	v_{N,\frac{1}{2}}, v_{N,\frac{3}{2}}, \cdots, v_{N,M+\frac{1}{2}}).
\end{align*}
\end{definition}
\begin{remark}
	When computing stencils, we may need values of $Q_{ij}$ such that the
	indexes $i$ and $j$ are out of the range $1,\cdots, N$, $1,\cdots, M$.
	These values are called ghost cell values.
	Since we are under the assumption of periodic boundary conditions, this problem
	is overcome by assuming periodicity on the grid function $Q$. 
	The same applies for $(\Delta x,\Delta y)$-C grid functions.
\end{remark}
Finally, we define the two-dimensional (2D) finite-volume (FV)
scheme problem as follows in Problem \ref{chp3-sec2-prob3}.
\begin{remark}
	For Problem \ref{chp2-sec2-prob2}, we define the $(\Delta x,\Delta y)$-grid functions
	$q^n$ and $Q(t^n)$ , where ${q}^n_{ij} = {q}(x_i, y_j, t^{n})$, $Q(t^n)_{ij} = Q_{ij}(t^n)$, for $n=0, \cdots, N_T$.
	We also define the $(\Delta x,\Delta y)$-C grid wind $(u^n,v^n)$ where $u_{i+\frac{1}{2},j}^n = u(x_{i+\frac{1}{2}},y_j,t^{n})$ and
	$v_{i,j+\frac{1}{2}}^n = v(x_i,y_{j+\frac{1}{2}},t^{n})$.
\end{remark}
\begin{prob}[2D-FV scheme]
	\label{chp3-sec2-prob3}
	Assume the framework defined in Problem \ref{chp3-sec2-prob2}.
	The finite-volume approach of Problem \ref{chp3-sec2-prob1}
	consists of a finding a scheme of the form:
	\begin{align}
		\label{chp3-2dfv}
		{Q}_{ij}^{n+1} =  {Q}_{ij}^{n} - {\lambda_x} \delta_i {F}_{i,j}^{n} - {\lambda_y } \delta_j {G}_{i,j}^{n},
		\\ \nonumber \quad \forall i = 1, \cdots, N, \quad \forall j = 1, \cdots, M,
		\quad \forall n = 0, \cdots, N_T-1,
	\end{align}
	where $ \delta_i F_{ij}^n =
    {F}_{i+\frac{1}{2},j}^{n} 
    - {F}_{i-\frac{1}{2},j}^{n}$,
    $ \delta_j G_{ij}^n =
    {G}_{i,j+\frac{1}{2}}^{n} 
    - {G}_{i,j-\frac{1}{2}}^{n}$ 
    and ${Q}^{n}$ is intended to be an approximation
	of ${Q}(t_{n})$ in some sense. We define ${Q}_{ij}^{0} = {Q}_{ij}(0)$ or
	${Q}_{ij}^{0} = {q}^0_{i,j}$.
    
    The term ${F}_{i+\frac{1}{2}, j}^{n} = {\mathbb{F}}
    (Q^n, \tilde{u}^n, \tilde{v}^n; i,j)
    $ is known as numerical flux in the 
    $x$ direction, where $\mathbb{F}$ is the numerical flux function, 
    and it approximates
	$\frac{1}{\Delta t}\int_{t_n}^{t_{n+1}} 
    \int_{y_{j-\frac{1}{2}}}^{y_{j+\frac{1}{2}}} 
    (uq)(x_{i+\frac{1}{2}}, y, t) \,dy \,dt $,
    $\forall i = 0, 1, \cdots, N$, and 
	${G}_{i, j+\frac{1}{2}}^{n} = 
    {\mathbb{G}} (Q^n, \tilde{u}^n, \tilde{v}^n; i, j)$ 
    is known as numerical flux in the 
    $y$ direction, where $\mathbb{G}$ 
    is the numerical flux function, and it approximates
	$\frac{1}{\Delta t}\int_{t_n}^{t_{n+1}}  
    \int_{x_{i-\frac{1}{2}}}^{x_{i+\frac{1}{2}}}
    (vq)(x, y_{j+\frac{1}{2}}, t) \,dx \,dt $,
    $\forall j = 0, 1, \cdots, M$,
	or, in other words, they estimate the time-averaged
    fluxes at the control volume $\Omega_{ij}$ boundaries.
  	The values $\tilde{u}_{i+\frac{1}{2},j}^n$ and $\tilde{v}_{i,j+\frac{1}{2}}^n$ are related to the time-averaged velocities
  	and depend on values of $u_{i+\frac{1}{2},j}^n$ and $v_{i,j+\frac{1}{2}}^n$.
\end{prob}
\begin{remark}
	A scheme of the form from Equation \eqref{chp3-2dfv} is referred to as a 2D-FV scheme and
	it is also known as a conservative scheme.
\end{remark}
\begin{definition}[Discrete divergence]
	\label{chp3-def-div}
	For Problem \ref{chp3-sec2-prob3}, we define the discrete divergence as a grid function $\mathbb{D}^n=\mathbb{D}^n(Q,\tilde{u}^n,\tilde{v}^n)$
	given by
	\begin{equation}
		\label{chp3-def-div-eq}
		\mathbb{D}_{ij}^n=\bigg( \frac{\mathbb{F}(Q,\tilde{u}^n,\tilde{v}^n;i,j) - \mathbb{F}(Q,\tilde{u}^n,\tilde{v}^n;i-1,j)}{\Delta x} \bigg) +
				 \bigg( \frac{\mathbb{G}(Q,\tilde{u}^n,\tilde{v}^n;i,j) - \mathbb{G}(Q,\tilde{u}^n,\tilde{v}^n;i,j-1)}{\Delta y} \bigg).
	\end{equation}
\end{definition}
For a 2D-FV the discrete total mass at the time-step $n$ is given by
\begin{equation*}
	M^n =  \Delta x \Delta y \sum_{i=1}^N \sum_{j=1}^M Q_{ij}^n.
\end{equation*}
Therefore, the discrete total mass is constant for a 2D-FV scheme,
which follows from a straightforward computation:
\begin{align*}
	M^{n+1} &=  \Delta x \sum_{i=1}^N  \sum_{j=1}^M Q_{ij}^{n+1} 
	= M^{n} - \Delta t  \sum_{i=1}^N  \sum_{j=1}^M (F^n_{i+\frac{1}{2},j}- F^n_{i-\frac{1}{2},j})
	 		 - \Delta t  \sum_{i=1}^N  \sum_{j=1}^M (G^n_{i,j+\frac{1}{2}}- G^n_{i,j-\frac{1}{2}})\\
	&= M^{n} - \Delta t \sum_{j=1}^M (F^n_{N+\frac{1}{2},j}- F^n_{\frac{1}{2},j})
			 - \Delta t \sum_{i=1}^N (G^n_{i,M+\frac{1}{2}}- G^n_{i,\frac{1}{2}})
	= M^{n},
\end{align*}
where we are using that $F^n_{N+\frac{1}{2},j} = F^n_{\frac{1}{2},j}$,
$G^n_{i,M+\frac{1}{2}} = G^n_{i,\frac{1}{2}}$ since we are assuming bi-periodic boundary
conditions.

\subsection{Convergence, consistency and stability}
\label{chp3-CCS}
As we mentioned in Problem \ref{chp3-sec2-prob3}, the initial condition may be assumed as $q_{ij}^0$ or $Q_{ij}(0)$. 
For two-dimensional simulations, we are going to assume  $q_{ij}^0$ as initial data to avoid the computation of integrals.
Furthermore, the errors will be calculated using the values $q_{ij}^n$ instead of $Q_{ij}(t_n)$.
Similarly to Proposition \ref{prop-bound-centroid}, we have that the centroid value approximates the average value
with second order, as Proposition \ref{prop-bound-centroid-2d} shows.
\begin{prop}
	\label{prop-bound-centroid-2d}
	If $q \in \mathcal{C}^2$, then $|Q_{ij}(t^n)-q_{ij}^n| \leq C_1 \Delta x^2 + C_2 \Delta x \Delta y + C_3 \Delta y^2$, where 
	$C_1$, $C_2$ and $C_3$ are constants.
\end{prop}
\begin{proof}
	Just apply Theorem \ref{prop-bound-midpoint2d} for the function $q(x,y,t^n)$.	
\end{proof}
The notions of convergence, consistency and stability for a 2D-FV schemes
are straightforward from these notions for 1D-FV schemes
(see Subsections \ref{chp2-sub-CC} and \ref{chp2-sub-stability}).
Indeed, in the context of Problem \ref{chp3-sec2-prob3}, we define the operators
$\mathcal{H}_{\Delta x ,\Delta y,n}: \mathbb{R}^{\Delta x \times \Delta y} \to \mathbb{R}^{\Delta x \times \Delta y}$ whose $(i,j)$ entry is given by:
\begin{align*}
	[\mathcal{H}_{\Delta x ,\Delta y,n}(Q)]_{ij} = Q_{ij} &-{\lambda_x} \bigg( \mathbb{F}(Q,\tilde{u}^n,\tilde{v}^n;i,j) - \mathbb{F}(Q,\tilde{u}^n,\tilde{v}^n;i-1,j) \bigg) \\ \nonumber
								 	   &-{\lambda_y} \bigg( \mathbb{G}(Q,\tilde{u}^n,\tilde{v}^n;i,j) - \mathbb{G}(Q,\tilde{u}^n,\tilde{v}^n;i,j-1) \bigg)
\end{align*}
for $i=1, \cdots, N$, $j=1, \cdots, M$, $n=0, \cdots, N_T-1$. The 2D-FV is then expressed as
\begin{equation*}
	Q^{n+1} = \mathcal{H}_{\Delta x ,\Delta y,n}(Q^n).
\end{equation*}
The local error truncation $\tau^n \in \mathbb{R}^{\Delta x \times \Delta y}$ is given by
\begin{equation*}
	Q(t^{n+1}) = \mathcal{H}_{\Delta x ,\Delta y,n}(Q(t^n)) + \Delta t \tau^n.
\end{equation*}
The error equation is given by
\begin{equation}
	E^{n+1} = \mathcal{H}_{\Delta x ,\Delta y,n}(Q(t^n)) - \mathcal{H}_{\Delta x ,\Delta y,n}(Q^n) +  \Delta t \tau^n.
\end{equation}
Given $r = (r_{ij})_{i=1,\cdots,N,j=1,\cdots,M}\in \mathbb{R}^{\Delta x \times \Delta y}$, we define the $p$-norm by
\begin{equation}
	\label{chp3-pnorm}
	\|r\|_{p,\Delta x \times \Delta y}=
	\begin{cases}
		\bigg( \sum_{i=1}^{N} \sum_{j=1}^{M}|r_{ij}|^p \bigg)^{\frac{1}{p}} & \text{if } 1\leq p < \infty,\\
		\max_{i=1, \cdots, N,j=1,\cdots,M}{|r_{ij}|} & \text{otherwise }.
	\end{cases}
\end{equation}
The stability in the $p$-norm is defined as in the 1D case.
\begin{definition}
	A 2D-FV scheme is stable in the $p-$norm if 
	\begin{equation}
		\|\mathcal{H}_{\Delta x ,\Delta y,n}(Q) - \mathcal{H}_{\Delta x ,\Delta y,n}(P)\|_{p,\Delta x \times \Delta y} \leq (1+\alpha \Delta t)  \|Q-P\|_{p,\Delta x \times \Delta y},
	\end{equation}
	for all $Q, P \in \mathbb{R}^{\Delta x \times \Delta y}$ and $\alpha$ is a constant
	that does not depend neither on $\Delta x$, $\Delta y$, $\Delta t$ nor on $n$.
\end{definition}
If a 2D-FV scheme is stable in the $p-$norm, similarly to Equation \eqref{chp2-sec2-erroreq} we have:
\begin{align*}
		\|E^{n+1}\|_{p,\Delta x \times \Delta y} &\leq e^{\alpha T}(\|E^0\|_{p,\Delta x \times \Delta y} + T\max_{n=1, \cdots, N_T}\|\tau^n\|_{p,\Delta x \times \Delta y}).\\
\end{align*}
Again, we point out that from Proposition \ref{prop-bound-centroid-2d}, we have that the initial error $E^0$ shall be second-order accurate.
Consistency is defined as in Definition \ref{chp2-def-cons} and convergence is defined as in Definition \ref{chp2-def-conv}.

The Von Neumann analysis can be applied when $\mathcal{H}_{\Delta x ,\Delta y,n}$ is linear, since we are considering periodic boundary conditions.
The idea is the same as in the one-dimensional case, we just apply the operator $\mathcal{H}_{\Delta x ,\Delta y,n}$ on the Fourier modes to obtain
the amplification factor.
We introduce the nodes $\theta_i = i\frac{2\pi}{N}$, $i=1, \cdots, N$, $\Delta \theta = \frac{2\pi}{N}$,
$\theta_i = (\theta_1, \theta_2, \cdots, \theta_N)$, $\phi_j = j\frac{2\pi}{M}$, $j=1, \cdots, M$, $\Delta \phi = \frac{2\pi}{M}$,
$\phi = (\phi_1, \phi_2, \cdots, \phi_M)$.
For $k_1=1, \cdots, N$, $k_2=1, \cdots, M$, the two-dimensional Fourier mode $\boldsymbol{k} = (k_1,k_2)$ from $\mathbb{C}^{N\times M}$ 
has its $(i,j)$ entry given by $[e^{\imath \boldsymbol{k} \theta}]_{ij} = e^{\imath k_1 \theta_i}e^{\imath k_2 \phi_j}$. 

Notice that if $q,u, v \in \mathcal{C}^3$, we can rewrite Equation \eqref{consistency-1d-eq2} as:
\begin{align*}
	\begin{split}
		\tau_{ij}^n &= 
		\bigg[ \frac{1}{\Delta x \Delta y \Delta t}  \int_{t^{n}}^{t^{n+1}}\int_{x_{i-\frac{1}{2}}}^{x_{i+\frac{1}{2}}} 
		\int_{y_{j-\frac{1}{2}}}^{y_{j+\frac{1}{2}}} {\nabla \cdot (\boldsymbol{u}q)}(x, y, t) \,dy \,dx \,dt 
		-\bigg( \frac{\mathbb{F}(Q,\tilde{u}^n,\tilde{v}^n;i,j) - \mathbb{F}(Q,\tilde{u}^n,\tilde{v}^n;i-1,j)}{\Delta x} \bigg) \\
		&-\bigg( \frac{\mathbb{G}(Q,\tilde{u}^n,\tilde{v}^n;i,j) - \mathbb{G}(Q,\tilde{u}^n,\tilde{v}^n;i,j-1)}{\Delta y} \bigg)
		\bigg].
	\end{split}
\end{align*}
Using the midpoint rule for integration (Theorem \ref{prop-bound-midpoint2d}), the mean value theorem for integrals
(Theorem \ref{anexo-numint-mv}) and recalling the discrete divergence (Definition \ref{chp3-def-div}), we have:
\begin{align}
	\begin{split}
		\label{consistency-2d-eq}
		\tau_{ij}^n 
		&= \frac{1}{\Delta t}  \int_{t^{n}}^{t^{n+1}}
		{\nabla \cdot (\boldsymbol{u}q)}(x_i, y_j, t)  \,dt - 
		\mathbb{D}^n_{ij} + O(\Delta x^2) + O(\Delta y^2).
	\end{split}
\end{align}
Therefore, in order to investigate the consistency, we may compare how well the discrete divergence approximates the divergence.
\section{Dimension splitting}
\label{sec-dsplit}
Before introducing the dimension splitting scheme from \citet{lin:1996}, it is useful to 
investigate a little bit of general operator splitting schemes, since the dimension splitting technique
is a particular case of operator splitting methods.
For a time interval $[0,T]$, we consider a $\Delta t-$temporal grid.
We consider the abstract Cauchy problems
\begin{align*}
	\begin{cases}
		\frac{dq}{dt}(t) &= Aq(t), \quad t \in [t^n,t^{n+1}],\\
		q(t^n) &= q_n,
	\end{cases}
\end{align*}
for $n=0,\cdots, N_T-1$, where $q(t) \in \mathcal{B}$ for some Banach space $\mathcal{B}$, and $A:\mathcal{B} \to \mathcal{B}$ is a linear operator
following the framework of \citet[Chapter~3]{richtmyer:1968}.
We are interested in finding $q(t^{n+1})$ given $q_n$.
Assuming that $A = A_1 + A_2$ for two linear operators $A_1, A_2:\mathcal{B} \to \mathcal{B}$, we consider the following abstract Cauchy sub-problems:
\begin{align*}
	\begin{cases}
		\frac{dq^1}{dt}(t) &= A_1q(t), \quad t \in [t^{n},t^{n+1}],\\
		q^{1}(t^n) &= q_n,
	\end{cases}
\end{align*}
and
\begin{align*}
	\begin{cases}
		\frac{dq^{21}}{dt}(t) &= A_2q(t), \quad t \in [t^n,t^{n+1}],\\
		q^{21}(t^n) &= q^1(t^{n+1}).
	\end{cases}
\end{align*}
Then we can approximate $q(t_0 + \Delta t)$ by $q^{21}(t^n + \Delta t)$ with error $O(\Delta t)$, if $A_1$ and $A_2$
do not commute; otherwise, this method is exact.
This approach is known as Lie-Trotter splitting. Observe the  Lie-Trotter splitting may be performed
in a reverse order solving the sub-problems
\begin{align*}
	\begin{cases}
		\frac{dq^2}{dt}(t) &= A_2q(t), \quad t \in [t^n,t^{n+1}],\\
		q^{2}(t^n) &= q_n,
	\end{cases}
\end{align*}
and 
\begin{align*}
	\begin{cases}
		\frac{dq^{21}}{dt}(t) &= A_1q(t), \quad t \in [t^n,t^{n+1}],\\
		q^{12}(t^n) &= q^1(t^{n+1}),
	\end{cases}
\end{align*}
and again we estimate $q(t^{n+1})$ by $q^{12}(t^{n+1})$ with error $O(\Delta t)$.
As observed by \citet{strang:1968}, we can consider
\begin{equation}
	q^*(t^{n+1}) = \frac{q^{21}(t^{n+1}) + q^{12}(t^{n+1})}{2},
\end{equation}
which is a second-order ($O(\Delta t^2)$) symmetric scheme to approximate $q(t^{n+1})$. 
We shall refer to this scheme as average Lie-Trotter splitting.
This process of averaging two Lie-Trotter splitting is a particular case of methods known in the literature as 
weighted sequential splitting methods. 
Furthermore, this process of averaging schemes may be extended to achieve higher-order schemes (c.f., e.g. \citet{jia:2011}).
For an accuracy analysis of weighted sequential splitting methods we refer to \citet{csomos:2005}.

We point out that one of the most widely used in the literature second-order splitting schemes is the Strang splitting \citep{strang:1968}.
This scheme requires the solution of 3 sub-problems per time-step, with one of them at time $t_n+\frac{\Delta t}{2}$,
while the average Lie-Trotter splitting requires the solution of 4 sub-problems per time-step.
Hence, the Strang splitting is computationally cheaper. 
However, as we shall see in this Chapter, the average Lie-Trotter splitting applied for the linear advection equation
allows a modification that eliminates a splitting error that appears when we consider a constant
scalar field and non-divergent velocity as observed by \citet{lin:1996}.

To move towards to the scheme from  \citet{lin:1996}, let us consider Problem \ref{chp3-sec2-prob1}
in its differential form:
\begin{equation*}
	\label{chp3-adv2deq-xydir}
	\begin{cases}
		\frac{\partial q}{\partial t}(x, y, t) +
		\frac{\partial (uq)}{\partial x}(x, y, t) +
		\frac{\partial (vq)}{\partial y}(x, y, t) 
		= 0, \quad \forall (x, y, t) \in \mathbb{T}^2_T\\
		q(x,y,0) = q_0(x,y),
	\end{cases}
\end{equation*}
where $q,u,v \in \mathcal{C}^1{(\mathbb{T}^2_T)}$.
We are going to consider the one-dimensional advection equation in the $x$-direction
\begin{equation*}
	\label{chp3-adv2deq-xdir1}
	\frac{\partial q^x}{\partial t}(x, y, t) +
	\frac{\partial (uq^x)}{\partial x}(x, y, t)
	= 0,
\end{equation*}
for each $y = y_j$, and the one-dimensional advection equation in the $y$-direction
\begin{equation*}
	\label{chp3-adv2deq-ydir1}
	\frac{\partial q^y}{\partial t}(x, y, t) +
	\frac{\partial (vq^y)}{\partial y}(x, y, t)
	= 0,   
\end{equation*}
for each $x = x_i$. We shall assume that these problems are solved using a 1D-FV scheme as in Problem \ref{chp2-sec2-prob3}
with numerical flux functions $\mathcal{F}$ and $\mathcal{G}$, respectively. We introduce the auxiliary functions
\begin{align*}
	\mathbf{F}({{Q}_{ij}^n}) = -\lambda_x \big(\mathcal{F} (Q^n_{ij}(\mathcal{S}_{i+\frac{1}{2}}); \tilde{u}^n_{i+\frac{1}{2},j})- 
	\mathcal{F} (Q^n_{ij}(\mathcal{S}_{i-\frac{1}{2}}); \tilde{u}^n_{i-\frac{1}{2},j})\big).
\end{align*}
and
\begin{align*}
	\mathbf{G}({Q}_{ij}^n) = -\lambda_y \big( \mathcal{G} (Q^n_{ij}(\mathcal{S}_{j+\frac{1}{2}}); \tilde{v}^n_{i,j+\frac{1}{2}})- 
	\mathcal{G} (Q^n_{ij}(\mathcal{S}_{j-\frac{1}{2}}); \tilde{v}^n_{i,j-\frac{1}{2}}) \big),
\end{align*}
which are the numerical flux update of the 1D-FV schemes in the $x$ and $y$ direction, respectively, that is
\begin{align*}
	\mathcal{F} (Q^n_{ij}(\mathcal{S}_{i+\frac{1}{2}}); \tilde{u}^n_{i+\frac{1}{2},j}) = \frac{1}{\Delta t} \int_{t_n}^{t^{n+1}} (uq)(x_{i+\frac{1}{2}},y_j,t) \,dt + O(\Delta t^P)  ,\\
 	\mathcal{G} (Q^n_{ij}(\mathcal{S}_{j+\frac{1}{2}}); \tilde{v}^n_{i,j+\frac{1}{2}}) = \frac{1}{\Delta t} \int_{t_n}^{t^{n+1}} (vq)(x_i,y_{j+\frac{1}{2}},t) \,dt + O(\Delta t^P).
\end{align*}
where $P$ is the order of accuracy of the flux.
The Lie-Trotter splitting is obtained by solving the advection in the $x$ direction
\begin{align*}
	{Q}_{ij}^{x,n+1} =  {Q}_{ij}^{n} + \mathbf{F}({Q_{ij}^n}),
\end{align*}
for $j=1, \cdots, M$, and then we advect in the $y$ direction with initial data ${Q}_{ij}^{x,n+1}$ 
\begin{align*}
	{Q}_{ij}^{yx,n+1} = Q_{ij}^{x,n+1} + \mathbf{G}({Q_{ij}^{x,n+1}}),
\end{align*}
for $i=1, \cdots, N$.
To get the average Lie-Trotter splitting we repeat the process in the reverse order by solving the advection equation
in the $y$ direction
\begin{align*}
	{Q}_{ij}^{y,n+1} =  {Q}_{ij}^{n} + \mathbf{G}({Q_{ij}^n}),
\end{align*}
for $i=1, \cdots, N$, and then we advect in the $x$-direction with initial data ${Q}_{ij}^{y,n+1}$ 
\begin{align*}
	{Q}_{ij}^{xy,n+1} = Q_{ij}^{y,n+1} + \mathbf{F}({Q_{ij}^{y,n+1}}),
\end{align*}
for $j=1, \cdots, M$, and thus we have the average Lie-Trotter solution:
\begin{align*}
	Q^{n+1}_{ij} = \frac{(Q^{xy,n+1} + Q^{yx,n+1})}{2} = Q_{ij}^n + \frac{1}{2}\mathbf{F}(Q_{ij}^n) + \frac{1}{2}\mathbf{G}(Q_{ij}^n) + \frac{1}{2}\mathbf{F}\bigg(Q_{ij}^n + \frac{1}{2}\mathbf{G}(Q_{ij}^n)\bigg) + \frac{1}{2}\mathbf{G}\bigg(Q_{ij}^n + \frac{1}{2}\mathbf{F}(Q_{ij}^n)\bigg),
\end{align*}
assuming that the numerical flux functions are linear in the input $Q$, we may rewrite a computationally cheaper version
of the average Lie-Trotter splitting as \citep{lin:1996}:
\begin{align}
	\label{chp3-avlt}
	Q^{n+1}_{ij} = \frac{(Q^{xy,n+1} + Q^{yx,n+1})}{2} = Q_{ij}^n +  
	\mathbf{F}\bigg(Q_{ij}^n + \frac{1}{2}\mathbf{G}(Q_{ij}^n)\bigg) +  
	\mathbf{G}\bigg(Q_{ij}^n + \frac{1}{2}\mathbf{F}(Q_{ij}^n)\bigg).
\end{align}
The numerical flux functions defined in Chapter \ref{chp-1d-fv} are indeed linear the input $Q$ if there are monotonic constrain, 
as we mention in Chapter \ref{chp-1d-fv}, but we are going to consider this scheme even when there are monotonic constraints since it requires fewer operations.
For a while let us assume that the numerical flux functions are exactly the time-averaged fluxes, which they must approximate, as pointed out in Problem \ref{chp2-sec2-prob3}. 
Under this assumption, we have:
\begin{align*}
	\mathbf{F}(Q_{ij}^n) = -\lambda_x{\delta_x}\bigg(\frac{1}{\Delta t} \int_{t_n}^{t^{n+1}} (uq)(x_i,y_j,t) \,dt \bigg) ,\\
	\mathbf{G}(Q_{ij}^n) = -\lambda_y{\delta_y}\bigg(\frac{1}{\Delta t} \int_{t_n}^{t^{n+1}} (vq)(x_i,y_j,t) \,dt \bigg).
\end{align*}
Further, if we if assume that $q = \overline{q}$ is constant and $\nabla \cdot \boldsymbol{u} = 0$ then the solution
remains constant and then, assuming also that $\boldsymbol{u}$ does not depend on $t$, then $\mathbf{F}$ and $\mathbf{G}$ are given by
\begin{align*}
	\mathbf{F}(Q_{ij}^n) = -\overline{q} \lambda_x{\delta_x} u(x_i,y_j),\\
	\mathbf{G}(Q_{ij}^n) = -\overline{q} \lambda_y{\delta_y} v(x_i,y_j).
\end{align*}
However, if we compute the updated solution using Equation \eqref{chp3-avlt}, we have that the error is given by
\begin{align*}
	Q^{n+1}_{ij} - \overline{q} &= 
	-\Delta t \bigg(\frac{\delta_x u(x_i,y_j)}{\Delta x} + 	\frac{\delta_y v(x_i,y_j)}{\Delta y} \bigg)
	-\Delta t^2 \overline{q}\bigg( \frac{\delta_y v \delta_x u(x_i,y_j) + \delta_x u\delta_y v(x_i,y_j)}{2\Delta x \Delta y} \bigg) \\
	&= \Delta t (O(\Delta x^2) + O(\Delta y^2))
	   -\Delta t^2 \overline{q}\bigg( \frac{\delta_y v \delta_x u(x_i,y_j) + \delta_x u\delta_y v(x_i,y_j)}{2\Delta x \Delta y} \bigg).
\end{align*}
Thus, the terms in the equation above multiplied by $\Delta t^2$ are related to an splitting error, even if we consider the exact fluxes.
Aiming to eliminate the error from, \citet{lin:1996} proposes to consider a modification of the average Lie-Trotter splitting as
\begin{align}
	\label{chp3-avlt2}
	Q^{n+1}_{ij} = Q_{ij}^n +  
	\mathbf{F}\bigg(Q_{ij}^n + \frac{1}{2}\mathbf{g}(Q_{ij}^n)\bigg) +  
	\mathbf{G}\bigg(Q_{ij}^n + \frac{1}{2}\mathbf{f}(Q_{ij}^n)\bigg),
\end{align}
where $\mathbf{f}$ and $\mathbf{g}$ are called inner advective operators and approximate
$-\Delta t u \frac{\partial q}{\partial x}$ and $-\Delta t v \frac{\partial q}{\partial y}$.

In this work, we shall consider the following  inner advective operators proposed by \citet{putman:2007}
and is currently used in the FV3 dynamical core:
\begin{align}
	\mathbf{f}(Q_{ij}) = \frac{1}{2},\\
	\mathbf{g}(Q_{ij}) = \frac{1}{2}.
\end{align}
Recalling the definition of discrete divergence (Definition \ref{chp3-def-div}) we have:
\begin{equation}
	\mathbb{D}_{ij}^n = 
	-\mathbf{F}\bigg(Q_{ij}^n + \frac{1}{2}\mathbf{g}(Q_{ij}^n)\bigg) 
	-\mathbf{G}\bigg(Q_{ij}^n + \frac{1}{2}\mathbf{f}(Q_{ij}^n)\bigg),
\end{equation}
and as pointed out in Section \ref{chp3-CCS}, we may use the discrete divergence to check the scheme consistency. 
\section{Numerical experiments}
\label{sec-ds-exp}

