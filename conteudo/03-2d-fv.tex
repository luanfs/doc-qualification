%!TeX root=../tese.tex
%("dica" para o editor de texto: este arquivo é parte de um documento maior)
% para saber mais: https://tex.stackexchange.com/q/78101

\chapter{Two-dimensional finite-volume methods}
\label{chp-2d-fv}

\section{Two-dimensional system of conservation laws in integral form}
\label{sec-conslaw2d}
Let us consider $\mathcal{C}^1$ flux functions
${f}: \mathbb{R}^m \to \mathbb{R}^m$ and
${g}: \mathbb{R}^m \to \mathbb{R}^m$  in $x$ and $y$ direction,
respectively. 
A two-dimensional system of conservation laws in the differential form in 
a domain $\Omega=[a,b]\times[c,d] \subset \mathbb{R}^2$
associated to the fluxes ${f}$ and ${g}$ is given by:
\begin{equation}
\label{sec-conslaw2d:eq1}
\frac{\partial}{\partial t}{q}(x, y, t) +
\frac{\partial}{\partial x} {f}({q}(x, y, t)) +
\frac{\partial}{\partial y} {g}({q}(x, y, t))
= 0, \quad \forall (x, y, t) \in \Omega^{\mathrm{o}}
 \times ]0, +\infty[. 
\footnote{$\Omega^{\mathrm{o}}$ denotes the interior of $\Omega$. 
	Namely, $\Omega^{\mathrm{o}} = ]a,b[ \times ]c,d[$.}
\end{equation}

The solution ${q}$ is interpreted as the vector of state variable
densities.
A classical or strong solution to this system of conservation laws is a 
$\mathcal{C}^1$ function ${q}$ satisfying Equation 
\eqref{sec-conslaw2d:eq1}.
As we did in Section \ref{chp1-sec1}, our goal is to deduce an
integral form of Equation \eqref{sec-conslaw2d:eq1}.
To do so, let us consider  $[x_1,x_2] \times [y_1, y_2]
\subset \Omega^{\mathrm{o}}$ and $[t_1,t_2] \subset [0, +\infty[$.
Integrating Equation $\eqref{sec-conslaw2d:eq1}$ over 
$[x_1,x_2] \times [y_1, y_2]$ yields:
\begin{align}
	\label{sec-conslaw2d:eq2}
	\frac{d}{d t} \bigg(\int_{x_1}^{x_2} \int_{y_1}^{y_2}
	{q}(x, y, t) \,dx \,dy \bigg)=
	&-\int_{y_1}^{y_2} \bigg({f}({q}(x_2, y, t))
	-{f}({q}(x_1, y, t)) \bigg) \,dy \\ \nonumber
	&-\int_{x_1}^{x_2} \bigg({g}({q}(x, y_2, t))
	-{g}({q}(x, y_1, t)) \bigg) \,dx.
\end{align}

Integrating Equation \eqref{sec-conslaw2d:eq2} over the time interval $[t_1,t_2]$, 
we have:

\begin{align}
	\label{sec-conslaw2d:eq3}
	\int_{x_1}^{x_2} \int_{y_1}^{y_2}
	{q}(x, y, t_{n+1}) \,dx \,dy = &\int_{x_1}^{x_2} \int_{y_1}^{y_2}
	{q}(x, y, t_n) \,dx \,dy \\ \nonumber
	&-\int_{t_1}^{t_2} \int_{y_1}^{y_2} \bigg({f}({q}(x_2, y, t))
	-{f}({q}(x_1, y, t)) \bigg) \,dy \,dt\\ \nonumber
	&-\int_{t_1}^{t_2} \int_{x_1}^{x_2} \bigg({g}({q}(x, y_2, t))
	-{g}({q}(x, y_1, t)) \bigg) \,dx \,dt.
\end{align}

Equation \eqref{sec-conslaw2d:eq3} is the integral form of Equation 
\eqref{sec-conslaw2d:eq1}. We say that ${q} \in 
L^{\infty}{(\Omega \times [0, +\infty[}, \mathbb{R}^m)$ is a weak
solution to the system of conservation laws  \eqref{sec-conslaw2d:eq1} if ${q}$
satisfies the integral form \eqref{sec-conslaw2d:eq3}, 
$\forall [x_1,x_2]\times[y_1,y_2] \subset \Omega^{\mathrm{o}}$ and 
$\forall [t_1,t_2] \subset [0,+\infty[$.
Similarly to Section \ref{chp1-sec1}, these problems are equivalent
when  ${q}$ is a $\mathcal{C}^1$ function.

We consider an initial condition ${q_0} \in L^{\infty}{(\Omega)}$,
${q}(x,y,0) =  {q_0}(x,y)$, $\forall (x,y) \in \Omega$.
Boundary conditions will be assumed bi-periodic.
At last, the matrix $\alpha D{f}(q) + \beta D{g}(q)$ is assumed to have
real eigenvalues and be diagonalizable
$\forall q \in \mathbb{R}^m , \forall \alpha, \beta \in \mathbb{R}$
\citep{leveque:1990}, so that we have a hyperbolic conservation law.
Therefore, we are again dealing with a Cauchy problem. 

To move in the direction of a discrete version of Equation \eqref{sec-conslaw2d:eq3},
let us discretize the domain $D = \Omega \times [0,T]$ following 
the notations of Section \ref{chp1-sec1}.
Given a positive integer $N_T$, we define the time step 
$\Delta t = \frac{T}{N_T}$, $t_n = n \Delta t$, for $n = 0, 1 ,\cdots, N_T$.
The spatial discretization is constructed through an uniformly spaced partition of $\Omega$ given by:
\begin{align}
	[a,b] &= \bigcup_{i=1}^N X_i, 
	\text{ where } X_i= [x_{i-\frac{1}{2}}, x_{i+\frac{1}{2}}] \text{ and } 
	a = x_{\frac{1}{2}} < x_{\frac{3}{2}} < \cdots < x_{N-\frac{1}{2}} < x_{N+\frac{1}{2}} = b, \\
	[c,d] &= \bigcup_{j=1}^M Y_j, 
\text{ where } Y_j= [y_{j-\frac{1}{2}}, y_{j+\frac{1}{2}}] \text{ and } 
c = y_{\frac{1}{2}} < y_{\frac{3}{2}} < \cdots < y_{M-\frac{1}{2}} < y_{M+\frac{1}{2}} = d, \\
    \Omega &=  \bigcup_{i=1}^N \bigcup_{j=1}^M \Omega_{ij}, \text{ where } \Omega_{ij} = X_i \times Y_j.
\end{align}

The regions $\Omega_{ij}$ are known as control volumes. \index{Control volume}
Similarly to  Chapter \ref{chp1-1d-fv} we employ the notations
$\Delta x = x_{i+\frac{1}{2}} - x_{i-\frac{1}{2}}$,
$\Delta y = y_{j+\frac{1}{2}} - y_{j-\frac{1}{2}}$ 
and $x_i = \frac{1}{2}(x_{i+\frac{1}{2}} + x_{i-\frac{1}{2}})$
$y_j = \frac{1}{2}(y_{j+\frac{1}{2}} + y_{j-\frac{1}{2}})$, $\forall i = 1, \cdots, N$, 
$\forall j = 1, \cdots, M$,
to define the control volume lengths and midpoints, respectively.
Finally, we denote by ${Q}_{ij}(t) \in \mathbb{R}^m$ as the vector of 
average values of state variable vector at time $t$
in the control volume $\Omega_{ij}$, that is:
\begin{equation}
	\label{sec-conslaw2d:eq4}
	{Q}_{ij}(t) = \frac{1}{\Delta x \Delta y}
	\int_{x_{i-\frac{1}{2}}}^{x_{i+\frac{1}{2}}} \int_{y_{j-\frac{1}{2}}}^{y_{j-\frac{1}{2}}} {q}(x,t) \,dx
	\in \mathbb{R}^m.
\end{equation}

Substituting $t_1, t_2, x_1, x_2, y_1$ and $y_2$ by 
$t_{n}, t_{n+1}, x_{i-\frac{1}{2}}, x_{i+\frac{1}{2}}, y_{j-\frac{1}{2}}, y_{j+\frac{1}{2}}$,
respectively, in Equation \eqref{sec-conslaw2d:eq3}, we obtain:
\begin{align}
	\label{sec-conslaw2d:eq5}
	{Q}_{ij}(t_{n+1})  = {Q}_{ij}(t_{n})
	&- \frac{\Delta t}{\Delta x \Delta y}
	\delta _x \bigg( \frac{1}{\Delta t}
	\int_{t_1}^{t_2} \int_{y_{j-\frac{1}{2}}}^{y_{j+\frac{1}{2}}} 
	{f}({q}(x_{i}, y, t))
	\,dy \,dt \bigg) \\ \nonumber
	&- \frac{\Delta t}{\Delta x \Delta y}
	\delta _y \bigg( \frac{1}{\Delta t}
	\int_{t_1}^{t_2} \int_{x_{i-\frac{1}{2}}}^{x_{i+\frac{1}{2}}} 
	{g}({q}(x, y_{j}, t))
	\,dx \,dt \bigg),
\end{align}
where we are using the centered finite-difference notation:
\begin{align}
	\label{sec-conslaw2d:eq6}
	\delta_x {h}(x_i,y, t) = 
	{h}(x_{i+\frac{1}{2}}, y, t) - 
	{h}(x_{i-\frac{1}{2}}, y, t), \\
	\delta_y {h}(x, y_j,t) = 
    {h}(x, y_{j+\frac{1}{2}},t) - 
    {h}(x, y_{j-\frac{1}{2}},t),
\end{align}
for any function ${h}$. The Equation \eqref{sec-conslaw2d:eq5} is useful to
motivate two-dimensional finite-volume schemes, as we shall see in the next section.

\section{The finite-volume approach}
\label{sec:fv-2d}
This Section is basically an extension to two dimensions 
of the concepts presented in Section \ref{chp1-sec2}.
The problem of two-dimensional system of conservation laws in the integral form 
presented Section \ref{sec-conslaw2d} is written in a concise way in Problem \ref{sec:fv-2d-prob1}.

\begin{prob}
	\label{sec:fv-2d-prob1}
	Given $\Omega = [a,b]\times [c,d]$, $D = \Omega \times [0,T]$, 
	$\mathcal{C}^1$ flux functions ${f},
	{g}: \mathbb{R}^m \to \mathbb{R}^m$,
	$m \geq 1$, we would like to find the weak solution
	$ {q} \in L^{\infty}(D, \mathbb{R}^m)$ 
	of the two-dimensional system of conservation laws in the integral form:
	\begin{align*}
		\int_{x_1}^{x_2} \int_{y_1}^{y_2}
		{q}(x, y, t) \,dx \,dy = &\int_{x_1}^{x_2} \int_{y_1}^{y_2}
		{q}(x, y, t) \,dx \,dy \\ \nonumber
		&-\int_{t_1}^{t_2} \int_{y_1}^{y_2} \bigg({f}({q}(x_2, y, t))
		-{f}({q}(x_1, y, t)) \bigg) \,dy \,dt\\ \nonumber
		&-\int_{t_1}^{t_2} \int_{x_1}^{x_2} \bigg({g}({q}(x, y_2, t))
		-{g}({q}(x, y_1, t)) \bigg) \,dx \,dt.
	\end{align*}
	$\forall [x_1, x_2]\times [y_1, y_2] \times[t_1, t_2] \subset D$, 
	given the initial condition 
	${q}(x, y, 0) = {q}_0(x, y)$, $\forall (x, y) \in \Omega$, 
	and assuming bi-periodic boundary conditions, \textit{i.e.}, 
	${q}(a,y,t) = {q}(b,y,t)$, $\forall t \in [0,T]$, $\forall y \in [c,d]$, and
	${q}(x,c,t) = {q}(x,d,t)$, $\forall t \in [0,T]$, $\forall x \in [a,b]$.
\end{prob}

For Problem \ref{sec:fv-2d-prob1}, the total mass in $\Omega$ is defined by: 
\begin{equation}
	{M}_{\Omega}(t) = \int_{\Omega} {q}(x,y,t) \,dx \,dy  \in \mathbb{R}^m, \quad \forall t \in [0,T],
\end{equation}
and is conserved within time: 
\begin{equation}
	{M}_{\Omega}(t) = {M}_{\Omega}(0), \quad \forall t \in [0,T].
\end{equation}

Section \ref{sec-conslaw2d} introduced a version of Problem \ref{sec:fv-2d-prob1}
considering a discretization of the domain $D$. 
This version is also summarized in Problem \ref{sec:fv-2d-prob2}.
\begin{prob}
	\label{sec:fv-2d-prob2}
	Assume the framework of Problem \ref{sec:fv-2d-prob1}.
	We consider positive integers $N$ and $N_T$, a spatial discretization of [a,b] given by
	$X_i = [x_{i-\frac{1}{2}}, x_{i+\frac{1}{2}}]$,
	$\forall i = 1, \cdots, N,$ 
	$a = x_{\frac{1}{2}} < x_{\frac{3}{2}} < \cdots < x_{N-\frac{1}{2}} < x_{N+\frac{1}{2}} = b$,
	$\Delta x = x_{i+\frac{1}{2}}-x_{i-\frac{1}{2}}$, $Y_j = [y_{j-\frac{1}{2}}, y_{j+\frac{1}{2}}]$,
	$\forall j = 1, \cdots, M,$ 
	$c = y_{\frac{1}{2}} < y_{\frac{3}{2}} < \cdots < y_{M-\frac{1}{2}} < y_{M+\frac{1}{2}} = d$,
	$\Delta y = y_{j+\frac{1}{2}}-y_{j-\frac{1}{2}}$,
	$\Omega_{ij} = X_i \times Y_j$,
	a time discretization
	$t_n = n\Delta t$, $\Delta t = \frac{T}{N_T}$, $\forall n = 1, \cdots, N_T$.
	Since we are in the framework of Problem \ref{sec:fv-2d-prob2}, it follows that:
	\begin{align*}
		{Q}_{ij}(t_{n+1})  = {Q}_{ij}(t_{n})
		&- \frac{\Delta t}{\Delta x \Delta y}
		\delta _x \bigg( \frac{1}{\Delta t}
		\int_{t_1}^{t_2} \int_{y_{j-\frac{1}{2}}}^{y_{j+\frac{1}{2}}} 
		{f}({q}(x_{i}, y, t))
		\,dy \,dt \bigg) \\ \nonumber
		&- \frac{\Delta t}{\Delta x \Delta y}
		\delta _y \bigg( \frac{1}{\Delta t}
		\int_{t_1}^{t_2} \int_{x_{i-\frac{1}{2}}}^{x_{i+\frac{1}{2}}} 
		{g}({q}(x, y_{j}, t))
		\,dx \,dt \bigg),
	\end{align*}
	where ${Q}_{ij}(t) = \frac{1}{\Delta x \Delta y}
	\int_{x_{i-\frac{1}{2}}}^{x_{i+\frac{1}{2}}} 
	\int_{y_{j-\frac{1}{2}}}^{y_{j+\frac{1}{2}}} {q}(x,y,t) \,dx \,dy$.
	
	Our problem now consists of finding the values ${Q}_{ij}(t_{n})$, 
	$\forall i = 1, \cdots, N$, $\forall j = 1, \cdots, M$, $\forall n = 1, \cdots, N_T$,
    given the initial values ${Q}_{ij}(0)$, $\forall i = 1, \cdots N$, $\forall j = 1, \cdots, M$.
	In other words, we would like to find the average values of ${q}$
	in each control volume $\Omega_{ij}$ at the considered time instants.
\end{prob}

Finally, we define the one-dimensional (2D) finite-volume (FV)
scheme problem as follows in Problem \ref{sec:fv-2d-prob3}.
We use the notation ${q}^n_{ij} = {q}(x_i, y_j, t_n)$
to represent the values of ${q}$ in the discrete domain $D$
and $u_{i+\frac{1}{2},j}^n = u(x_{i+\frac{1}{2}}, y_j, t_n)$
to represent the velocity in $x$ direction at control volume
edges midpoints in the $x$ direction, 
and $v_{i,j+\frac{1}{2}}^n = v(x_i,y_{j+\frac{1}{2}},t_n)$
to represent the velocity in $y$ direction at control volume
edges midpoints in the $y$ direction, 
\begin{prob}[2D-FV scheme]
	\label{sec:fv-2d-prob3}
	Assume the framework defined in Problem \ref{sec:fv-2d-prob2}.
	The finite-volume approach of Problem \ref{sec:fv-2d-prob1}
	consists of a finding a scheme of the form:
	\begin{align*}
		{Q}_{ij}^{n+1} =  {Q}_{ij}^{n} -
		\frac{\Delta t}{\Delta x \Delta y} 
        \delta_i {F}_{i,j}^{n} - 
		\frac{\Delta t}{\Delta x \Delta y}
        \delta_j {G}_{i,j}^{n},
		\\ \nonumber \quad \forall i = 1, \cdots, N, \quad \forall j = 1, \cdots, M,
		\quad \forall n = 1, \cdots, N_T-1,
	\end{align*}
	where $ \delta_i F_{ij}^n =
    {F}_{i+\frac{1}{2},j}^{n} 
    - {F}_{i-\frac{1}{2},j}^{n}$,
    $ \delta_j G_{ij}^n =
    {G}_{i,j+\frac{1}{2}}^{n} 
    - {G}_{i,j-\frac{1}{2}}^{n}$ 
    and ${Q}_{ij}^{n} \in \mathbb{R}^m$ is intended to be an approximation
	of ${Q}_{ij}(t_{n})$ in some sense. We define by ${Q}_{ij}^{0} = {Q}_{ij}(0)$ or
	${Q}_{ij}^{0} = {q}^0_{i,j}$.
    
    The term ${F}_{i+\frac{1}{2}, j}^{n} = {\mathcal{F}}
    (Q^n; u^n; i,j)
    $ is known as numerical flux in the 
    $x$ direction, where $\mathcal{F}$ is the numerical flux function, 
    and it approximates
	$\frac{1}{\Delta t}\int_{t_n}^{t_{n+1}} 
    \int_{y_{j-\frac{1}{2}}}^{y_{j+\frac{1}{2}}} 
    {f}({q}(x_{i+\frac{1}{2}}, y, t)) \,dy \,dt $,
    $\forall i = 0, 1, \cdots, N$,
    and 
	${G}_{i, j+\frac{1}{2}}^{n} = 
    {\mathcal{G}} (Q^n, v^n; i, j)$ 
    is known as numerical flux in the 
    $y$ direction, where $\mathcal{G}$ 
    is the numerical flux function, and it approximates
	$\frac{1}{\Delta t}\int_{t_n}^{t_{n+1}}  
    \int_{x_{i-\frac{1}{2}}}^{x_{i+\frac{1}{2}}}
    {g}({q}(x, y_{j+\frac{1}{2}}, t)) \,dx \,dt $,
    $\forall j = 0, 1, \cdots, M$,
	or, in other words, they estimate the time-averaged
    fluxes at the control volume $\Omega_{ij}$ boundaries.
\end{prob}
$Q^n = (Q_{ij}^n)_{i=1,\cdots, N, j = 1, \cdots, M}$

\section{Dimension splitting}
\label{sec-dimsplit}


\citep{lin:1997}
\citep{lin:2004}
\citep{putmanthesis:2007}
\citep{putman:2007}
