\chapter{Numerical Analysis}
\label{anexo-na}

\section{Finite-difference estimates}
\label{anexo-fd}
This Section aims to prove all finite-difference error estimations 
used throughout this text.
All the proves are very simple and consist of applying Taylor's expansions,
as it is usual when computing the accuracy order of many numerical schemes.
%\theoremstyle{plain} % As próximas definições usam este estilo
%\newtheorem{lema-fd}{Lemma}

\begin{lema}
	\label{lemma:fd-ppm-est1}
	Let $F \in \mathcal{C}^{5}(\mathbb{R})$, $x_0 \in \mathbb{R}$ and $h>0$.
	Then, the following identity holds:
	\begin{equation}	
		\label{lemma:fd-ppm-est1-eq0}
		F'(x_0) =  \frac{4}{3} \bigg(\frac{F(x_0+h) - F(x_0-h)}{2h}\bigg)
                      - \frac{1}{3} \bigg(\frac{F(x_0+2h) - F(x_0-2h)}{4h}\bigg)
		      + C_1h^4,
	\end{equation}
	where $C_1$ is a constant that depends only on $F$ and $h$.
\end{lema}

\begin{proof}
	Given $\delta \in ]0,2h]$, then $x_0 + \delta \in ]x_0,x_0+2h]$ and $x_0 - \delta \in ]x_0-2h,x_0]$.
	Then, we get using the Taylor expansion of $F$:
	\begin{align*}
		F(x_0 + \delta) = F(x_0) +  &F'(x_0)\delta + F^{(2)}(x_0)\frac{\delta ^2}{2} 
		+ F^{(3)}(x_0)\frac{\delta ^3}{3!}
		+ F^{(4)}(x_0)\frac{\delta ^3}{4!}
		+ F^{(5)}(\theta_{\delta} )\frac{\delta ^5}{5!}
		\quad \theta_{\delta} \in [x_0,x_0+\delta ],\\
		F(x_0 - \delta) = F(x_0) -&F'(x_0)\delta  + F^{(2)}(x_0)\frac{\delta ^2}{2}
		- F^{(3)}(x_0)\frac{\delta ^3}{3!}
		+ F^{(4)}(x_0)\frac{\delta ^4}{4!}
		- F^{(5)}(\theta_{-\delta})\frac{\delta ^5}{5!},		
		\quad \theta_{-\delta} \in [x_0-\delta ,x_0].
	\end{align*}
	
	Thus:
	\begin{equation}
		\label{lemma:fd-ppm-est1-eq1}
		\frac{F(x_0 + \delta ) - F(x_0 - \delta )}{2\delta } = F'(x_0) + 
		F^{(3)}(x_0)\frac{\delta^2}{3!} +
		\bigg( F^{(5)}(\theta_{\delta}) + 
		F^{5}(\theta_{-\delta}) \bigg) \frac{\delta ^4}{2\cdot5!} ,		
	\end{equation}
	
	Applying Equation \eqref{lemma:fd-ppm-est1-eq1} for $\delta  = h$ and 
	$\delta  = 2h$, we get, respectively:
	\begin{equation}
		\label{lemma:fd-ppm-est1-eq2}
		\frac{F(x_0 + h) - F(x_0-h)}{2h} = F'(x_0) +
		F^{(3)}(x_0)\frac{h^2}{3!} +
		\bigg(F^{(5)}(\theta_{h}) + F^{5)}(\theta_{-h})\bigg)\frac{h^4}{2\cdot5!}, 
		\quad \theta_{h} \in [x_0,x_0+h], \quad \theta_{-h}\in [x_0-h,x_0],
	\end{equation}
	and
	\begin{align}
	    \label{lemma:fd-ppm-est1-eq3}
		\frac{F(x_0+2h) - F(x_0-2h)}{4h} = F'(x_0) +
		F^{(3)}(x_0)\frac{4h^2}{3!} + 
		\bigg( F^{(5)}(\theta_{2h}) + F^{(5)}(\theta_{-2h})\bigg) \frac{16h^4}{2\cdot5!},
		\\ \nonumber
		\quad \theta_{2h} \in [x_0,x_0+2h], \quad \theta_{-2h}\in [x_0-2h,x_0].
	\end{align}

	Using Equations \eqref{lemma:fd-ppm-est1-eq2} and \eqref{lemma:fd-ppm-est1-eq3}, we obtain:
	\begin{align}
		\label{lemma:fd-ppm-est1-eq4}
		\frac{4}{3} \bigg(\frac{F(x_0+h) - F(x_0-h)}{2h}\bigg) &= \frac{4}{3} F'(x_0) +
				F^{(3)}(x_0)\frac{4h^2}{3\cdot3!} +
                \bigg( F^{(5)}(\theta_{h}) + F^{(5)}(\theta_{-h})\bigg)\frac{h^4}{2\cdot5!},\\
        \label{lemma:fd-ppm-est1-eq5}
	        	\frac{1}{3} \bigg(\frac{F(x_0+2h) - F(x_0-2h)}{4h}\bigg) &= \frac{1}{3} F'(x_0)+
		       	F^{(3)}(x_0)\frac{4h^2}{3\cdot3!} +
		        \bigg( F^{(5)}(\theta_{2h}) + F^{(5)}(\theta_{-2h})\bigg) \frac{16h^4}{3\cdot2\cdot5!} 
	\end{align}

	Subtracting Equation \eqref{lemma:fd-ppm-est1-eq5} from Equation \eqref{lemma:fd-ppm-est1-eq4} we get 
	the desired Equation \eqref{lemma:fd-ppm-est1-eq0} with
	\begin{equation}
		C_1 = \frac{1}{720}\bigg( 3F^{(5)}(\theta_{h}) + 3F^{(5)}(\theta_{-h})
            -16F^{(5)}(\theta_{2h}) - 16F^{(5)}(\theta_{-2h})\bigg), 
		%C = \frac{1}{240}\bigg( F^{(5)}(\theta_{h}) + F^{(5)}(\theta_{-h})\bigg)
		%- \frac{1}{45}\bigg( F^{(5)}(\theta_{2h}) + F^{(5)}(\theta_{-2h})\bigg), 
	\end{equation}
	where $\theta_{h} \in [x_0,x_0+h], \theta_{-h}\in [x_0-h,x_0]$, 
	$\theta_{2h} \in [x_0,x_0+2h], \theta_{-2h}\in [x_0-2h,x_0]$.
	Using the intermediate value theorem, we can express $C_1$ in a more compact way as
	\begin{equation}
	\label{lemma:fd-ppm-est1-eq6}
	C_1 = \frac{1}{720}\bigg( 6F^{(5)}(\eta_{1}) -32F^{(5)}(\eta_{2})\bigg), 
\end{equation}
where $\eta_{1}, \eta_{2} \in [x_0-2h,x_0+2h]$, which concludes the proof.
\end{proof}


\begin{lema}
	\label{lemma:fd-ppm-est2}
	Let $F \in \mathcal{C}^{4}(\mathbb{R})$, $x_0 \in \mathbb{R}$ and $h>0$.
	Then, the following identity holds:
	\begin{equation}	
		\label{lemma:fd-ppm-est2-eq0}
		F''(x_0) =  \frac{-2F(x_0-2h) + 15F(x_0-h) -28F(x_0) + 20F(x_0+h) -6F(x_0+2h) + F(x_0+3h)}{6h^2}
		      + C_2h^2,
	\end{equation}
	where $C_2$ is a constant that depends only on $F$ and $h$.
\end{lema}

\begin{proof}
From the Taylor's expansion, we have:
	\begin{align*}
		F(x_0-2h) &= F(x_0) - 2F'(x_0)h +           2F^{(2)}(x_0)h^2  - \frac{8}{6}F^{(3)}(x_0)h^3 + \frac{16}{24}F^{(4)}(\theta_{-2h})h^4,\\
		F(x_0-h)  &= F(x_0) -  F'(x_0)h +  \frac{1}{2}F^{(2)}(x_0)h^2 - \frac{1}{6}F^{(3)}(x_0)h^3 + \frac{1}{24}F^{(4)}(\theta_{-h})h^4, \\
		F(x_0+h)  &= F(x_0) +  F'(x_0)h +  \frac{1}{2}F^{(2)}(x_0)h^2 + \frac{1}{6}F^{(3)}(x_0)h^3 + \frac{1}{24}F^{(4)}(\theta_{h})h^4, \\
		F(x_0+2h) &= F(x_0) + 2F'(x_0)h +           2F^{(2)}(x_0)h^2  + \frac{8}{6}F^{(3)}(x_0)h^3 + \frac{16}{24}F^{(4)}(\theta_{2h})h^4,\\
		F(x_0+3h) &= F(x_0) + 3F'(x_0)h + \frac{9}{2}F^{(2)}(x_0)h^2  + \frac{27}{6}F^{(3)}(x_0)h^3+ \frac{81}{24}F^{(4)}(\theta_{3h})h^4,\\
	\end{align*}
	where $ \theta_{-2h}\in [x_0-2h,x_0-h], \theta_{-h}\in [x_0-h,x_0], \theta_{h} \in [x_0,x_0+h], 
	\theta_{2h} \in [x_0+h,x_0+2h], \theta_{3h}\in [x_0+2h,x_0+3h]$. 
	Multiplying these equations by their respective coefficients given in 
	Equation \eqref{lemma:fd-ppm-est2-eq0}, one get:
	
	\begin{align*}
		 -2F(x_0-2h) &=  -2F(x_0) + 4F'(x_0)h -          4F^{(2)}(x_0)h^2   + \frac{16}{6}F^{(3)}(x_0)h^3 - \frac{32}{24}F^{(4)}(\theta_{-2h})h^4,\\
		 15F(x_0-h)  &=  15F(x_0) -15F'(x_0)h + \frac{15}{2}F^{(2)}(x_0)h^2 - \frac{15}{6}F^{(3)}(x_0)h^3 + \frac{15}{24}F^{(4)}(\theta_{-h})h^4,\\
		-28F(x_0)    &= -28F(x_0),\\
		 20F(x_0+h)  &=  20F(x_0) +20F'(x_0)h +          10F^{(2)}(x_0)h^2  + \frac{20}{6}F^{(3)}(x_0)h^3 + \frac{20}{24}F^{(4)}(\theta_{h})h^4,\\
		 -6F(x_0+2h) &=  -6F(x_0) -12F'(x_0)h -          12F^{(2)}(x_0)h^2  -            8F^{(3)}(x_0)h^3 - \frac{96}{24}F^{(4)}(\theta_{2h})h^4,\\
		   F(x_0+3h) &=    F(x_0) + 3F'(x_0)h + \frac{9}{2}F^{(2)}(x_0)h^2  + \frac{27}{6}F^{(3)}(x_0)h^3 + \frac{81}{24}F^{(4)}(\theta_{3h})h^4.\\
	\end{align*}
	
	Summing all these equations, we get the desired Formula \eqref{lemma:fd-ppm-est2-eq0} with $C_2$ given by:
	\begin{equation}
		C_2 = \frac{1}{24}\bigg(32F^{(4)}(\theta_{-2h}) - 15F^{(4)}(\theta_{-h}) -20F^{(4)}(\theta_{h}) +96 F^{(4)}(\theta_{2h}) - 81F^{(4)}(\theta_{3h})\bigg).
	\end{equation}
Using the intermediate value theorem, we can express $C_2$ in a more compact way as
\begin{equation}
		\label{lemma:fd-ppm-est2-eq1}
	C_2 = \frac{1}{24}\bigg( 128F^{(5)}(\eta_{1}) -116F^{(5)}(\eta_{2})\bigg), 
\end{equation}
where $\eta_{1}, \eta_{2} \in [x_0-2h,x_0+3h]$, which concludes the proof.
\end{proof}


\begin{lema}
	\label{lemma:fd-ppm-est3}
	Let $F \in \mathcal{C}^{4}(\mathbb{R})$, $x_0 \in \mathbb{R}$ and $h>0$.
	Then, the following identity holds:
	\begin{equation}	
		\label{lemma:fd-ppm-est3-eq0}
		F^{(3)}(x_0) =  \frac{F(x_0-2h) - 7F(x_0-h) +16F(x_0) - 16F(x_0+h) +7F(x_0+2h) - F(x_0+3h)}{2h^3}
		      + C_3h,
	\end{equation}
	where $C_3$ is a constant that depends only on $F$ and $h$.
\end{lema}

\begin{proof}
From the Taylor's expansion, we have:
	\begin{align*}
		F(x_0-2h) &= F(x_0) - 2F'(x_0)h +           2F^{(2)}(x_0)h^2  - \frac{8}{6}F^{(3)}(x_0)h^3 + \frac{16}{24}F^{(4)}(\theta_{-2h})h^4,\\
		F(x_0-h)  &= F(x_0) -  F'(x_0)h +  \frac{1}{2}F^{(2)}(x_0)h^2 - \frac{1}{6}F^{(3)}(x_0)h^3 + \frac{1}{24}F^{(4)}(\theta_{-h})h^4, \\
		F(x_0+h)  &= F(x_0) +  F'(x_0)h +  \frac{1}{2}F^{(2)}(x_0)h^2 + \frac{1}{6}F^{(3)}(x_0)h^3 + \frac{1}{24}F^{(4)}(\theta_{h})h^4, \\
		F(x_0+2h) &= F(x_0) + 2F'(x_0)h +           2F^{(2)}(x_0)h^2  + \frac{8}{6}F^{(3)}(x_0)h^3 + \frac{16}{24}F^{(4)}(\theta_{2h})h^4,\\
		F(x_0+3h) &= F(x_0) + 3F'(x_0)h + \frac{9}{2}F^{(2)}(x_0)h^2  + \frac{27}{6}F^{(3)}(x_0)h^3+ \frac{81}{24}F^{(4)}(\theta_{3h})h^4,\\
	\end{align*}
	where $ \theta_{-2h}\in [x_0-2h,x_0-h], \theta_{-h}\in [x_0-h,x_0], \theta_{h} \in [x_0,x_0+h], 
	\theta_{2h} \in [x_0+h,x_0+2h], \theta_{3h}\in [x_0+2h,x_0+3h]$. 
	Multiplying these equations by their respective coefficients given in 
	Equation \eqref{lemma:fd-ppm-est3-eq0}, one get:
	
	\begin{align*}
		  F(x_0-2h) &=    F(x_0)  - 2F'(x_0)h  + \frac{4}{2}F^{(2)}(x_0)h^2  - \frac{8}{6}F^{(3)}(x_0)h^3 + \frac{16}{24}F^{(4)}(\theta_{-2h})h^4,\\
		 -7F(x_0-h)  &=  -7F(x_0) + 7F'(x_0)h  - \frac{7}{2}F^{(2)}(x_0)h^2 + \frac{7}{6}F^{(3)}(x_0)h^3 - \frac{7}{24}F^{(4)}(\theta_{-h})h^4,\\
		 16F(x_0)    &=  16F(x_0),\\
		-16F(x_0+h)  &= -16F(x_0) -16F'(x_0)h  - \frac{16}{2}F^{(2)}(x_0)h^2  - \frac{16}{6}F^{(3)}(x_0)h^3 - \frac{16}{24}F^{(4)}(\theta_{h})h^4,\\
		  7F(x_0+2h) &=   7F(x_0)  +14F'(x_0)h + \frac{28}{2}F^{(2)}(x_0)h^2  + \frac{56}{6}F^{(3)}(x_0)h^3 + \frac{112}{24}F^{(4)}(\theta_{2h})h^4,\\
		  -F(x_0+3h) &=   -F(x_0) - 3F'(x_0)h - \frac{9}{2}F^{(2)}(x_0)h^2  - \frac{27}{6}F^{(3)}(x_0)h^3 - \frac{81}{24}F^{(4)}(\theta_{3h})h^4.\\
	\end{align*}
	
	Summing all these equations, we have:
	\begin{align*}
	F(x_0-2h) - 7F(x_0-h) +16F(x_0) - 16F(x_0+h) +7F(x_0+2h) - F(x_0+3h) = 2F^{(3)}(x_0)h^3 - 2C_3h^4,
	\end{align*}
	we get the desired Formula \eqref{lemma:fd-ppm-est3-eq0} with $C_3$ given by:
	\begin{equation}
		C_3 = \frac{1}{48}\bigg(-16F^{(4)}(\theta_{-2h}) + 7F^{(4)}(\theta_{-h}) +16F^{(4)}(\theta_{h}) - 112F^{(4)}(\theta_{2h}) + 81F^{(4)}(\theta_{3h})\bigg). 
	\end{equation}
Using the intermediate value theorem, we can express $C_3$ in a more compact way as
\begin{equation}
		\label{lemma:fd-ppm-est2-eq2}
	C_3 = \frac{1}{48}\bigg( 104F^{(5)}(\eta_{1}) -128F^{(5)}(\eta_{2})\bigg), 
\end{equation}
where $\eta_{1}, \eta_{2} \in [x_0-2h,x_0+3h]$, which concludes the proof.
\end{proof}

\section{Lagrange interpolation}
\label{anexo-interp}
Given real numbers, called nodes, $x_0< x_1 < \ldots< x_m$, we define the $k$-th Lagrange polynomial by
\begin{equation*}
	L_k(x) = \prod_{j=0, j \neq k}^{m}\frac{x-x_j}{x_k-x_j}.
\end{equation*}
They satisfy $L_k(x_j) = \delta_{kj}$, where  $\delta_{kj}$ is the Kronecker delta.
Given a function $f$ defined at the nodes $x_j$, its interpolating polynomial of 
degree $m$ is given by:
\begin{equation*}
	P_m(x) = \sum_{k=0}^{m} f(x_k)L_k(x).
\end{equation*}
Indeed, this polynomial interpolates $f$ since $P_m(x_j) = f(x_j)$.
It is well known that $P_m$ always exists and is unique. Besides that, we have the following error formula
for Lagrange interpolation.
\begin{thrm}
	\label{anexo-interp-error1}
	Let $f \in \mathcal{C}^{m+1}(\mathbb{R})$.
	Then, then there is $\xi$ in the smallest interval containing $x_0, \ldots, x_m, x$ such that:
	\begin{equation}
		f(x)-P_m(x) = \omega(x)\frac{f^{(m+1)}(\xi)}{(m+1)!},
	\end{equation}
	where $\omega(x) = (x-x_0)(x-x_1) \ldots (x-x_m)$.
\end{thrm}
\begin{proof}
	See \citet[Theorem~2.1.4.1. on \pno~49]{stoer:2002}.
\end{proof}

\section{Numerical integration}
\label{anexo-numint}
The following mean value theorem for integrals is a very useful tool 
when working with numerical integration errors.
\begin{thrm}[Mean value theorem for integrals]
	\label{anexo-numint-mv}
	If $f \in \mathcal{C}([a,b])$, and $g$ is a integrable function in $[a,b]$
	whose sign does not change in $[a,b]$,
	then there exists $c \in ]a,b[$ such that
	\begin{equation*}
		\int_{a}^{b}f(x)g(x) \,dx = f(c)\int_{a}^{b}g(x) \,dx.
	\end{equation*}
\end{thrm}
\begin{proof}
	See \citet[\pno~143]{courant:1999}.
\end{proof}

\begin{corollary}
	\label{anexo-numint-col1}
	If $f\in \mathcal{C}^1([0,T])$, $t^n = n\Delta t$, $n=0, \ldots, N_T$, $\Delta t = \frac{T}{N_T}$ for some 
	$N_T \in \mathbb{N}$, then:
	\begin{equation}
		\int_{t^n}^{t^{n+1}} f(t)\,dt = \Delta t  f(t_{n})
		+ \frac{\Delta t^{2}}{2} f^{'}(\overline{t}),
	\end{equation}
for some $\overline{t} \in [t^n, t^{n+1}]$.
\end{corollary}

When considering finite-volume schemes, it is useful to compare the average value on a  control volume 
of a function with its value at the  control volume centroid. 
In the following theorems, for the one and two dimensional cases, respectively,
we show that the value of a function at the centroid of a control volume given a second-order approximation to its average value
on the control volume.
\begin{thrm}
	\label{prop-bound-midpoint1d}
	If $f \in \mathcal{C}^2([x_{i-\frac{1}{2}},x_{i+\frac{1}{2}}])$, then 
	\begin{equation}
		\frac{1}{\Delta x}\int_{x_{i-\frac{1}{2}}}^{x_{i+\frac{1}{2}}}{f(x)\,dx}-f(x_i) = C_1 \Delta x^2, 
	\end{equation}
	where $C_1$ is a constant that depends only on $f$, and $x_i = \frac{x_{i+\frac{1}{2}} + x_{i-\frac{1}{2}}}{2}$,
	$\Delta x = x_{i+\frac{1}{2}}-x_{i-\frac{1}{2}}$.
\end{thrm}
\begin{proof}
	From Taylor's expansion, it follows that, for $x \in [x_{i-\frac{1}{2}},x_{i+\frac{1}{2}}]$, we have:
	\begin{equation}
		f(x) = f(x_i) +  f(x_i)(x-x_i) + f''(\xi)\frac{(x-x_i)^2}{2},
	\end{equation}
	for some $\xi$ between $x$ and $x_i$. Therefore:
	\begin{align*}
		\frac{1}{\Delta x} \int_{x_{i-\frac{1}{2}}}^{x_{i+\frac{1}{2}}} {f(x)\,dx} - f(x_i)  
		&= \frac{1}{\Delta x} \int_{x_{i-\frac{1}{2}}}^{x_{i+\frac{1}{2}}} 
		\bigg( f'(x_i)(x-x_i) + f''(\xi)\frac{(x-x_i)^2}{2} \bigg) \,dx \\ 
		&=  \frac{1}{\Delta x} \int_{x_{i-\frac{1}{2}}}^{x_{i+\frac{1}{2}}} 
		f''(\xi)\frac{(x-x_i)^2}{2}  \,dx.
	\end{align*}
	Using the mean value theorem for integrals (see Theorem \ref{anexo-numint-mv}), we have:
	\begin{align*}
		\frac{1}{\Delta x} \int_{x_{i-\frac{1}{2}}}^{x_{i+\frac{1}{2}}} {f(x)\,dx} - f(x_i)  = 
		f''(\eta) \frac{1}{\Delta x} \int_{x_{i-\frac{1}{2}}}^{x_{i+\frac{1}{2}}} 
		\frac{(x-x_i)^2}{2}  \,dx = f''(\eta)\frac{\Delta x^2}{24}
	\end{align*}	
	for some $\eta \in [x_{i-\frac{1}{2}},x_{i+\frac{1}{2}}]$, from which the proposition follows with
	\begin{equation}
		C_1 = \frac{1}{24}f''(\eta).
	\end{equation}
\end{proof}

\begin{thrm}
	\label{prop-bound-midpoint2d}
	If $f \in \mathcal{C}^2([x_{i-\frac{1}{2}},x_{i+\frac{1}{2}}]\times [y_{j-\frac{1}{2}},y_{j+\frac{1}{2}}])$, then 
	\begin{equation}
		\bigg|\frac{1}{\Delta x \Delta y}\int_{x_{i-\frac{1}{2}}}^{x_{i+\frac{1}{2}}}
		\int_{y_{j-\frac{1}{2}}}^{y_{j+\frac{1}{2}}}{f(x,y)\,dx \,dy}-f(x_i,y_j)\bigg| \leq C_1 \Delta x^2 + C_2 \Delta x \Delta y + C_3 \Delta y^2, 
	\end{equation}
	where $C_1, C_2$  and $C_3$ are constants that depend only on $f$, and $x_i = \frac{x_{i+\frac{1}{2}} + x_{i-\frac{1}{2}}}{2}$,
	$y_i = \frac{y_{j+\frac{1}{2}} + y_{j-\frac{1}{2}}}{2}$, $\Delta x = x_{i+\frac{1}{2}}-x_{i-\frac{1}{2}}$,
	$\Delta y= y_{j+\frac{1}{2}}-y_{j-\frac{1}{2}}$.
\end{thrm}
\begin{proof}
	From Taylor's expansion, it follows that, for $x \in [x_{i-\frac{1}{2}},x_{i+\frac{1}{2}}]$, 
	$y \in [y_{j-\frac{1}{2}},y_{j+\frac{1}{2}}]$, we have:
	\begin{align*}
		f(x,y) &= f(x_i,y_j) +  \frac{\partial f}{\partial x}(x_i,y_j)(x-x_i) +\frac{\partial f}{\partial y}(x_i,y_j)(y-y_j) \\
		&+\frac{1}{2}\bigg(\frac{\partial^2 f}{\partial x^2}(\xi,\theta)(x-x_i)^2 +
		2\frac{\partial^2 f}{\partial x \partial y}(\xi,\theta)(x-x_i)(y-y_j)
		\frac{\partial^2 f}{\partial y^2}(\xi,\theta)(y-y_j)^2 \bigg)
	\end{align*}
	for some $\xi$ between $x$ and $x_i$, and $\theta$ between $y$ and $y_j$. Therefore:
	\begin{align*}
		\int_{x_{i-\frac{1}{2}}}^{x_{i+\frac{1}{2}}} 
		\int_{y_{j-\frac{1}{2}}}^{y_{j+\frac{1}{2}}} {f(x,y)\,dy \,dx} - \Delta x \Delta y f(x_i,y_j)  
		= 
		\frac{1}{2} 	
		\int_{x_{i-\frac{1}{2}}}^{x_{i+\frac{1}{2}}} 
		\int_{y_{j-\frac{1}{2}}}^{y_{j+\frac{1}{2}}} 
		{\frac{\partial^2 f}{\partial x^2}(\xi,\theta)(x-x_i)^2 \,dy \,dx} + \\
		\int_{x_{i-\frac{1}{2}}}^{x_{i+\frac{1}{2}}} 
		\int_{y_{j-\frac{1}{2}}}^{y_{j+\frac{1}{2}}} 
		{\frac{\partial^2 f}{\partial x \partial y}(\xi,\theta)(x-x_i)(y-y_j) \,dy \,dx}+
		\frac{1}{2} 	
		\int_{x_{i-\frac{1}{2}}}^{x_{i+\frac{1}{2}}} 
		\int_{y_{j-\frac{1}{2}}}^{y_{j+\frac{1}{2}}} 
		{\frac{\partial^2 f}{\partial y^2}(\xi,\theta)(y-y_j)^2 \,dy \,dx}
	\end{align*}
	Using the mean value theorem for integrals (see Theorem \ref{anexo-numint-mv}), we have:
	\begin{align*}
		&\frac{1}{\Delta x \Delta y}
		\int_{x_{i-\frac{1}{2}}}^{x_{i+\frac{1}{2}}} 
		\int_{y_{j-\frac{1}{2}}}^{y_{j+\frac{1}{2}}} 
		{f(x,y)\,dx \,dy} - f(x_i, y_j)  = 
		{\frac{\partial^2f}{\partial x^2}}(\eta_1, \lambda_1)
		\int_{x_{i-\frac{1}{2}}}^{x_{i+\frac{1}{2}}} 
		\int_{y_{j-\frac{1}{2}}}^{y_{j+\frac{1}{2}}}
		\frac{(x-x_i)^2}{2 \Delta x \Delta y}  \,dx \,dy \\  
		&+
		\int_{x_{i-\frac{1}{2}}}^{x_{i+\frac{1}{2}}} 
		\int_{y_{j-\frac{1}{2}}}^{y_{j+\frac{1}{2}}}
		{\frac{\partial^2 f}{\partial x \partial y}}(\xi,\theta)
		 \frac{(x-x_i)(y-y_j)}{\Delta x \Delta y}  \,dx \,dy +
		{\frac{\partial^2 f}{\partial x^2}}(\eta_2, \lambda_2)
		\int_{x_{i-\frac{1}{2}}}^{x_{i+\frac{1}{2}}} 
		\int_{y_{j-\frac{1}{2}}}^{y_{j+\frac{1}{2}}}
		\frac{(y-y_j)^2}{2 \Delta x \Delta y}  \,dx \,dy \\
		&= {\frac{\partial^2f}{\partial x^2}}(\eta_1, \lambda_1) \frac{\Delta x ^2}{24} +
	      {\frac{\partial^2f}{\partial y^2}}(\eta_2, \lambda_2) \frac{\Delta y ^2}{24} +
	   		\int_{x_{i-\frac{1}{2}}}^{x_{i+\frac{1}{2}}} 
	   \int_{y_{j-\frac{1}{2}}}^{y_{j+\frac{1}{2}}}
	   {\frac{\partial^2 f}{\partial x \partial y}}(\xi,\theta)
	   \frac{(x-x_i)(y-y_j)}{\Delta x \Delta y}  \,dx \,dy
	\end{align*}	
	for $\eta_1, \eta_2 \in [x_{i-\frac{1}{2}},x_{i+\frac{1}{2}}]$, 
	$\lambda_1, \lambda_2  \in [y_{j-\frac{1}{2}},y_{j+\frac{1}{2}}]$,
	from which the proposition follows.
\end{proof}

\section{PPM reconstruction accuracy analysis}
\label{chp2-sec-numerical-analysis-ppm}
In this Section, we are going to investigate the accuracy of the PPM reconstruction process.
As we pointed out in Section \ref{chp2-sec-ppm}, the approximation of $q$ at the control volumes edges
given by Equation \eqref{chp2-sec-recon-ppm-eq9} is fourth-order accurate when $q \in \mathcal{C}^4(\mathbb{R})$. 
This is proved as a Corollary of the following Proposition \ref{prop:ppm-bound1}.
\begin{prop}
	\label{prop:ppm-bound1}
	Let $q \in \mathcal{C}^{4}(\mathbb{R})$, $\overline{x} \in \mathbb{R} $ and $h>0$.
	Then, the following identity holds:
	\begin{equation}
		\label{prop:ppm-bound1-eq1}
		q(\overline{x} ) = \frac{7}{12}\bigg( \frac{1}{h} \int_{\overline{x} }^{\overline{x}+h} q(x) \,dx 
		+ \frac{1}{h} \int_{\overline{x} -h}^{\overline{x} } q(x) \,dx  \bigg)
		- \frac{1}{12}\bigg( \frac{1}{h} \int_{\overline{x} +h}^{\overline{x}+2h} q(x) \,dx 
		+ \frac{1}{h} \int_{\overline{x} -2h}^{\overline{x} -h} q(x) \,dx  \bigg) + C_1h^4,
	\end{equation}
	where $C_1$ is a constant that depends on $q$ and $h$.
\end{prop}
\begin{proof}
	We define $Q(x) = \int_{a}^{x} q(\xi) \,d\xi$ for fixed $a \in \mathbb{R}$ as in 
	Equation \eqref{chp2-sec-recon-ppm-eq5}. It follows that:
	\begin{align*}
		\int_{\overline{x}}^{\overline{x}+h} q(\xi) \,d\xi + \int_{\overline{x}-h}^{\overline{x}} q(\xi) \,d\xi &=
		Q(\overline{x}+h) - Q(\overline{x}-h), \\
		\int_{\overline{x}+h}^{\overline{x}+2h} q(\xi) \,d\xi + \int_{\overline{x}-2h}^{\overline{x}-h} q(\xi) \,d\xi &=
		Q(\overline{x}+2h) - Q(\overline{x}-2h) - (Q(\overline{x}+h) - Q(\overline{x}-h)). \\
	\end{align*}
	Using these identities, Equation \eqref{prop:ppm-bound1-eq1} may be rewritten as:
	\begin{equation}
		\label{prop:ppm-bound1-eq2}
		q(\overline{x}) = \frac{4}{3} \bigg(\frac{Q(\overline{x}+h) - Q(\overline{x}-h)}{2h}\bigg)
		- \frac{1}{3} \bigg(\frac{Q(\overline{x}+2h) - Q(\overline{x}-2h)}{4h}\bigg) + C_1h^4,
	\end{equation}
	which consists of finite-difference approximations. 
	Thus, Equation \eqref{prop:ppm-bound1-eq1} follows from Lemma \ref{lemma:fd-ppm-est1}
	with:
	\begin{equation}
		\label{prop:ppm-bound1-eq3}
		C_1 = C_1(\mu_1,\mu_2) = \frac{1}{720}\bigg(6q^{(4)}(\mu{1}) -32q^{(4)}(\mu_{2})\bigg), 
	\end{equation}
	where $\mu_{1}, \mu_{2} \in [\overline{x}-2h,\overline{x}+2h]$,
	which concludes the proof.
\end{proof}
\begin{corollary}
	\label{prop:ppm-bound1-corollary}
	It follows from Proposition \ref{prop:ppm-bound1} with
	$\overline{x} = x_{i+\frac{1}{2}}$ and $h = \Delta x$
	that $q_{i+\frac{1}{2}}$ given by Equation \eqref{chp2-sec-recon-ppm-eq9} satisfies:
	\begin{equation}
		\label{ppm-edges-bound1}
		q{(x_{i+\frac{1}{2}})} - q_{i+\frac{1}{2}} = C_1\Delta x^4,
	\end{equation}
	with $C_1$ given by Equation \eqref{prop:ppm-bound1-eq3}, whenever $q \in \mathcal{C}^4(\mathbb{R}).$
\end{corollary}
\begin{remark}
	Similarly, one can show that the formulas are given by Equation
	\eqref{chp2-sec-recon-ppm-eq12} and Equation \eqref{chp2-sec-recon-ppm-eq12}
	are fifth-order accurate.
\end{remark}
The parabolic function from \eqref{chp2-sec-recon-ppm-eq1}  given with 
coefficients specified before approximates $q$ with order 3 when 
$q \in \mathcal{C}^4(\mathbb{R})$.
In order to check this, for $x \in X_i$ we rewrite Equation 
\eqref{chp2-sec-recon-ppm-eq1} as: 
\begin{equation}
	\label{chp2-ppm-newton}
	q_i(x;Q) = q_{L,i} + \frac{(\Delta q_i + q_{6, i})}{\Delta x}(x-x_{i-\frac{1}{2}})
	-\frac{q_{6, i}}{\Delta x^2}(x-x_{i-\frac{1}{2}})^2
\end{equation}
and we write $q$ using its Taylor expansion assuming $q \in \mathcal{C}^4(\mathbb{R})$:
\begin{equation}
	\label{chp2-ppm-taylor}
	q(x) = q(x_{i-\frac{1}{2}}) + q'(x_{i-\frac{1}{2}})(x-x_{i-\frac{1}{2}})
	+ \frac{q''(x_{i-\frac{1}{2}})}{2}(x-x_{i-\frac{1}{2}})^2
	+ \frac{q^{(3)}(\theta_i)}{6}(x-x_{i-\frac{1}{2}})^3,
\end{equation}
where $\theta_i \in X_i$.
Comparing Equation \eqref{chp2-ppm-newton} with Equation \eqref{chp2-ppm-taylor},
it is reasonable to seek to some bound to the expressions:
\begin{equation}
	\label{chp2-ppm-bound2eq}
	q'(x_{i-\frac{1}{2}})-\frac{(\Delta q_i + q_{6, i})}{\Delta x},
\end{equation}
and:
\begin{equation} 
	\label{chp2-ppm-bound3eq}
	\frac{q''(x_{i-\frac{1}{2}})}{2} -\bigg(-\frac{q_{6, i}}{\Delta x^2}\bigg).
\end{equation}
We have seen that term $q_{L,i}$ gives a fourth-order approximation to $q(x_{i-\frac{1}{2}})$.
The Corollary \ref{prop:ppm-bound2-corollary} shall prove that 
the term \eqref{chp2-ppm-bound2eq} has a bound proportional to $\Delta x^2$, and
the Corollary \ref{prop:ppm-bound3-corollary} shall prove that the
term \eqref{chp2-ppm-bound3eq} is bounded by a constant times $\Delta x$.

Before proving the desired bounds, it is useful to rewrite some terms
explicitly as functions of the values of the $\Delta x$-grid function $Q$.
Combining Equation \eqref{chp2-sec-recon-ppm-eq4} with Equations
\eqref{chp2-sec-recon-ppm-eq10} and \eqref{chp2-sec-recon-ppm-eq11}, 
we may write $q_{6,i}$ as:
\begin{equation}
	\label{def:q6i-2}
	q_{6,i} = \frac{1}{4} \bigg( Q_{i-2} - 6Q_{i-1} + 10Q_{i} -6Q_{i+1}  + Q_{i+2} \bigg).
\end{equation}
Recalling the definition of $\Delta q_i$ from Equation \eqref{chp2-sec-recon-ppm-eq2},
and applying Equations \eqref{chp2-sec-recon-ppm-eq10} and \eqref{chp2-sec-recon-ppm-eq11}, 
we may express $\Delta q_i$ as:
\begin{equation}
	\label{def:dqi-2}
	\Delta q_i = \frac{1}{12} \bigg(Q_{i-2} -8Q_{i-1} + 8Q_{i+1} -Q_{i+2} \bigg).
\end{equation}
Finally, we combine Equations \eqref{def:q6i-2} and \eqref{def:dqi-2} and write their sum as:
\begin{equation} 
	\label{def:dqi_d6i}
	\frac{(\Delta q_i + q_{6, i})}{\Delta x} = 
	\frac{2Q_{i-2}-13Q_{i-1} +15Q_i -5Q_{i+1} + Q_{i+2}}{6\Delta x}.
\end{equation}
The next Proposition \ref{prop:ppm-bound2} proves that Equation \eqref{def:dqi_d6i}
approximates $q'(x_{i-\frac{1}{2}})$ with order 2.
\begin{prop}
	\label{prop:ppm-bound2}
	Let $q \in \mathcal{C}^{3}(\mathbb{R})$, $\overline{x}\in \mathbb{R}$ and $h>0$.
	Then, the following identity holds:
	\begin{equation}
		\begin{split}
			\label{prop:ppm-bound2-eq1}
			q'(\overline{x} ) = \frac{1}{6h}
			\bigg( \frac{2}{h} \int_{\overline{x}-2h}^{\overline{x}-h} q(x) \,dx 
			-\frac{13}{h}\int_{\overline{x}-h}^{\overline{x}} q(x) \,dx   
			+\frac{15}{h}\int_{\overline{x}}^{\overline{x}+h} q(x) \,dx  \\ 
			-\frac{5}{h} \int_{\overline{x}+h}^{\overline{x}+2h} q(x) \,dx   
			+\frac{1}{h} \int_{\overline{x}+2h}^{\overline{x}+3h} q(x) \,dx   
			\bigg) + C_2h^2,
		\end{split}
	\end{equation}
	where $C_2$ is a constant that depends on $q$ and $h$.
\end{prop}

\begin{proof}
	We consider again $Q(x) = \int_{a}^{x} q(\xi) \,d\xi$ for $a \in \mathbb{R}$ fixed as in 
	Equation \eqref{chp2-sec-recon-ppm-eq5}. 
	Like in Proposition \ref{prop:ppm-bound2}, we have:
	\begin{align*}
		\frac{1}{6h}
		\bigg( \frac{2}{h} \int_{\overline{x}-2h}^{\overline{x}-h} q(x) \,dx 
		-\frac{13}{h}\int_{\overline{x}-h}^{\overline{x}} q(x) \,dx   
		+\frac{15}{h}\int_{\overline{x}}^{\overline{x}+h} q(x) \,dx 
		-\frac{5}{h} \int_{\overline{x}+h}^{\overline{x}+2h} q(x) \,dx   
		+\frac{1}{h} \int_{\overline{x}+2h}^{\overline{x}+3h} q(x) \,dx   
		\bigg)\\
		= \frac{1}{6h} \bigg(
		\frac{2}{h}   \big( Q(\overline{x}-h)- Q(\overline{x}-2h)\big) 
		-\frac{13}{h} \big( Q(\overline{x}) - Q(\overline{x}-h) \big) 
		+\frac{15}{h} \big( Q(\overline{x}+h) - Q(\overline{x})  \big) \\
		-\frac{5}{h}  \big( Q(\overline{x}+2h) - Q(\overline{x}+h)  \big) 
		+\frac{1}{h}  \big( Q(\overline{x}+3h) - Q(\overline{x}+2h)  \big) 
		\bigg)\\
		= \frac{1}{6h^2}\bigg(-2Q(\overline{x}-2h) + 15Q(\overline{x}-h) - 28Q(\overline{x}) 
		+20Q(\overline{x}+h) -6Q(\overline{x}+2h) + Q(\overline{x}+3h)  \bigg),
	\end{align*}
	which consists of the finite-difference scheme from Lemma \ref{lemma:fd-ppm-est2}. 
	Therefore, Equation \eqref{prop:ppm-bound2-eq1} follows from Lemma \ref{lemma:fd-ppm-est2} with:
	\begin{equation}
		\label{prop:ppm-bound2-eq2}
		C_2 = C_2(\mu_1,\mu_2) = \frac{1}{24}\bigg(128q^{(3)}(\mu_{1}) - 116q^{(3)}(\mu_{2})\bigg), 
	\end{equation}
	where $\mu_{1}, \mu_{2} \in [x_0-2h,x_0+3h]$, which concludes the proof.
\end{proof}
\begin{corollary}
	\label{prop:ppm-bound2-corollary}
	It follows from Proposition \ref{prop:ppm-bound2} with $\overline{x} = x_{i-\frac{1}{2}}$
	and $h = \Delta x$ that $\Delta q_i$ given by Equation \eqref{def:dqi-2} 
	and $q_{6,i}$ given by Equation \eqref{def:q6i-2} satisfy:
	\begin{equation}
		q'(x_{i-\frac{1}{2}})-\frac{(\Delta q_i + q_{6, i})}{\Delta x}  = C_2\Delta x^2,
	\end{equation}
	with $C_2$ given by Equation \eqref{prop:ppm-bound2-eq2}, whenever $q \in \mathcal{C}^3(\mathbb{R})$.
\end{corollary}
Now, we analyse the following expression:
\begin{equation}
	\label{def:q6i-3}
	-\frac{2q_{6,i}}{\Delta x^2} = 
	-\frac{1}{2\Delta x^2} \bigg( Q_{i-2} - 6Q_{i-1} + 10Q_{i} -6Q_{i+1}  + Q_{i+2} \bigg).
\end{equation}
deduced from Equation \eqref{def:q6i-2}
and we prove in Proposition \ref{prop:ppm-bound3} that Equation \eqref{def:q6i-3}
approximates $q''(x_{i-\frac{1}{2}})$ with order 1.
\begin{prop}
	\label{prop:ppm-bound3}
	Let $q \in \mathcal{C}^{3}(\mathbb{R})$, $\overline{x} \in \mathbb{R}$ and $h>0$.
	Then, the following identity holds:
	\begin{equation}
		\begin{split}
			\label{prop:ppm-bound3-eq1}
			q''(\overline{x} ) = \frac{1}{2h^2}
			\bigg(-\frac{1}{h} \int_{\overline{x}-2h}^{\overline{x}-h} q(x) \,dx 
			+\frac{6}{h}\int_{\overline{x}-h}^{\overline{x}} q(x) \,dx   
			-\frac{10}{h}\int_{\overline{x}}^{\overline{x}+h} q(x) \,dx  \\ 
			+\frac{6}{h} \int_{\overline{x}+h}^{\overline{x}+2h} q(x) \,dx   
			-\frac{1}{h} \int_{\overline{x}+2h}^{\overline{x}+3h} q(x) \,dx   
			\bigg) + C_3h,
		\end{split}
	\end{equation}
	where $C_3$ is a constant that depends on $q$ and $h$.
\end{prop}
\begin{proof}
	Similarly to Proposition \ref{prop:ppm-bound2} using the same function $Q$, we have:
	\begin{align*}
		\frac{1}{2h^2}
		\bigg(-\frac{1}{h} \int_{\overline{x}-2h}^{\overline{x}-h} q(x) \,dx 
		+\frac{6}{h}\int_{\overline{x}-h}^{\overline{x}} q(x) \,dx   
		-\frac{10}{h}\int_{\overline{x}}^{\overline{x}+h} q(x) \,dx  
		+\frac{6}{h} \int_{\overline{x}+h}^{\overline{x}+2h} q(x) \,dx   
		-\frac{1}{h} \int_{\overline{x}+2h}^{\overline{x}+3h} q(x) \,dx   
		\bigg)\\ 
		= \frac{1}{2h^2} \bigg(
		-\frac{1}{h}   \big( Q(\overline{x}-h)- Q(\overline{x}-2h)\big) 
		+\frac{6}{h} \big( Q(\overline{x}) - Q(\overline{x}-h) \big) 
		-\frac{10}{h} \big( Q(\overline{x}+h) - Q(\overline{x})  \big) \\
		+\frac{6}{h}  \big( Q(\overline{x}+2h) - Q(\overline{x}+h)  \big) 
		-\frac{1}{h}  \big( Q(\overline{x}+3h) - Q(\overline{x}+2h)  \big) 
		\bigg)\\
		= \frac{1}{2h^3}\bigg(Q(\overline{x}-2h) - 7Q(\overline{x}-h) + 16Q(\overline{x}) 
		-16Q(\overline{x}+h) +7Q(\overline{x}+2h) - Q(\overline{x}+3h)  \bigg),
	\end{align*}
	which consists of the finite-difference scheme from Lemma \ref{lemma:fd-ppm-est3}. 
	Therefore, Equation \eqref{prop:ppm-bound3-eq1} follows from 
	Lemma \ref{lemma:fd-ppm-est3} with:
	\begin{equation}
		\label{prop:ppm-bound3-eq2}
		C_3 = C_3(\mu_1,\mu_{2}) = \frac{1}{48}\bigg(104q^{(3)}(\mu_{1}) - 128q^{(3)}(\mu_{2}) \bigg), 
	\end{equation}
	where $\mu_{1}, \mu_{2} \in [x_0-2h,x_0+3h]$, which concludes the proof.
\end{proof}
\begin{corollary}
	\label{prop:ppm-bound3-corollary}
	It follows from Proposition \ref{prop:ppm-bound3} with 
	$\overline{x} = x_{i-\frac{1}{2}}$ and $h = \Delta x$
	that $q_{6,i}$ given by Equation \eqref{chp2-sec-recon-ppm-eq9} satisfies:
	\begin{equation}
		\label{ppm-edges-bound3}
		q''(x_{i-\frac{1}{2}}) -\bigg(-\frac{2q_{6, i}}{\Delta x^2}\bigg) = C_3\Delta x,
	\end{equation}
	with $C_3$ given by Equation \eqref{prop:ppm-bound3-eq2}, whenever $q \in \mathcal{C}^3(\mathbb{R})$.
\end{corollary}
With the aid of Corollaries \ref{prop:ppm-bound1-corollary}, \ref{prop:ppm-bound2-corollary},
and  \ref{prop:ppm-bound3-corollary}, we are able to prove
that the PPM reconstruction approximates $q$ with order 3.
Indeed, we prove this on the follow up Proposition \ref{prop:ppm-bound4}.
\begin{prop}
	\label{prop:ppm-bound4}
	Let $q \in \mathcal{C}^{4}([a,b])$.
	Then, the Piecewise-Parabolic function given by
	Equation \eqref{chp2-sec-recon-ppm-eq1} with 
	the parameters $q_{R,i}$ and $q_{L,i}$ obeying Equations
	\eqref{chp2-sec-recon-ppm-eq10} and \eqref{chp2-sec-recon-ppm-eq11}
	gives a third-order approximation to $q$ on the control volume $X_i$.
	Namely, there exist constants $M_1$ and $M_2$ such that
	\begin{equation*}
		\begin{split}
			|q(x)-q_i(x;Q)| \leq  M_1 \Delta x ^4 + M_2 \Delta x ^3, \quad \forall x \in X_i.
		\end{split}
	\end{equation*}
\end{prop}
\begin{proof}
	For $x \in X_i$, from Equations \eqref{chp2-ppm-taylor} and \eqref{chp2-ppm-newton}, we have:
	\begin{equation*}
		\begin{split}
			q(x)-q_i(x;Q) = (q'(x_{i-\frac{i}{2}})-q_{L,i})	
			+ \bigg(q'(x_{i-\frac{1}{2}})-  \frac{(\Delta q_i + q_{6, i})}{\Delta x}\bigg)(x-x_{i-\frac{1}{2}})
			\\+ \bigg(\frac{q''(x_{i-\frac{1}{2}})}{2} + \frac{q_{6, i}}{\Delta x^2}\bigg)(x-x_{i-\frac{1}{2}})^2
			+ \frac{q^{(3)}(\theta_i)}{6}(x-x_{i-\frac{1}{2}})^3.
		\end{split}
	\end{equation*}
	Using this fact with Corollaries \ref{prop:ppm-bound1-corollary}, \ref{prop:ppm-bound2-corollary},
	and  \ref{prop:ppm-bound3-corollary}, we have:
	\begin{equation*}
		\begin{split}
			q(x)-q_i(x;Q) =  C_1 \Delta x ^4 + C_2 \Delta x ^2(x-x_{i-\frac{1}{2}})
			+ \frac{C_3}{2} \Delta x (x-x_{i-\frac{1}{2}})^2
			+ C_4(x-x_{i-\frac{1}{2}})^3,
		\end{split}
	\end{equation*}
	where $C_1, C_2$ and $C_3$ are given by Equations \eqref{prop:ppm-bound1-eq3},
	\eqref{prop:ppm-bound2-eq2} and \eqref{prop:ppm-bound3-eq2}, respectively, and
	\begin{equation}
		\label{prop:ppm-bound4-eq3}
		C_4 = C_4(\theta_i) = \frac{q^{(3)}(\theta_i)}{6}.
	\end{equation}
	For $x\in X_i$, we have $|x -x_{i-\frac{i}{2}}| \leq \Delta x$, thus:
	\begin{equation*}
		\begin{split}
			|q(x)-q_i(x;Q)| \leq  M_1 \Delta x ^4 + M_2 \Delta x ^3,
		\end{split}
	\end{equation*}
	where 
	\begin{equation*}
		M_1 = \frac{38}{720}\sup_{\xi \in [a,b]} |q^{(4)}(\xi)|, 
	\end{equation*}
	\begin{equation*}
		M_2 = \bigg(\frac{244}{24} + \frac{232}{96} + \frac{1}{6} \bigg)
		\sup_{\xi \in [a,b]} |q^{(3)}(\xi)| = \frac{143}{12} \sup_{\xi \in [a,b]} |q^{(3)}(\xi)|,
	\end{equation*}
	which concludes the proof.
\end{proof}

\begin{remark}
	Replacing the formulas for $q_{R,i}$ and $q_{L,i}$ given by Equations
	\eqref{chp2-sec-recon-ppm-eq10}and \eqref{chp2-sec-recon-ppm-eq11}
	by the formulas given by Equations \eqref{chp2-sec-recon-ppm-eq12}
	and \eqref{chp2-sec-recon-ppm-eq13}, does not change the order of convergence
	of the parabolic approximation.
\end{remark}

\section{Convergence of 1D FV-SL schemes}
\label{convergence-1dfvsl}
\subsection{Consistency and convergence}
\label{chp2-sub-CC}
Hereafter, we are going to use the notations introduced in Section \ref{chp2-sec-not}.
To move towards the convergence of 1D-FV schemes, for  Problem \ref{chp2-sec2-prob4} we introduce the local truncation error (LTE hereafter)
$\tau_i^n$ following \citet{leveque:2002}:
\begin{equation}
	\label{consistency-1d-eq1}
	Q_i(t^{n+1}) = Q_i(t^n) - \lambda
	\bigg(\mathcal{F}(Q(t^n)(\mathcal{S}_{i+\frac{1}{2}}),\tilde{u}^n_{i+\frac{1}{2}})-
	\mathcal{F}(Q(t^n)(\mathcal{S}_{i-\frac{1}{2}}),\tilde{u}^n_{i-\frac{1}{2}}) \bigg) + \Delta t \tau_i^n.
\end{equation}
Notice the LTE is obtained by replacing the exact solution in Equation \eqref{1d-fv-scheme}.
Since $Q_i(t^n)$ is the exact solution of Equation \eqref{1d-fvexact-scheme}, 
the LTE may be rewritten as
\begin{align}
	\begin{split}
		\label{consistency-1d-eq2}
		\tau_{i}^n = 
		\frac{1}{\Delta x} \bigg[  \bigg( \frac{1}{\Delta t}\int_{t^{n}}^{t^{n+1}}
		{(uq)}(x_{i+\frac{1}{2}}, t) \,dt - \mathcal{F}(Q(t^n)(\mathcal{S}_{i+\frac{1}{2}} ),\tilde{u}^n_{i+\frac{1}{2}}) \bigg) +\\
		\bigg( \frac{1}{\Delta t}\int_{t^{n}}^{t^{n+1}}
		{(uq)}(x_{i-\frac{1}{2}}, t) \,dt - \mathcal{F}(Q(t^n)(\mathcal{S}_{i-\frac{1}{2}} ),\tilde{u}^n_{i-\frac{1}{2}}) \bigg)
		\bigg].
	\end{split}
\end{align}
The LTE gives a measure of how well the 1D-FV scheme approximates the integral form
of the considered conservation law. 
Another interpretation of the LTE is that the LTE gives the error obtained after applying
the scheme for a single time-step using the exact solution.
The 1D-FV scheme is said to be consistent
if the LTE converges to zero.


We the define $\tau^n = (\tau_1^n, \ldots, \tau^n_N)$, which represent the LTEs at the time-step $n$.
Now we can define consistency.
\begin{definition}[Consistency]
	\label{chp2-def-cons}
	Let us consider the framework of Problem \ref{chp2-sec2-prob4}.
	A 1D-FV scheme is said to be consistency in the $p$-norm if for any sequence of $(\Delta x^{(k)}, \Delta t^{(k)},\lambda)$-discretizations, 
	$k \in \mathbb{N}$, with $\lim_{k\to \infty }{\Delta x^{(k)}} = \lim_{k\to \infty }{\Delta t^{(k)}} = 0$, we have:
	\begin{equation*}
		\lim_{k \to \infty}\bigg[ {\max_{1\leq n\leq N_T^{(k)}}}{\|\tau^n\|_{p,\Delta x^{(k)}}} \bigg] = 0,
	\end{equation*}
	and it is said to be consistent with order $P$ in the $p-$norm if %there 
	%exists a constant 
	%$C$ that does not depend neither on $\Delta t$ nor on $\Delta x$, such that
	\begin{equation*}
		{\max_{1\leq n\leq N_T^{(k)}}}{\|\tau^n\|_{p,\Delta x^{(k)}}} = O(\Delta x^P).
	\end{equation*}
\end{definition}
From Equation \eqref{consistency-1d-eq2}, it follows that we basically need to ensure that 
the numerical flux function $\mathcal{F}$ converges to the time-averaged flux at edges
when $\Delta x \to 0$ in order to guarantee consistency.
In Section \ref{chp2-sec-flux} we shall address how the numerical flux from PPM
approximates the time-averaged flux at edges.

At last, we define the point-wise error at time-step $n$ by:
\begin{equation*}
	E_i^n = Q_i(t^n) - Q_i^n, \quad i=1, \ldots, N,
\end{equation*}
and we define the vector of errors by $E^n = (E_1^n, \ldots, E_N^n)$. 
\begin{definition}[Convergence]
	\label{chp2-def-conv}
	Let us consider the framework of Problem \ref{chp2-sec2-prob4}.
	A 1D-FV scheme is said to be convergent in the $p$-norm if for any sequence of $(\Delta x^{(k)}, \Delta t^{(k)},\lambda)$-discretizations, 
	$k \in \mathbb{N}$, with $\lim_{k\to \infty }{\Delta x^{(k)}} = \lim_{k\to \infty }{\Delta t^{(k)}} = 0$, we have:
	\begin{equation*}
		\lim_{k\to \infty}\bigg[ {\max_{1\leq n\leq N_T^{(k)}}}{\|E^n\|_{p,\Delta x^{(k)}}} \bigg] = 0,
	\end{equation*}
	and it is said to converge with order $P$ in the $p-$norm if
	%there exists a constant 
	%$C$ that does not depend neither on $\Delta t$ nor on $\Delta x$, such that
	\begin{equation*}
		{\max_{1\leq n\leq N_T^{(k)}}}{\|E^n\|_{p,\Delta x^{(k)}}} = O(\Delta x^P).
	\end{equation*}
\end{definition}
Subtracting Equation \eqref{1d-fv-scheme} from Equation \eqref{consistency-1d-eq1} we get
the following equation for the error:
\begin{align}
	\begin{split}
		\label{erroreq-1d-eq2}
		E^{n+1}_i = E^n_i -
		&\lambda
		\bigg[
		\bigg( \mathcal{F}(Q(t^n)(\mathcal{S}_{i+\frac{1}{2}} ),\tilde{u}^n_{i+\frac{1}{2}}) - \mathcal{F}(Q^n(\mathcal{S}_{i+\frac{1}{2}} ),\tilde{u}^n_{i+\frac{1}{2}}) \bigg) \\
		&-\bigg( \mathcal{F}(Q(t^n)(\mathcal{S}_{i-\frac{1}{2}} ),\tilde{u}^n_{i-\frac{1}{2}}) - \mathcal{F}(Q^n(\mathcal{S}_{i-\frac{1}{2}} ),\tilde{u}^n_{i-\frac{1}{2}}) \bigg)
		\bigg] 
		+ \tau_{i}^n \Delta t .
	\end{split}
\end{align}
Notice that if $q,u \in \mathcal{C}^3$, we can rewrite Equation \ref{consistency-1d-eq2} as:
\begin{align*}
	\begin{split}
		\tau_{i}^n = 
		\bigg[ \frac{1}{\Delta x \Delta t}  \int_{t^{n}}^{t^{n+1}} \int_{x_{i-\frac{1}{2}}}^{x_{i+\frac{1}{2}}}
		{\frac{\partial (uq)}{\partial x}}(x, t) \,dx \,dt - 
		\bigg(\frac{\mathcal{F}(Q(t^n)(\mathcal{S}_{i+\frac{1}{2}} ),\tilde{u}^n_{i-\frac{1}{2}})) -\mathcal{F}(Q(t^n)(\mathcal{S}_{i+\frac{1}{2}} ),\tilde{u}^n_{i-\frac{1}{2}})}{\Delta x} \bigg)
		\bigg].
	\end{split}
\end{align*}
Using the midpoint rule for integration (Theorem \ref{prop-bound-midpoint1d}) and the mean value theorem for integrals
(Theorem \ref{anexo-numint-mv}), 
we have:
\begin{align}
	\begin{split}
		\label{consistency-1d-eq3}
		\tau_{i}^n 
		&= 
		\bigg[ \frac{1}{\Delta t}  \int_{t^{n}}^{t^{n+1}}
		{\bigg(\frac{\partial (uq)}{\partial x}}(x_i, t) + \frac{\Delta x^2}{24}\frac{\partial (uq)}{\partial x}(\xi, t) \bigg) \,dt - 
		\bigg(\frac{\mathcal{F}(Q(t^n)(\mathcal{S}_{i+\frac{1}{2}}),\tilde{u}^n_{i+\frac{1}{2}})) 
			-\mathcal{F}(Q(t^n)(\mathcal{S}_{i-\frac{1}{2}}),\tilde{u}^n_{i-\frac{1}{2}})}{\Delta x} \bigg)
		\bigg] \\
		& = 
		\bigg[ \frac{1}{\Delta t}  \int_{t^{n}}^{t^{n+1}}
		{\frac{\partial (uq)}{\partial x}}(x_i, t) \,dt - 
		\bigg(\frac{\mathcal{F}(Q(t^n)(\mathcal{S}_{i+\frac{1}{2}}),\tilde{u}^n_{i+\frac{1}{2}})) -\mathcal{F}(Q(t^n)(\mathcal{S}_{i-\frac{1}{2}}),\tilde{u}^n_{i-\frac{1}{2}})}{\Delta x} \bigg)
		\bigg] +  \frac{\Delta x^2}{24}\frac{\partial^3 (uq)}{\partial x^3}(\xi, \overline{t}), 
	\end{split}
\end{align}
for $\xi \in X_i$ and $\overline{t} \in [t^{n},t^{n+1}]$. Therefore, if $q,u \in \mathcal{C}^3$ the scheme is
consistent, if and only if, $\frac{1}{\Delta t}  \int_{t^{n}}^{t^{n+1}} {\frac{\partial (uq)}{\partial x}}(x_i, t) \,dt$
is approximated by 
$\frac{\mathcal{F}(Q(t^n)(\mathcal{S}_{i+\frac{1}{2}}),\tilde{u}^n_{i+\frac{1}{2}})) -\mathcal{F}(Q(t^n)(\mathcal{S}_{i-\frac{1}{2}} ),\tilde{u}^n_{i-\frac{1}{2}})}{\Delta x}$.
This shall be very useful when we consider two-dimensional schemes, where we are going to use the discrete operators to estimate the divergence
of velocity fields.

\subsection{Stability}
\label{chp2-sub-stability}
In order to define the concept of stability, it is useful to introduce an operator
representation of 1D-FV schemes.
In the context of Problem \ref{chp2-sec2-prob4}, we define the operators
$\mathcal{H}_{\Delta x,n}: \mathbb{R}^{\Delta x} \to \mathbb{R}^{\Delta x}$ whose $i$-th entry is given by:
\begin{equation}
	[\mathcal{H}_{\Delta x,n}(Q)]_i = Q_i -
	\lambda \bigg( \mathcal{F}(Q(\mathcal{S}_{i+\frac{1}{2}} ),\tilde{u}^n_{i+\frac{1}{2}}) 
	- \mathcal{F}(Q(\mathcal{S}_{i-\frac{1}{2}} ),\tilde{u}^n_{i-\frac{1}{2}}) \bigg),
\end{equation}
for $i=1, \ldots, N$, $n=0, \ldots, N_T-1$.
Notice that the dependence on $n$ is due to the velocity that may be allowed
to vary with time.
As it is usual, we are assuming periodicity in the entries of $Q$ when 
we apply the operator $\mathcal{H}_{\Delta x,n}$.
Thus, Equation \eqref{1d-fv-scheme} may be rewritten in a vector form by
\begin{equation*}
	Q^{n+1} = \mathcal{H}_{\Delta x,n}(Q^n),
\end{equation*}
and Equation \eqref{consistency-1d-eq1} in a vector form reads
\begin{equation*}
	Q(t^{n+1}) = \mathcal{H}_{\Delta x,n}(Q(t^n)) + \Delta t \tau^n,
\end{equation*}
and the error equation \eqref{erroreq-1d-eq2} is given by
\begin{equation}
	\label{erroreq-1d-eq3}
	E^{n+1} = \mathcal{H}_{\Delta x,n}(Q(t^n)) - \mathcal{H}_{\Delta x,n}(Q^n) +  \Delta t \tau^n.
\end{equation}
The stability theory focus on uniformly bounding the norm of $\mathcal{H}_{\Delta x,n}(Q(t^n)) - \mathcal{H}_{\Delta x,n}(Q^n)$ \citep{leveque:2002}.
We define stability as follows.

\begin{definition}[Stability]
	In the context of Problem  \ref{chp2-sec2-prob4},
	a 1D-FV scheme is stable in the $p-$norm if for any $(\Delta x, \Delta t, \lambda)$-discretization of $[a,b]\times [0,T]$ we have:
	\begin{equation}
		\|\mathcal{H}_{\Delta x,n}(Q) - \mathcal{H}_{\Delta x,n}(P)\|_{p,\Delta x} \leq (1+\alpha \Delta t)  \|Q-P\|_{p,\Delta x},
	\end{equation}
	for all $Q, P \in \mathbb{R}^{\Delta x}$ and $\alpha$ is a constant
	that does not depend neither on $\Delta x$ nor on  $\Delta t$.
\end{definition}
Assuming that the scheme is stable in the $p-$norm, then it follows from Equation \eqref{erroreq-1d-eq3} that:
\begin{align}
	\label{chp2-sec2-erroreq}
	\begin{split}
		\|E^{n+1}\|_{p,\Delta x} 
		&\leq \|\mathcal{H}_{\Delta x,n}(Q(t^n)) - \mathcal{H}_{\Delta x,n}(Q^n)\|_{p,\Delta x} +  \Delta t \max_{n=1, \ldots, N_T}\|\tau^n\|_{p,\Delta x}\\
		&\leq (1+\alpha \Delta t)\|E^n\|_{p,\Delta x} +  \Delta t \max_{n=1, \ldots, N_T}\|\tau^n\|_{p,\Delta x}\\
		&\leq (1+\alpha \Delta t)^n\|E^0\|_{p,\Delta x} +  \Delta t \max_{n=1, \ldots, N_T}\|\tau^n\|_{p,\Delta x}
		\sum_{k=0}^{n-1} (1+\alpha \Delta t)^k\\
		&\leq e^{\alpha T}(\|E^0\|_{p,\Delta x} + T\max_{n=1, \ldots, N_T}\|\tau^n\|_{p,\Delta x}), \\
	\end{split}
\end{align}
where we used $n \Delta t\leq T $, $T=N\Delta t$ and the inequality $e^t> 1+t$.
When computing the initial average values using the value at the cell centroid, 
the initial error $E^0$ converges to zero provided $q$ is twice continuously differentiable
by Proposition \ref{prop-bound-centroid}.
Therefore, it follows that if the scheme is stable and consistent
then it is convergent. 
Furthermore, if it is stable and consistent with order $P$, then
the convergence order is at least equal to $\min\{{P,2}\}$.
In the case where both the conservation law and $\mathcal{H}_{\Delta x,n}$ are linear,
this result is a particular case of the Lax-Ritchmyer stability and the convergence
is guaranteed by the Lax equivalence theorem \citep{leveque:2002}.
In this Chapter, we are interested only in the linear advection equation.
However, as we shall see in Section \ref{chp2-sec-flux}, the operator $\mathcal{H}_{\Delta x,n}$
may become non-linear when monotonicity constraints are activated.

Notice that, if $\mathcal{H}_{\Delta x,n}$ is linear, then stability is equivalent to require that
\begin{equation*}
	\|\mathcal{H}_{\Delta x,n}\|_{p,\Delta x} \leq 1+ \alpha \Delta t,
\end{equation*}
where
\begin{equation*}
	\|\mathcal{H}_{\Delta x,n}\|_{p,\Delta x} = \sup_{Q\in \mathbb{R}^{\Delta x}} \frac{\|\mathcal{H}_{\Delta x,n}(Q)\|_{p,\Delta x}}{\|Q\|_{p,\Delta x}},
\end{equation*}
is the operator $p$-norm.

For linear operators, we may use the discrete Fourier transform \citep{trefethen:2000}
to estimate the 2-norm of $\mathcal{H}_{\Delta x,n}$. This approach is known as Von Neumann stability analysis.
We define the nodes $\theta_i = i\frac{2\pi}{N}$, $i=1, \ldots, N$, $\Delta \theta = \frac{2\pi}{N}$,
$\theta = (\theta_1, \theta_2, \ldots, \theta_N)$.
The imaginary unit is denoted by $\imath$.
The Fourier modes are given by:
\begin{equation*}
	e^{\imath k \theta} = (e^{\imath k\theta_1}, e^{\imath k\theta_2}, \ldots, e^{\imath k\theta_N}) \in \mathbb{C}^N, 
\end{equation*}
for $k=1, \ldots, N$. 
Each $k$ is referred to wavenumber and $\theta_k$ is called  dimensionless wavenumber.
The Fourier modes form an orthogonal basis of $\mathbb{C}^N$ with respect to the 
inner product
\begin{equation*}
	\langle Q, P \rangle = \frac{1}{N}\sum_{i=1}^{N}{Q_i \overline{P_i}},
\end{equation*}
for $P, Q \in \mathbb{C}$ and $\overline{z}$ denotes the complex conjugate of $z$. 
Given $Q \in \mathbb{R}^{\Delta x}$, we may may express it in terms of the Fourier modes
\begin{equation*}
	Q = \sum_{k=1}^{N} a_k \exp(\imath k \theta),
\end{equation*}
where $a_k \in \mathbb{C}$. The 2-norm of $Q$ is then given by:
\begin{equation*}
	\|Q\|_{2,\Delta x} = \sqrt{N \sum_{k=1}^{N} |a_k|^2}.
\end{equation*}
The idea of Von Neumann stability analysis is to apply the operator $\mathcal{H}_{\Delta x,n}$ on each Fourier mode and
analyze how it modifies its amplitude.
For ease of analysis, we assume that the velocity is constant, which implies that
the operator $\mathcal{H}_{\Delta x,n}$ has constant coefficients and does not depend on $n$.
For the general case, where the velocity is not constant, the stability
can be ensured using the frozen coefficients method \citep[p.~59]{strikwerda:2004}.
This method boils down to performing multiple times the stability analysis with a constant velocity
being equal to each one of the possible values of the velocity on the grid.
If the scheme is stable for all the possible constant velocities, then stability is ensured.
Since the operator is supposed to be linear with constant coefficients and we
are assuming periodic boundaries conditions, we may write:
\begin{equation*}
	\mathcal{H}_{\Delta x,n}(e^{\imath k \theta}) = \rho(k)  e^{\imath k \theta},
\end{equation*}
where the term $\rho(k)$ is called amplification factor and it is an eigenvalue of $\mathcal{H}_{\Delta x,n}$.
The norm of $\mathcal{H}_{\Delta x,n}(Q)$ is bounded by:
\begin{equation*}
	\|\mathcal{H}_{\Delta x,n}(Q)\|_{2,\Delta x}^2 = N \sum_{k=1}^{N} |a_k|^2 |\rho(k)|^2 \leq 
	\max_{k=1, \ldots, N}{|\rho(k)|}^2 \|Q\|_{2,\Delta x}^2.
\end{equation*}
Therefore:
\begin{equation*}
	\|\mathcal{H}_{\Delta x,n}\|_{2,\Delta x} \leq \max_{k=1, \ldots, N}{|\rho(k)|}.
\end{equation*}
If we show that $\max_{k=1, \ldots, N}{|\rho(k)|} \leq 1 + \alpha \Delta t$, 
with $\alpha$ independent of $\Delta t$, $N$ and $n$, then we ensure the stability of $\mathcal{H}_{\Delta x,n}$.
Generally speaking, the numerical flux can be written as a linear function
\begin{equation*}
	\mathcal{F}(Q(\mathcal{S}_{i+\frac{1}{2}}), \tilde{u}^n_{i+\frac{1}{2}}) = \sum_{l \in \mathcal{S}_{i+\frac{1}{2}}} \alpha_{l,i} Q_{l}, 
\end{equation*}
when no monotonicity constraint is imposed, 
where the coefficients $\alpha_{l,i}$ depend on $\tilde{u}^n, \Delta t$ and $\Delta x$.
We can then express $\rho$ in terms of $\alpha_{l,i}$.
Indeed, when we apply the operator $\mathcal{H}_{\Delta x,n}$ in a Fourier mode, we get:
\begin{align*}
	\begin{split}
		[\mathcal{H}_{\Delta x,n}(e^{\imath k\theta})]_i &= e^{\imath k\theta_i} - \lambda 
		\bigg(\sum_{l \in \mathcal{S}_{i+\frac{1}{2}}} \alpha_{l,i}   e^{\imath k\theta_{l}} - 
		\sum_{l \in \mathcal{S}_{i-\frac{1}{2}}} \alpha_{l,i-1} e^{\imath k\theta_{l-1}} \bigg)\\
		&= e^{\imath k \theta_i}\bigg(1 - \lambda 
		\bigg(\sum_{l \in \mathcal{S}_{i+\frac{1}{2}}} \alpha_{l,i}   e^{\imath k\theta_{l-i}} - 
		\sum_{l \in \mathcal{S}_{i-\frac{1}{2}}} \alpha_{l,i-1} e^{\imath k\theta_{l-1-i}} \bigg) \bigg).\\
	\end{split}
\end{align*}
Hence, the amplification factor has the form
\begin{equation}
	\label{chp2-amplification}
	\rho(k) = 1 - \lambda
	\bigg(\sum_{l \in \mathcal{S}_{i+\frac{1}{2}}} \alpha_{l,i}   e^{\imath k\theta_{l-i}} - 
	\sum_{l \in \mathcal{S}_{i-\frac{1}{2}}} \alpha_{l,i-1} e^{\imath k\theta_{l-1-i}} \bigg).
\end{equation}
In Section \ref{chp2-sec-flux} we shall analyse $|\rho(k)|$ in terms of the PPM coefficients.


\subsection{Flux accuracy analysis}
\label{chp2-sub-flux}
With the stencil coefficients, we can compute the amplification factor 
(Equation \eqref{chp2-amplification}) for the PPM and the hybrid PPM schemes, both without monotonization.
We assume a constant velocity equal to one and $N=100$ (number of control volumes).
In Figure \ref{chp2-fig-amplification} we show the amplification factor for both PPM and hybrid PPM schemes
considering different CFL numbers.
We can observe that both schemes damp most of the Fourier modes for larger $k$, regardless of the CFL number.
Besides that, the hybrid scheme is more effective when reducing the Fourier modes amplitude.
We point out that both schemes are exact when the CFL number is equal to 1.
From this analysis, we can conclude that the PPM and hybrid PPM schemes satisfy the
Von Neumann stability criteria when the CFL restriction is respected.
\begin{figure}[ht]
	\centering
	\includegraphics[width=0.49\linewidth]{stability_PPM-0}
	\includegraphics[width=0.49\linewidth]{stability_PPM-PL07}
	\caption{Amplification factor for the PPM (left) and hybrid PPM (right) schemes for different CFL numbers.}
	\label{chp2-fig-amplification}
\end{figure}
In order to investigate the consistency of the PPM scheme, we notice that when we are deducing the time average flux, 
we are making approximations of the form:
\begin{equation}
	\label{chp2-sec-flux:analysis-eq1}
	\int_{t^n}^{t^{n+1}} (uq)(x_{i+\frac{1}{2}},t) \,dt \approx
	\int_{x_{i+\frac{1}{2}}-\tilde{u}_{i+\frac{1}{2}}^n \Delta t}^{x_{i+\frac{1}{2}}}
	q_{PP}(x;Q^n)\,dx, 
\end{equation}
as we can see from Equations \eqref{chp-sec-flux:fL_1} and \eqref{chp-sec-flux:fR_1},
which basically replace $q$ by $q_{PP}$ and $X(t^n, t^{n+1},x_{i+\frac{1}{2}})$ by 
$x_{i+\frac{1}{2}}-\tilde{u}_{i+\frac{1}{2}}^n \Delta t$
on the right-hand side of Equation \eqref{chp2-sec-flux:approx1}.
As we shall see, the approximation \eqref{chp2-sec-flux:analysis-eq1}
has two sources of error: one related to the departure point estimation and another due to the
parabolic approximation.
The next Proposition \eqref{chp2-sec-flux:prop2} investigates how the departure point error
impacts on the approximation \eqref{chp2-sec-flux:analysis-eq1} replacing $q_{PP}$ by $q$.

\begin{prop}
	\label{chp2-sec-flux:prop2}
	Assume the framework of Problem \ref{chp2-sec2-prob2} with $q \in \mathcal{C}^1$  and $u \in \mathcal{C}^P$, for some $P\geq2$. 
	Furthermore, assume the CFL condition and that $\Delta x$ and $ \Delta t $ are small enough as in Proposition \ref{chp2-sec-flux:departurebound}.
	
	Assume also that for some time-average velocity $\tilde{u}^n_{i+\frac{1}{2}}$, we have
	$X(t^n,t^{n+1};x_{i+\frac{1}{2}}) = x_{i+\frac{1}{2}} - \tilde{u}^n_{i+\frac{1}{2}}\Delta t + C_{i+\frac{1}{2}}\Delta t^P$, 
	where the constants $C_{i+\frac{1}{2}}$ can be written as $C_{i+\frac{1}{2}} = F(\tilde{x}_{i+\frac{1}{2}}, \tilde{t}_{i+\frac{1}{2}})$ for 
	a $\mathcal{C}^1$ function $F:[a,b]^d \times [0,T]^d \to \mathbb{R}$ . 
	Besides that, assume $\tilde{x}_{i+\frac{1}{2}} \in [x_{i+\frac{1}{2}}-k\Delta x, x_{i+\frac{1}{2}} + k\Delta x]^d$, 
	where $k$ do not depend on $\Delta x$ and $\tilde{t}_{i+\frac{1}{2}} \in [t^n, t^{n+1}]^d$.
	Under all these assumptions, we have:
	\begin{align*}
		&\bigg|\int_{t^n}^{t^{n+1}} (uq)(x_{i+\frac{1}{2}},s) \,ds 
		-\int^{x_{i+\frac{1}{2}}}_{x_{i+\frac{1}{2}}-\tilde{u}_{i+\frac{1}{2}}^n \Delta t} q(x,t^n)\,dx
		-\bigg( \int_{t^n}^{t^{n+1}} (uq)(x_{i-\frac{1}{2}},s) \,ds 
		-\int^{x_{i-\frac{1}{2}}}_{x_{i-\frac{1}{2}}-\tilde{u}_{i-\frac{1}{2}}^n \Delta t} q(x,t^n)\,dx  \bigg) \bigg| \\
		& \leq K_1 \Delta t^{P+1},
	\end{align*}
	where $K_1$ depends on $q$ and $u$.
\end{prop}
\begin{proof}
	Using Equation \eqref{chp2-sec-flux:approx1} and the mean value theorem for integrals, we get: 
	\begin{align*}
		\label{chp-sec-flux:depint_5}
		&\int_{t^n}^{t^{n+1}} (uq)(x_{i+\frac{1}{2}},s) \,ds 
		-\int^{x_{i+\frac{1}{2}}}_{x_{i+\frac{1}{2}}-\tilde{u}_{i+\frac{1}{2}}^n \Delta t} q(x,t^n)\,dx =
		\int^{x_{i+\frac{1}{2}}}_{X(t^n,t^{n+1};x_{i+\frac{1}{2}})} q(x,t^n)\,dx
		-\int^{x_{i+\frac{1}{2}}}_{x_{i+\frac{1}{2}}-\tilde{u}_{i+\frac{1}{2}}^n \Delta t} q(x,t^n)\,dx \\ 
		&= \int_{X(t^n,t^{n+1};x_{i+\frac{1}{2}})}^{x_{i+\frac{1}{2}}-\tilde{u}_{i+\frac{1}{2}}^n \Delta t} q(x,t^n)\,dx  
		= \big(X(t^n,t^{n+1};x_{i+\frac{1}{2}}) - x_{i+\frac{1}{2}}+\tilde{u}_{i+\frac{1}{2}}^n \Delta t \big)
		q(\mu_i,t^n) =  F(\tilde{x}_{i+\frac{1}{2}}, \tilde{t}_{i+\frac{1}{2}}) \Delta t^P q(\mu_{i+\frac{1}{2}},t^n),
	\end{align*}
	for some $\mu_{i+\frac{1}{2}} \in X_{i}\cup X_{i+1}$. Similarly, we have:
	\begin{align*}
		&\int_{t^n}^{t^{n+1}} (uq)(x_{i-\frac{1}{2}},s) \,ds 
		-\int^{x_{i-\frac{1}{2}}}_{x_{i-\frac{1}{2}}-\tilde{u}_{i-\frac{1}{2}}^n \Delta t} q(x,t^n)\,dx =
		F(\tilde{x}_{i-\frac{1}{2}},\tilde{t}_{i-\frac{1}{2}}) \Delta t^P q(\mu_{i-\frac{1}{2}},t^n),
	\end{align*}
	and again, $\mu_{i-\frac{1}{2}} \in X_{i-1}\cup X_{i}$
	We introduce the following auxiliary $\mathcal{C}^1$ function:
	\begin{equation*}
		G(\nu) = F(\nu_1,\nu_2)q(\nu_3,t^n).
	\end{equation*}
	where $\nu =(\nu_1,\nu_2, \nu_3)$, $\nu_1 \in [a,b]^d$, $\nu_2 \in [0, T]^d$, $\nu_3 \in [a,b]$.
	Introducing $\nu_{i+\frac{1}{2}} = (\tilde{x}_{i+\frac{1}{2}}, \tilde{t}_{i+\frac{1}{2}}, \mu_{i+\frac{1}{2}})$,
	$\nu_{i-\frac{1}{2}} = (\tilde{x}_{i-\frac{1}{2}}, \tilde{t}_{i-\frac{1}{2}}, \mu_{i-\frac{1}{2}})$
	and using the mean value theorem, we have:
	\begin{align*}
		|G(\nu_{i+\frac{1}{2}})-G(\nu_{i-\frac{1}{2}})|  
		&\leq \bigg(\sup_{\nu \in [a,b]^d\times[0,T]^d\times[a,b]}{\|\nabla G(\nu) \|_{2d+1}} \bigg)\|\nu_{i+\frac{1}{2}}-\nu_{i-\frac{1}{2}}\|_{2d+1} \\
		&\leq \bigg(\sqrt{(d(2k+1)^2 + 9){\lambda^2}+d}\sup_{\nu \in [a,b]^d\times[0,T]^d\times[a,b]}{\|\nabla G(\nu) \|_{2d+1}} \bigg) \Delta t,
	\end{align*}
	where $\|\cdot\|_{D}$ is the 2-norm of $\mathbb{R}^{D}$ and we used that $\|\tilde{x}_{i+\frac{1}{2}}-\tilde{x}_{i-\frac{1}{2}}\|_{d}^2  \leq d(2k+1)^2\Delta x^2$, 
	$|\tilde{\mu}_{i+\frac{1}{2}}-\tilde{\mu}_{i-\frac{1}{2}}| \leq 3 \Delta x$,
	and $\|\tilde{t}_{i+\frac{1}{2}}-\tilde{t}_{i-\frac{1}{2}}\|_{d}^2 \leq d\Delta t^2$.
	Finally, we have the desired bound:
	\begin{align*}
		&\bigg|\int_{t^n}^{t^{n+1}} (uq)(x_{i+\frac{1}{2}},s) \,ds 
		-\int^{x_{i+\frac{1}{2}}}_{x_{i+\frac{1}{2}}-\tilde{u}_{i+\frac{1}{2}}^n \Delta t} q(x,t^n)\,dx
		-\bigg( \int_{t^n}^{t^{n+1}} (uq)(x_{i-\frac{1}{2}},s) \,ds 
		-\int^{x_{i-\frac{1}{2}}}_{x_{i-\frac{1}{2}}-\tilde{u}_{i-\frac{1}{2}}^n \Delta t} q(x,t^n)\,dx  \bigg) \bigg| \\
		&=|(G(\nu_{i+\frac{1}{2}})-G(\nu_{i-\frac{1}{2}}))|\Delta t^P \leq  K_1 \Delta t^{P+1},
	\end{align*}
	where $K_1 = \sqrt{(d(2k+1)^2 + 9){\lambda^2}+d}\sup_{\nu \in [a,b]^d\times[0,T]^d\times[a,b]}{\|\nabla G(\nu) \|_{2d+1}}$.
\end{proof}
\begin{remark}
	If the departure point is computed using Equation \eqref{chp-sec-flux:departurepoint3}, it follows from Proposition \ref{chp-sec-flux:dp_euler}
	that $P=2$, $k=3$, $d=2$.
	The function $F$ is defined by the right-hand side of Equation \eqref{chp-sec-flux:departurepoint6}.
\end{remark}
The next proposition gives a measure of the impact of the Piecewise-Parabolic approximation
on the time average flux, considering this last computed using an estimated departure point.
\begin{prop}
	\label{chp2-sec-flux:prop3}
	Assume the framework of Problem \ref{chp2-sec2-prob2}.
	If $q \in \mathcal{C}^5$ and $u \in \mathcal{C}^1$, then:
	\begin{align*}
		&\bigg|
		\frac{1}{\Delta t}\int^{x_{i+\frac{1}{2}}}_{x_{i+\frac{1}{2}}-\tilde{u}_{i+\frac{1}{2}}^n \Delta t} q(x,t^n)\,dx -
		\mathcal{F}(Q(t_n)(\mathcal{S}_{i+\frac{1}{2}}),\tilde{u}^n_{i+\frac{1}{2}}) 
		-\bigg(\frac{1}{\Delta t}\int^{x_{i-\frac{1}{2}}}_{x_{i-\frac{1}{2}}-\tilde{u}_{i-\frac{1}{2}}^n \Delta t} q(x,t^n)\,dx -
		\mathcal{F}(Q(t_n)\mathcal{S}_{i-\frac{1}{2}},\tilde{u}^n_{i-\frac{1}{2}})\bigg) \bigg|\\
		&\leq K_2\Delta x^4,
	\end{align*}
	where $K_2$ depends on $q$ and $u$.
\end{prop}
\begin{proof}
	We denote by $q_{PP}$ the piecewise-parabolic approximation of $Q(t^n)$. Then:
	\begin{equation*}
		\frac{1}{\Delta t}\int^{x_{i+\frac{1}{2}}}_{x_{i+\frac{1}{2}}-\tilde{u}_{i+\frac{1}{2}}^n \Delta t} q(x,t^n)\,dx -
		\mathcal{F}(Q(t_n)(\mathcal{S}_{i+\frac{1}{2}}),\tilde{u}^n_{i+\frac{1}{2}}) = 	
		\frac{1}{\Delta t}\int^{x_{i+\frac{1}{2}}}_{x_{i+\frac{1}{2}}-\tilde{u}_{i+\frac{1}{2}}^n \Delta t} q(x,t^n)\,dx -
		\frac{1}{\Delta t}\int^{x_{i+\frac{1}{2}}}_{x_{i+\frac{1}{2}}-\tilde{u}_{i+\frac{1}{2}}^n \Delta t} q_{PP}(x;Q(t^n))\,dx
	\end{equation*}
	and
	\begin{equation*}
		\frac{1}{\Delta t}\int^{x_{i-\frac{1}{2}}}_{x_{i-\frac{1}{2}}-\tilde{u}_{i-\frac{1}{2}}^n \Delta t} q(x,t^n)\,dx -
		\mathcal{F}(Q(t_n)(\mathcal{S}_{i-\frac{1}{2}}),\tilde{u}^n_{i-\frac{1}{2}}) = 	
		\frac{1}{\Delta t}\int^{x_{i-\frac{1}{2}}}_{x_{i-\frac{1}{2}}-\tilde{u}_{i-\frac{1}{2}}^n \Delta t} q(x,t^n)\,dx -
		\frac{1}{\Delta t}\int^{x_{i-\frac{1}{2}}}_{x_{i-\frac{1}{2}}-\tilde{u}_{i-\frac{1}{2}}^n \Delta t} q_{PP}(x;Q(t^n))\,dx.
	\end{equation*}
	Similarly to Proposition \ref{prop:ppm-bound4}, we can write:
	\begin{align*}
		q(x,t^n)-q_{PP}(x;Q(t^n)) = &C_1(\mu_1, \mu_2) \Delta x ^4 + C_2(\mu_3, \mu_4) \Delta x ^2(x-x_L)
		+ \frac{C_3}{2}(\mu_5, \mu_6)\Delta x (x-x_L)^2  \\&+C_4(\mu_7)(x-x_L)^3,
	\end{align*}
	where $C_1, C_2$, $C_3$ and $C_4$ are given by Equations \eqref{prop:ppm-bound1-eq3},
	\eqref{prop:ppm-bound2-eq2}, \eqref{prop:ppm-bound3-eq2} and \eqref{prop:ppm-bound4-eq3} respectively,
	$x_L$ is the left boundary of the control volume that contains $x$ ($X_i$ or $X_{i+1}$) and 
	$\mu_k \in [x_{i+\frac{1}{2}} - 3\Delta x,x_{i+\frac{1}{2}} + 3\Delta x]$, $\forall k =1, \ldots, 7$.
	Similarly to Proposition \ref{chp2-sec-flux:prop2}, using the mean value theorem for integrals, one can write:
	\begin{equation}
		\label{chp2-sec-flux:prop3-eq1}
		\int^{x_{i+\frac{1}{2}}}_{x_{i+\frac{1}{2}}-\tilde{u}_{i+\frac{1}{2}}^n \Delta t} (q(x,t^n)-q_{PP}(x;Q(t^n)))\,dx
		= F(\mu^1)\Delta x^4,
	\end{equation}
	and
	\begin{equation}
		\label{chp2-sec-flux:prop3-eq2}
		\int^{x_{i-\frac{1}{2}}}_{x_{i-\frac{1}{2}}-\tilde{u}_{i-\frac{1}{2}}^n \Delta t} (q(x,t^n)-q_{PP}(x;Q(t^n)))\,dx
		= F(\mu^2)\Delta x^4,
	\end{equation}
	for an auxiliary function $F:[a,b]^8 \to \mathbb{R}$, $F\in\mathcal{C}^1$, where $F$ depends on $q$, $u$,
	and $C_1, C_2, C_3$ and $C_4$. Subtracting Equation \eqref{chp2-sec-flux:prop3-eq2} from Equation \eqref{chp2-sec-flux:prop3-eq1}
	and using the mean value theorem, we get the desired inequality.
\end{proof}
Now we are able to tackle the consistency problem on the next proposition.
\begin{prop}
	\label{chp2-sec-flux:prop4}
	Assume the same hypothesis of Proposition \ref{chp2-sec-flux:prop2} and Proposition \ref{chp2-sec-flux:prop3}.
	Denote by $q_{PP}$ the Piecewise-Parabolic approximation of $q(x,t^n)$. Then, the LTE given by Equation
	\eqref{consistency-1d-eq2} satisfies:
	\begin{equation}
		|\tau_i^n|\leq M_1 \Delta t^{P-1} + M_2\Delta x^3,
	\end{equation}
	where $M_1$ and $M_2$ are constants depending only on $q$ and $u$.
\end{prop}
\begin{proof}
	We have:
	\begin{align*}
		&\Delta x  \tau_i^n = \bigg|
		\frac{1}{\Delta t}\int_{t^n}^{t^{n+1}} (uq)(x_{i+\frac{1}{2}},s) \,ds - 
		\mathcal{F}(Q(t_n)(\mathcal{S}_{i+\frac{1}{2}}),\tilde{u}^n_{i+\frac{1}{2}})
		- \frac{1}{\Delta t}\int_{t^n}^{t^{n+1}} (uq)(x_{i-\frac{1}{2}},s) \,ds +
		\mathcal{F}(Q(t_n)(\mathcal{S}_{i-\frac{1}{2}}),\tilde{u}^n_{i-\frac{1}{2}}) \bigg| = \\
		& \frac{1}{\Delta t} \bigg|
		\int_{t^n}^{t^{n+1}} (uq)(x_{i+\frac{1}{2}},s) \,ds - 
		\int^{x_{i+\frac{1}{2}}}_{x_{i+\frac{1}{2}}-\tilde{u}_{i+\frac{1}{2}}^n \Delta t} q_{PP}(x)\,dx
		-\bigg(\int_{t^n}^{t^{n+1}} (uq)(x_{i-\frac{1}{2}},s) \,ds - 
		\int^{x_{i-\frac{1}{2}}}_{x_{i-\frac{1}{2}}-\tilde{u}_{i-\frac{1}{2}}^n \Delta t} q_{PP}(x)\,dx  
		\bigg)\bigg| =\\
		&\frac{1}{\Delta t} \bigg|
		\int_{t^n}^{t^{n+1}} (uq)(x_{i+\frac{1}{2}},s) \,ds 
		-\int^{x_{i+\frac{1}{2}}}_{x_{i+\frac{1}{2}}-\tilde{u}_{i+\frac{1}{2}}^n \Delta t} q(x,t^n)\,dx 
		+\int^{x_{i+\frac{1}{2}}}_{x_{i+\frac{1}{2}}-\tilde{u}_{i+\frac{1}{2}}^n \Delta t} q(x,t^n)\,dx 
		-\int^{x_{i+\frac{1}{2}}}_{x_{i+\frac{1}{2}}-\tilde{u}_{i+\frac{1}{2}}^n \Delta t} q_{PP}(x)\,dx \\
		&-\bigg(\int_{t^n}^{t^{n+1}} (uq)(x_{i-\frac{1}{2}},s) \,ds 
		-\int^{x_{i-\frac{1}{2}}}_{x_{i-\frac{1}{2}}-\tilde{u}_{i-\frac{1}{2}}^n \Delta t} q(x,t^n)\,dx 
		+\int^{x_{i-\frac{1}{2}}}_{x_{i-\frac{1}{2}}-\tilde{u}_{i-\frac{1}{2}}^n \Delta t} q(x,t^n)\,dx 
		-\int^{x_{i-\frac{1}{2}}}_{x_{i-\frac{1}{2}}-\tilde{u}_{i-\frac{1}{2}}^n \Delta t} q_{PP}(x)\,dx 
		\bigg) \bigg| \leq \\
		&\frac{1}{\Delta t} \bigg|
		\int_{t^n}^{t^{n+1}} (uq)(x_{i+\frac{1}{2}},s) \,ds 
		-\int^{x_{i+\frac{1}{2}}}_{x_{i+\frac{1}{2}}-\tilde{u}_{i+\frac{1}{2}}^n \Delta t} q(x,t^n)\,dx 
		-\bigg(\int_{t^n}^{t^{n+1}} (uq)(x_{i-\frac{1}{2}},s) \,ds 
		-\int^{x_{i-\frac{1}{2}}}_{x_{i-\frac{1}{2}}-\tilde{u}_{i-\frac{1}{2}}^n \Delta t} q(x,t^n)\,dx \bigg)\bigg|+\\
		&\frac{1}{\Delta t}\bigg| 
		\int^{x_{i+\frac{1}{2}}}_{x_{i+\frac{1}{2}}-\tilde{u}_{i+\frac{1}{2}}^n \Delta t} q(x,t^n)\,dx 
		-\int^{x_{i+\frac{1}{2}}}_{x_{i+\frac{1}{2}}-\tilde{u}_{i+\frac{1}{2}}^n \Delta t} q_{PP}(x) \,dx 
		-\bigg(\int^{x_{i-\frac{1}{2}}}_{x_{i-\frac{1}{2}}-\tilde{u}_{i-\frac{1}{2}}^n \Delta t} q(x,t^n)\,dx 
		-\int^{x_{i-\frac{1}{2}}}_{x_{i-\frac{1}{2}}-\tilde{u}_{i-\frac{1}{2}}^n \Delta t} q_{PP}(x) \,dx 
		\bigg) \bigg| 
	\end{align*}
	Therefore, it follows from Propositions \ref{chp2-sec-flux:prop2} and \ref{chp2-sec-flux:prop3} that
	\begin{align*}
		|\tau_i^n| \leq \frac{1}{\Delta x \Delta t} K_1 \Delta t^{P+1} + \frac{1}{\Delta x} K_2 \Delta x^4 = 
		K_1 \lambda \Delta t^{P-1} + K_2 \Delta x^3,
	\end{align*}
	from which the proposition follows.
\end{proof}
Thus, it follows from Proposition \ref{chp2-sec-flux:prop3} that the PPM scheme is consistent in the $\infty$-norm
and has two sources of error: one related to the departure point calculation and another due to the parabolic approximation.
In particular, the PPM flux using the departure point from Equation \eqref{chp-sec-flux:departurepoint3}, 
we have a first-order error related to the departure point computation. 
We point out that, if the velocity is constant, then no error is obtained
using the PPM flux, except for the approximation $q_{PP}$ to $q$.
