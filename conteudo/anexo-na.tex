\chapter{Numerical Analysis}
\label{anexo-na}

\section{Finite-difference estimates}
\label{anexo-fd}
This Section aims to prove all finite-difference error estimations 
used throughout this text.
All the proves are very simple and consist of applying Taylor's expansions,
as it is usual when computing the accuracy order of many numerical schemes.
%\theoremstyle{plain} % As próximas definições usam este estilo
%\newtheorem{lema-fd}{Lemma}

\begin{lema}
	\label{lemma:fd-ppm-est1}
	Let $F \in \mathcal{C}^{5}(\mathbb{R})$, $x_0 \in \mathbb{R}$ and $h>0$.
	Then, the following identity holds:
	\begin{equation}	
		\label{lemma:fd-ppm-est1-eq0}
		F'(x_0) =  \frac{4}{3} \bigg(\frac{F(x_0+h) - F(x_0-h)}{2h}\bigg)
                      - \frac{1}{3} \bigg(\frac{F(x_0+2h) - F(x_0-2h)}{4h}\bigg)
		      + C_1h^4,
	\end{equation}
	where $C_1$ is a constant that depends only on $F$ and $h$.
\end{lema}

\begin{proof}
	Given $\delta \in ]0,2h]$, then $x_0 + \delta \in ]x_0,x_0+2h]$ and $x_0 - \delta \in ]x_0-2h,x_0]$.
	Then, we get using the Taylor expansion of $F$:
	\begin{align*}
		F(x_0 + \delta) = F(x_0) +  &F'(x_0)\delta + F^{(2)}(x_0)\frac{\delta ^2}{2} 
		+ F^{(3)}(x_0)\frac{\delta ^3}{3!}
		+ F^{(4)}(x_0)\frac{\delta ^3}{4!}
		+ F^{(5)}(\theta_{\delta} )\frac{\delta ^5}{5!}
		\quad \theta_{\delta} \in [x_0,x_0+\delta ],\\
		F(x_0 - \delta) = F(x_0) -&F'(x_0)\delta  + F^{(2)}(x_0)\frac{\delta ^2}{2}
		- F^{(3)}(x_0)\frac{\delta ^3}{3!}
		+ F^{(4)}(x_0)\frac{\delta ^4}{4!}
		- F^{(5)}(\theta_{-\delta})\frac{\delta ^5}{5!},		
		\quad \theta_{-\delta} \in [x_0-\delta ,x_0].
	\end{align*}
	
	Thus:
	\begin{equation}
		\label{lemma:fd-ppm-est1-eq1}
		\frac{F(x_0 + \delta ) - F(x_0 - \delta )}{2\delta } = F'(x_0) + 
		F^{(3)}(x_0)\frac{\delta^2}{3!} +
		\bigg( F^{(5)}(\theta_{\delta}) + 
		F^{5}(\theta_{-\delta}) \bigg) \frac{\delta ^4}{2\cdot5!} ,		
	\end{equation}
	
	Applying Equation \eqref{lemma:fd-ppm-est1-eq1} for $\delta  = h$ and 
	$\delta  = 2h$, we get, respectively:
	\begin{equation}
		\label{lemma:fd-ppm-est1-eq2}
		\frac{F(x_0 + h) - F(x_0-h)}{2h} = F'(x_0) +
		F^{(3)}(x_0)\frac{h^2}{3!} +
		\bigg(F^{(5)}(\theta_{h}) + F^{5)}(\theta_{-h})\bigg)\frac{h^4}{2\cdot5!}, 
		\quad \theta_{h} \in [x_0,x_0+h], \quad \theta_{-h}\in [x_0-h,x_0],
	\end{equation}
	and
	\begin{align}
	    \label{lemma:fd-ppm-est1-eq3}
		\frac{F(x_0+2h) - F(x_0-2h)}{4h} = F'(x_0) +
		F^{(3)}(x_0)\frac{4h^2}{3!} + 
		\bigg( F^{(5)}(\theta_{2h}) + F^{(5)}(\theta_{-2h})\bigg) \frac{16h^4}{2\cdot5!},
		\\ \nonumber
		\quad \theta_{2h} \in [x_0,x_0+2h], \quad \theta_{-2h}\in [x_0-2h,x_0].
	\end{align}

	Using Equations \eqref{lemma:fd-ppm-est1-eq2} and \eqref{lemma:fd-ppm-est1-eq3}, we obtain:
	\begin{align}
		\label{lemma:fd-ppm-est1-eq4}
		\frac{4}{3} \bigg(\frac{F(x_0+h) - F(x_0-h)}{2h}\bigg) &= \frac{4}{3} F'(x_0) +
				F^{(3)}(x_0)\frac{4h^2}{3\cdot3!} +
                \bigg( F^{(5)}(\theta_{h}) + F^{(5)}(\theta_{-h})\bigg)\frac{h^4}{2\cdot5!},\\
        \label{lemma:fd-ppm-est1-eq5}
	        	\frac{1}{3} \bigg(\frac{F(x_0+2h) - F(x_0-2h)}{4h}\bigg) &= \frac{1}{3} F'(x_0)+
		       	F^{(3)}(x_0)\frac{4h^2}{3\cdot3!} +
		        \bigg( F^{(5)}(\theta_{2h}) + F^{(5)}(\theta_{-2h})\bigg) \frac{16h^4}{3\cdot2\cdot5!} 
	\end{align}

	Subtracting Equation \eqref{lemma:fd-ppm-est1-eq5} from Equation \eqref{lemma:fd-ppm-est1-eq4} we get 
	the desired Equation \eqref{lemma:fd-ppm-est1-eq0} with
	\begin{equation}
		C_1 = \frac{1}{720}\bigg( 3F^{(5)}(\theta_{h}) + 3F^{(5)}(\theta_{-h})
            -16F^{(5)}(\theta_{2h}) - 16F^{(5)}(\theta_{-2h})\bigg), 
		%C = \frac{1}{240}\bigg( F^{(5)}(\theta_{h}) + F^{(5)}(\theta_{-h})\bigg)
		%- \frac{1}{45}\bigg( F^{(5)}(\theta_{2h}) + F^{(5)}(\theta_{-2h})\bigg), 
	\end{equation}
	where $\theta_{h} \in [x_0,x_0+h], \theta_{-h}\in [x_0-h,x_0]$, 
	$\theta_{2h} \in [x_0,x_0+2h], \theta_{-2h}\in [x_0-2h,x_0]$.
	Using the intermediate value theorem, we can express $C_1$ in a more compact way as
	\begin{equation}
	\label{lemma:fd-ppm-est1-eq6}
	C_1 = \frac{1}{720}\bigg( 6F^{(5)}(\eta_{1}) -32F^{(5)}(\eta_{2})\bigg), 
\end{equation}
where $\eta_{1}, \eta_{2} \in [x_0-2h,x_0+2h]$, which concludes the proof.
\end{proof}


\begin{lema}
	\label{lemma:fd-ppm-est2}
	Let $F \in \mathcal{C}^{4}(\mathbb{R})$, $x_0 \in \mathbb{R}$ and $h>0$.
	Then, the following identity holds:
	\begin{equation}	
		\label{lemma:fd-ppm-est2-eq0}
		F''(x_0) =  \frac{-2F(x_0-2h) + 15F(x_0-h) -28F(x_0) + 20F(x_0+h) -6F(x_0+2h) + F(x_0+3h)}{6h^2}
		      + C_2h^2,
	\end{equation}
	where $C_2$ is a constant that depends only on $F$ and $h$.
\end{lema}

\begin{proof}
From the Taylor's expansion, we have:
	\begin{align*}
		F(x_0-2h) &= F(x_0) - 2F'(x_0)h +           2F^{(2)}(x_0)h^2  - \frac{8}{6}F^{(3)}(x_0)h^3 + \frac{16}{24}F^{(4)}(\theta_{-2h})h^4,\\
		F(x_0-h)  &= F(x_0) -  F'(x_0)h +  \frac{1}{2}F^{(2)}(x_0)h^2 - \frac{1}{6}F^{(3)}(x_0)h^3 + \frac{1}{24}F^{(4)}(\theta_{-h})h^4, \\
		F(x_0+h)  &= F(x_0) +  F'(x_0)h +  \frac{1}{2}F^{(2)}(x_0)h^2 + \frac{1}{6}F^{(3)}(x_0)h^3 + \frac{1}{24}F^{(4)}(\theta_{h})h^4, \\
		F(x_0+2h) &= F(x_0) + 2F'(x_0)h +           2F^{(2)}(x_0)h^2  + \frac{8}{6}F^{(3)}(x_0)h^3 + \frac{16}{24}F^{(4)}(\theta_{2h})h^4,\\
		F(x_0+3h) &= F(x_0) + 3F'(x_0)h + \frac{9}{2}F^{(2)}(x_0)h^2  + \frac{27}{6}F^{(3)}(x_0)h^3+ \frac{81}{24}F^{(4)}(\theta_{3h})h^4,\\
	\end{align*}
	where $ \theta_{-2h}\in [x_0-2h,x_0-h], \theta_{-h}\in [x_0-h,x_0], \theta_{h} \in [x_0,x_0+h], 
	\theta_{2h} \in [x_0+h,x_0+2h], \theta_{3h}\in [x_0+2h,x_0+3h]$. 
	Multiplying these equations by their respective coefficients given in 
	Equation \eqref{lemma:fd-ppm-est2-eq0}, one get:
	
	\begin{align*}
		 -2F(x_0-2h) &=  -2F(x_0) + 4F'(x_0)h -          4F^{(2)}(x_0)h^2   + \frac{16}{6}F^{(3)}(x_0)h^3 - \frac{32}{24}F^{(4)}(\theta_{-2h})h^4,\\
		 15F(x_0-h)  &=  15F(x_0) -15F'(x_0)h + \frac{15}{2}F^{(2)}(x_0)h^2 - \frac{15}{6}F^{(3)}(x_0)h^3 + \frac{15}{24}F^{(4)}(\theta_{-h})h^4,\\
		-28F(x_0)    &= -28F(x_0),\\
		 20F(x_0+h)  &=  20F(x_0) +20F'(x_0)h +          10F^{(2)}(x_0)h^2  + \frac{20}{6}F^{(3)}(x_0)h^3 + \frac{20}{24}F^{(4)}(\theta_{h})h^4,\\
		 -6F(x_0+2h) &=  -6F(x_0) -12F'(x_0)h -          12F^{(2)}(x_0)h^2  -            8F^{(3)}(x_0)h^3 - \frac{96}{24}F^{(4)}(\theta_{2h})h^4,\\
		   F(x_0+3h) &=    F(x_0) + 3F'(x_0)h + \frac{9}{2}F^{(2)}(x_0)h^2  + \frac{27}{6}F^{(3)}(x_0)h^3 + \frac{81}{24}F^{(4)}(\theta_{3h})h^4.\\
	\end{align*}
	
	Summing all these equations, we get the desired Formula \eqref{lemma:fd-ppm-est2-eq0} with $C_2$ given by:
	\begin{equation}
		C_2 = \frac{1}{24}\bigg(32F^{(4)}(\theta_{-2h}) - 15F^{(4)}(\theta_{-h}) -20F^{(4)}(\theta_{h}) +96 F^{(4)}(\theta_{2h}) - 81F^{(4)}(\theta_{3h})\bigg).
	\end{equation}
Using the intermediate value theorem, we can express $C_2$ in a more compact way as
\begin{equation}
		\label{lemma:fd-ppm-est2-eq1}
	C_2 = \frac{1}{24}\bigg( 128F^{(5)}(\eta_{1}) -116F^{(5)}(\eta_{2})\bigg), 
\end{equation}
where $\eta_{1}, \eta_{2} \in [x_0-2h,x_0+3h]$, which concludes the proof.
\end{proof}


\begin{lema}
	\label{lemma:fd-ppm-est3}
	Let $F \in \mathcal{C}^{4}(\mathbb{R})$, $x_0 \in \mathbb{R}$ and $h>0$.
	Then, the following identity holds:
	\begin{equation}	
		\label{lemma:fd-ppm-est3-eq0}
		F^{(3)}(x_0) =  \frac{F(x_0-2h) - 7F(x_0-h) +16F(x_0) - 16F(x_0+h) +7F(x_0+2h) - F(x_0+3h)}{2h^3}
		      + C_3h,
	\end{equation}
	where $C_3$ is a constant that depends only on $F$ and $h$.
\end{lema}

\begin{proof}
From the Taylor's expansion, we have:
	\begin{align*}
		F(x_0-2h) &= F(x_0) - 2F'(x_0)h +           2F^{(2)}(x_0)h^2  - \frac{8}{6}F^{(3)}(x_0)h^3 + \frac{16}{24}F^{(4)}(\theta_{-2h})h^4,\\
		F(x_0-h)  &= F(x_0) -  F'(x_0)h +  \frac{1}{2}F^{(2)}(x_0)h^2 - \frac{1}{6}F^{(3)}(x_0)h^3 + \frac{1}{24}F^{(4)}(\theta_{-h})h^4, \\
		F(x_0+h)  &= F(x_0) +  F'(x_0)h +  \frac{1}{2}F^{(2)}(x_0)h^2 + \frac{1}{6}F^{(3)}(x_0)h^3 + \frac{1}{24}F^{(4)}(\theta_{h})h^4, \\
		F(x_0+2h) &= F(x_0) + 2F'(x_0)h +           2F^{(2)}(x_0)h^2  + \frac{8}{6}F^{(3)}(x_0)h^3 + \frac{16}{24}F^{(4)}(\theta_{2h})h^4,\\
		F(x_0+3h) &= F(x_0) + 3F'(x_0)h + \frac{9}{2}F^{(2)}(x_0)h^2  + \frac{27}{6}F^{(3)}(x_0)h^3+ \frac{81}{24}F^{(4)}(\theta_{3h})h^4,\\
	\end{align*}
	where $ \theta_{-2h}\in [x_0-2h,x_0-h], \theta_{-h}\in [x_0-h,x_0], \theta_{h} \in [x_0,x_0+h], 
	\theta_{2h} \in [x_0+h,x_0+2h], \theta_{3h}\in [x_0+2h,x_0+3h]$. 
	Multiplying these equations by their respective coefficients given in 
	Equation \eqref{lemma:fd-ppm-est3-eq0}, one get:
	
	\begin{align*}
		  F(x_0-2h) &=    F(x_0)  - 2F'(x_0)h  + \frac{4}{2}F^{(2)}(x_0)h^2  - \frac{8}{6}F^{(3)}(x_0)h^3 + \frac{16}{24}F^{(4)}(\theta_{-2h})h^4,\\
		 -7F(x_0-h)  &=  -7F(x_0) + 7F'(x_0)h  - \frac{7}{2}F^{(2)}(x_0)h^2 + \frac{7}{6}F^{(3)}(x_0)h^3 - \frac{7}{24}F^{(4)}(\theta_{-h})h^4,\\
		 16F(x_0)    &=  16F(x_0),\\
		-16F(x_0+h)  &= -16F(x_0) -16F'(x_0)h  - \frac{16}{2}F^{(2)}(x_0)h^2  - \frac{16}{6}F^{(3)}(x_0)h^3 - \frac{16}{24}F^{(4)}(\theta_{h})h^4,\\
		  7F(x_0+2h) &=   7F(x_0)  +14F'(x_0)h + \frac{28}{2}F^{(2)}(x_0)h^2  + \frac{56}{6}F^{(3)}(x_0)h^3 + \frac{112}{24}F^{(4)}(\theta_{2h})h^4,\\
		  -F(x_0+3h) &=   -F(x_0) - 3F'(x_0)h - \frac{9}{2}F^{(2)}(x_0)h^2  - \frac{27}{6}F^{(3)}(x_0)h^3 - \frac{81}{24}F^{(4)}(\theta_{3h})h^4.\\
	\end{align*}
	
	Summing all these equations, we have:
	\begin{align*}
	F(x_0-2h) - 7F(x_0-h) +16F(x_0) - 16F(x_0+h) +7F(x_0+2h) - F(x_0+3h) = 2F^{(3)}(x_0)h^3 - 2C_3h^4,
	\end{align*}
	we get the desired Formula \eqref{lemma:fd-ppm-est3-eq0} with $C_3$ given by:
	\begin{equation}
		C_3 = \frac{1}{48}\bigg(-16F^{(4)}(\theta_{-2h}) + 7F^{(4)}(\theta_{-h}) +16F^{(4)}(\theta_{h}) - 112F^{(4)}(\theta_{2h}) + 81F^{(4)}(\theta_{3h})\bigg). 
	\end{equation}
Using the intermediate value theorem, we can express $C_3$ in a more compact way as
\begin{equation}
		\label{lemma:fd-ppm-est2-eq2}
	C_3 = \frac{1}{48}\bigg( 104F^{(5)}(\eta_{1}) -128F^{(5)}(\eta_{2})\bigg), 
\end{equation}
where $\eta_{1}, \eta_{2} \in [x_0-2h,x_0+3h]$, which concludes the proof.
\end{proof}

\section{Lagrange interpolation}
\label{anexo-interp}
Given real numbers, called nodes, $x_0< x_1 < \cdots< x_m$, we define the $k$-th Lagrange polynomial by
\begin{equation*}
	L_k(x) = \prod_{j=0, j \neq k}^{m}\frac{x-x_j}{x_k-x_j}.
\end{equation*}
They satisfy $L_k(x_j) = \delta_{kj}$, where  $\delta_{kj}$ is the Kronecker delta.
Given a function $f$ defined at the nodes $x_j$, its interpolating polynomial of 
degree $m$ is given by:
\begin{equation*}
	P_m(x) = \sum_{k=0}^{m} f(x_k)L_k(x).
\end{equation*}
Indeed, this polynomial interpolates $f$ since $P_m(x_j) = f(x_j)$.
It is well known that $P_m$ always exists and is unique. Besides that, we have the following error formula
for Lagrange interpolation.
\begin{thrm}
	\label{anexo-interp-error1}
	Let $f \in \mathcal{C}^{m+1}(\mathbb{R})$.
	Then, then there is $\xi$ in the smallest interval containing $x_0, \cdots, x_m, x$ such that:
	\begin{equation}
		f(x)-P_m(x) = \omega(x)\frac{f^{(m+1)}(\xi)}{(m+1)!},
	\end{equation}
	where $\omega(x) = (x-x_0)(x-x_1) \cdots (x-x_m)$.
\end{thrm}
\begin{proof}
	See \citet[Theorem~2.1.4.1. on \pno~49]{stoer:2002}.
\end{proof}

\section{Numerical integration}
\label{anexo-numint}
The following mean value theorem for integrals is a very useful tool 
when working with numerical integration errors.
\begin{thrm}[Mean value theorem for integrals]
	\label{anexo-numint-mv}
	If $f \in \mathcal{C}([a,b])$, and $g$ is a integrable function in $[a,b]$
	whose sign does not change in $[a,b]$,
	then there exists $c \in ]a,b[$ such that
	\begin{equation*}
		\int_{a}^{b}f(x)g(x) \,dx = f(c)\int_{a}^{b}g(x) \,dx.
	\end{equation*}
\end{thrm}
\begin{proof}
	See \citet[\pno~143]{courant:1999}.
\end{proof}


\subsection{Multi-step schemes}
Let us consider the following problem: given a function $f \in \mathcal{C}^{m+1}([0,T])$,
a discretization of $[0,T]$ given by $t^n= n\Delta t$, $\Delta t = \frac{T}{N_T}$, for some $N_T \in \mathbb{N}$, we wish
to estimate $\int_{t^n}^{t^{n+1}} f(t)\,dt$ using the values $f(t_{n-k})$, for $k=0,\cdots, m$.
This kind of problem arises, for instance, when we are interested in computing departure points as in Equation \ref{chp-sec-flux:analysis-eq4}.
We can estimate the desired integral by computing the interpolating polynomial of $f(t_{n-k})$, for $k=0,\cdots, m$ and
then integrating this polynomial.
This approach is exactly what is used in multi-step Adams-Bashforth methods.
On the next theorem, we give an expression the error of this approach.

\begin{thrm}
	\label{anexo-numint-adams}
	If $f\in \mathcal{C}^{m+1}([0,T])$, $t^n = n\Delta t$, $n=0, \cdots, N_T$, $\Delta t = \frac{T}{N_T}$ for some 
	$N_T \in \mathbb{N}$, then:
	\begin{equation}
		\label{anexo-numint-eq0}
		\int_{t^n}^{t^{n+1}} f(t)\,dt = \Delta t \sum_{k=0}^{m} \bigg(\int_{0}^{1} L_k(s) \,ds \bigg) f(t_{n-k})
		+ \frac{(\Delta t)^{k+1}}{(m+1)!} f^{(m+1)}(\eta)\int_{0}^{1} \omega(s)\,ds, 
	\end{equation}
where $w(s) = s(s+1)\cdots(s+m)$, $\eta \in [t^{n-m}, t^{n}]$.
\end{thrm}
	
\begin{proof}
We introduce auxiliary functions $\theta(s) = s\Delta t + t_n$, $s \in [-m,1]$ and $g(s) = f(\theta(s))$.
It is clear that $f(t_{n-k}) = g(-k)$, for $k=-1,0, \cdots, m$.
Hence, we can write:
\begin{equation}
	\label{anexo-numint-eq1}
	\int_{t^n}^{t^{n+1}} f(t)\,dt = \Delta t \int_{0}^{1} f(\theta(s))\,ds = \Delta t \int_{0}^{1} g(s)\,ds.
\end{equation}

Defining the nodes $s_k=-k$ for $k=0, \cdots, m$, it follows from Theorem 
\ref{anexo-interp-error1} that the interpolating polynomial $P_m$ of $g(s_k)$ satisfies:
\begin{equation}
	\label{anexo-numint-eq2}
		g(s)-P_m(s) = \omega(s)\frac{g^{(m+1)}(\xi)}{(m+1)!},
\end{equation}
where $\xi \in [-m,1]$.
Substituting Equation \eqref{anexo-numint-eq2} in Equation \eqref{anexo-numint-eq1}, we obtain
\begin{equation}
	\label{anexo-numint-eq3}
	\int_{t^n}^{t^{n+1}} f(t)\,dt = \Delta t \sum_{k=0}^{m} \bigg(\int_{0}^{1} L_k(s) \,ds \bigg) g(-k)
	+ \frac{\Delta t}{(m+1)!} \int_{0}^{1} g^{(m+1)}(\xi) \omega(s)\,ds.
\end{equation}
Since $w(s)$ does not change its sign in $[0,1]$ it follows from Theorem \ref{anexo-numint-mv} that:
\begin{equation}
	\label{anexo-numint-eq4}
	\int_{t^n}^{t^{n+1}} f(t)\,dt = \Delta t \sum_{k=0}^{m} \bigg(\int_{0}^{1} L_k(s) \,ds \bigg) g(-k)
	+ \frac{\Delta t}{(m+1)!} g^{(m+1)}(\overline{\xi})\int_{0}^{1} \omega(s)\,ds, 
\end{equation}
for some $\overline{\xi} \in [-m,1]$. 
Notice that by the chain rule we get $g^{(m+1)}(s) = (\Delta t)^k f^{(m+1)}(\theta(s))$, therefore
Equation \eqref{anexo-numint-eq4} in terms of $f$ reads:
\begin{equation}
	\label{anexo-numint-eq5}
	\int_{t^n}^{t^{n+1}} f(t)\,dt = \Delta t \sum_{k=0}^{m} \bigg(\int_{0}^{1} L_k(s) \,ds \bigg) f(t_{n-k})
	+ \frac{(\Delta t)^{k+1}}{(m+1)!} f^{(m+1)}({\eta})\int_{0}^{1} \omega(s)\,ds, 
\end{equation}
where $\eta \in [t^{n-m},t^n]$, which is the desired identity.
\end{proof}

In the following corollaries, we give the explicit formulas for Equation \eqref{anexo-numint-eq5}
for $m=0,m=1,m=2$. This is achieved by computing the terms $\int_{0}^{1} L_k(s) \,ds$ and $\int_{0}^{1} \omega(s)\,ds$,
which are trivial to be computed.
\begin{corollary}
	\label{anexo-numint-col1}
	If $f\in \mathcal{C}^1([0,T])$, $t^n = n\Delta t$, $n=0, \cdots, N_T$, $\Delta t = \frac{T}{N_T}$ for some 
	$N_T \in \mathbb{N}$, then:
	\begin{equation}
		\int_{t^n}^{t^{n+1}} f(t)\,dt = \Delta t  f(t_{n})
		+ \frac{\Delta t^{2}}{2} f^{'}(\overline{t}),
	\end{equation}
for some $\overline{t} \in [t^n, t^{n+1}]$.
\end{corollary}

\begin{corollary}
	\label{anexo-numint-col2}
	If $f\in \mathcal{C}^2([0,T])$, $t^n = n\Delta t$, $n=0, \cdots, N_T$, $\Delta t = \frac{T}{N_T}$ 
	for some $N_T \in \mathbb{N}$, then:
	\begin{equation}
		\int_{t^n}^{t^{n+1}} f(t)\,dt = \frac{\Delta t}{2} (3f(t_{n}) - f(t_{n-1}) )
		+ \frac{5\Delta t^{3}}{12} f^{(2)}(\overline{t}),
	\end{equation}
	for some $\overline{t} \in [t^{n-1}, t^{n+1}]$.
\end{corollary}

\begin{corollary}
	\label{anexo-numint-col3}
	If $f\in \mathcal{C}^3([0,T])$, $t^n = n\Delta t$, $n=0, \cdots, N_T$, $\Delta t = \frac{T}{N_T}$ 
	for some $N_T \in \mathbb{N}$, then:
	\begin{equation}
		\int_{t^n}^{t^{n+1}} f(t)\,dt = \frac{\Delta t}{12} (23f(t_{n}) - 16f(t_{n-1}) + 5f(t_{n-2}) )
		+ \frac{3\Delta t^{4}}{8} f^{(3)}(\overline{t}),
	\end{equation}
	for some $\overline{t} \in [t^{n-2}, t^{n+1}]$.
\end{corollary}

When using these schemes for and ODE written in its integral form,
$m=0$ gives the classical Euler method; for $m=1$ we get the second-order Adams-Bashforth scheme
and for $m=2$ we have the third-order Adams-Bashforth scheme.

\subsection{Midpoint rule}
When considering finite-volume schemes, it is useful to compare the average value on a  control volume 
of a function with its value at the  control volume centroid. 
In the following theorems, for the one and two dimensional cases, respectively,
we show that the value of a function at the centroid of a control volume given a second-order approximation to its average value
on the control volume.
\begin{thrm}
	\label{prop-bound-midpoint1d}
	If $f \in \mathcal{C}^2([x_{i-\frac{1}{2}},x_{i+\frac{1}{2}}])$, then 
	\begin{equation}
		\frac{1}{\Delta x}\int_{x_{i-\frac{1}{2}}}^{x_{i+\frac{1}{2}}}{f(x)\,dx}-f(x_i) = C_1 \Delta x^2, 
	\end{equation}
	where $C_1$ is a constant that depends only on $f$, and $x_i = \frac{x_{i+\frac{1}{2}} + x_{i-\frac{1}{2}}}{2}$,
	$\Delta x = x_{i+\frac{1}{2}}-x_{i-\frac{1}{2}}$.
\end{thrm}
\begin{proof}
	From Taylor's expansion, it follows that, for $x \in [x_{i-\frac{1}{2}},x_{i+\frac{1}{2}}]$, we have:
	\begin{equation}
		f(x) = f(x_i) +  f(x_i)(x-x_i) + f''(\xi)\frac{(x-x_i)^2}{2},
	\end{equation}
	for some $\xi$ between $x$ and $x_i$. Therefore:
	\begin{align*}
		\frac{1}{\Delta x} \int_{x_{i-\frac{1}{2}}}^{x_{i+\frac{1}{2}}} {f(x)\,dx} - f(x_i)  
		&= \frac{1}{\Delta x} \int_{x_{i-\frac{1}{2}}}^{x_{i+\frac{1}{2}}} 
		\bigg( f'(x_i)(x-x_i) + f''(\xi)\frac{(x-x_i)^2}{2} \bigg) \,dx \\ 
		&=  \frac{1}{\Delta x} \int_{x_{i-\frac{1}{2}}}^{x_{i+\frac{1}{2}}} 
		f''(\xi)\frac{(x-x_i)^2}{2}  \,dx.
	\end{align*}
	Using the mean value theorem for integrals (see Theorem \ref{anexo-numint-mv}), we have:
	\begin{align*}
		\frac{1}{\Delta x} \int_{x_{i-\frac{1}{2}}}^{x_{i+\frac{1}{2}}} {f(x)\,dx} - f(x_i)  = 
		f''(\eta) \frac{1}{\Delta x} \int_{x_{i-\frac{1}{2}}}^{x_{i+\frac{1}{2}}} 
		\frac{(x-x_i)^2}{2}  \,dx = f''(\eta)\frac{\Delta x^2}{24}
	\end{align*}	
	for some $\eta \in [x_{i-\frac{1}{2}},x_{i+\frac{1}{2}}]$, from which the proposition follows with
	\begin{equation}
		C_1 = \frac{1}{24}f''(\eta).
	\end{equation}
\end{proof}

\begin{thrm}
	\label{prop-bound-midpoint2d}
	If $f \in \mathcal{C}^2([x_{i-\frac{1}{2}},x_{i+\frac{1}{2}}]\times [y_{j-\frac{1}{2}},y_{j+\frac{1}{2}}])$, then 
	\begin{equation}
		\bigg|\frac{1}{\Delta x \Delta y}\int_{x_{i-\frac{1}{2}}}^{x_{i+\frac{1}{2}}}
		\int_{y_{j-\frac{1}{2}}}^{y_{j+\frac{1}{2}}}{f(x,y)\,dx \,dy}-f(x_i,y_j)\bigg| \leq C_1 \Delta x^2 + C_2 \Delta x \Delta y + C_3 \Delta y^2, 
	\end{equation}
	where $C_1, C_2$  and $C_3$ are constants that depend only on $f$, and $x_i = \frac{x_{i+\frac{1}{2}} + x_{i-\frac{1}{2}}}{2}$,
	$y_i = \frac{y_{j+\frac{1}{2}} + y_{j-\frac{1}{2}}}{2}$, $\Delta x = x_{i+\frac{1}{2}}-x_{i-\frac{1}{2}}$,
	$\Delta y= y_{j+\frac{1}{2}}-y_{j-\frac{1}{2}}$.
\end{thrm}
\begin{proof}
	From Taylor's expansion, it follows that, for $x \in [x_{i-\frac{1}{2}},x_{i+\frac{1}{2}}]$, 
	$y \in [y_{j-\frac{1}{2}},y_{j+\frac{1}{2}}]$, we have:
	\begin{align*}
		f(x,y) &= f(x_i,y_j) +  \frac{\partial f}{\partial x}(x_i,y_j)(x-x_i) +\frac{\partial f}{\partial y}(x_i,y_j)(y-y_j) \\
		&+\frac{1}{2}\bigg(\frac{\partial^2 f}{\partial x^2}(\xi,\theta)(x-x_i)^2 +
		2\frac{\partial^2 f}{\partial x \partial y}(\xi,\theta)(x-x_i)(y-y_j)
		\frac{\partial^2 f}{\partial y^2}(\xi,\theta)(y-y_j)^2 \bigg)
	\end{align*}
	for some $\xi$ between $x$ and $x_i$, and $\theta$ between $y$ and $y_j$. Therefore:
	\begin{align*}
		\int_{x_{i-\frac{1}{2}}}^{x_{i+\frac{1}{2}}} 
		\int_{y_{j-\frac{1}{2}}}^{y_{j+\frac{1}{2}}} {f(x,y)\,dy \,dx} - \Delta x \Delta y f(x_i,y_j)  
		= 
		\frac{1}{2} 	
		\int_{x_{i-\frac{1}{2}}}^{x_{i+\frac{1}{2}}} 
		\int_{y_{j-\frac{1}{2}}}^{y_{j+\frac{1}{2}}} 
		{\frac{\partial^2 f}{\partial x^2}(\xi,\theta)(x-x_i)^2 \,dy \,dx} + \\
		\int_{x_{i-\frac{1}{2}}}^{x_{i+\frac{1}{2}}} 
		\int_{y_{j-\frac{1}{2}}}^{y_{j+\frac{1}{2}}} 
		{\frac{\partial^2 f}{\partial x \partial y}(\xi,\theta)(x-x_i)(y-y_j) \,dy \,dx}+
		\frac{1}{2} 	
		\int_{x_{i-\frac{1}{2}}}^{x_{i+\frac{1}{2}}} 
		\int_{y_{j-\frac{1}{2}}}^{y_{j+\frac{1}{2}}} 
		{\frac{\partial^2 f}{\partial y^2}(\xi,\theta)(y-y_j)^2 \,dy \,dx}
	\end{align*}
	Using the mean value theorem for integrals (see Theorem \ref{anexo-numint-mv}), we have:
	\begin{align*}
		&\frac{1}{\Delta x \Delta y}
		\int_{x_{i-\frac{1}{2}}}^{x_{i+\frac{1}{2}}} 
		\int_{y_{j-\frac{1}{2}}}^{y_{j+\frac{1}{2}}} 
		{f(x,y)\,dx \,dy} - f(x_i, y_j)  = 
		{\frac{\partial^2f}{\partial x^2}}(\eta_1, \lambda_1)
		\int_{x_{i-\frac{1}{2}}}^{x_{i+\frac{1}{2}}} 
		\int_{y_{j-\frac{1}{2}}}^{y_{j+\frac{1}{2}}}
		\frac{(x-x_i)^2}{2 \Delta x \Delta y}  \,dx \,dy \\  
		&+
		\int_{x_{i-\frac{1}{2}}}^{x_{i+\frac{1}{2}}} 
		\int_{y_{j-\frac{1}{2}}}^{y_{j+\frac{1}{2}}}
		{\frac{\partial^2 f}{\partial x \partial y}}(\xi,\theta)
		 \frac{(x-x_i)(y-y_j)}{\Delta x \Delta y}  \,dx \,dy +
		{\frac{\partial^2 f}{\partial x^2}}(\eta_2, \lambda_2)
		\int_{x_{i-\frac{1}{2}}}^{x_{i+\frac{1}{2}}} 
		\int_{y_{j-\frac{1}{2}}}^{y_{j+\frac{1}{2}}}
		\frac{(y-y_j)^2}{2 \Delta x \Delta y}  \,dx \,dy \\
		&= {\frac{\partial^2f}{\partial x^2}}(\eta_1, \lambda_1) \frac{\Delta x ^2}{24} +
	      {\frac{\partial^2f}{\partial y^2}}(\eta_2, \lambda_2) \frac{\Delta y ^2}{24} +
	   		\int_{x_{i-\frac{1}{2}}}^{x_{i+\frac{1}{2}}} 
	   \int_{y_{j-\frac{1}{2}}}^{y_{j+\frac{1}{2}}}
	   {\frac{\partial^2 f}{\partial x \partial y}}(\xi,\theta)
	   \frac{(x-x_i)(y-y_j)}{\Delta x \Delta y}  \,dx \,dy
	\end{align*}	
	for $\eta_1, \eta_2 \in [x_{i-\frac{1}{2}},x_{i+\frac{1}{2}}]$, 
	$\lambda_1, \lambda_2  \in [y_{j-\frac{1}{2}},y_{j+\frac{1}{2}}]$,
	from which the proposition follows.
\end{proof}