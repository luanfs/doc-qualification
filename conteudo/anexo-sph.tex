\chapter{Spherical coordinates and geometry}
\label{anexo-sph}
%\theoremstyle{plain} % As próximas definições usam este estilo
%\newtheorem{lema-fd}{Lemma}

Given $R>0$, we denote the sphere of radius $R$ 
centered at the origin of  $\mathbb{R}^3$:
\begin{equation*}
	\mathbb{S}^2_R = \{(X,Y,Z) \in \mathbb{R}^3: X^2 + Y^2 + Z^2 = R^2\}.
\end{equation*}
The tangent space at $P \in \mathbb{S}^2_R$ by $T_P \mathbb{S}^2$.
It is easy to see that:
\begin{equation*}
	T_P\mathbb{S}^2_R = \{Q \in \mathbb{R}^3: \langle P,Q\rangle = 0\},
\end{equation*}
where $\langle \cdot, \cdot \rangle$ denotes 
the standard inner product of $\mathbb{R}^3$.
The tangent bundle is denoted by:
\begin{equation*}
T\mathbb{S}_R^2 = \bigcup_{P\in \mathbb{S}^R_2} T_P \mathbb{S}_R^2.
\end{equation*}
We are going to consider three ways to represent an element of $\mathbb{S}_R^2$:
using $(X,Y,Z)$ coordinates, or using $(\lambda, \phi)$
latitude-longitude coordinates, or, at last, using the cubed-sphere
coordinates $(x,y,p)$, where $(x,y)$ are the cube face coordinates and 
$p \in \{1,2,\cdots, 6\}$ stands for a cube panel, as presented in 
Chapter \ref{chp-cs-grids}.

\section{Tangent vectors}
\label{anexo-tangent}
Let us consider $P \in \mathbb{S}_R^2$, we consider the projection
on the tangent space $T_P\mathbb{S}^2_R$ which is given by:
\begin{equation*}
	\Pi_P(Q) = \frac{\langle P, Q \rangle}{R}P - Q
\end{equation*}

We introduce the tangent vectors at $\Psi_p(x,y)$
given by:
$\boldsymbol{g}_x(x,y;p) = \Pi_P(\Psi_p(x+ \Delta x,y))$, 
$\boldsymbol{g}_y(x,y;p) = \Pi_P(\Psi_p(x,y + \Delta y))$,
where $P = \Psi_p(x,y + \Delta y)$.
We then introduce the normalized vectors:
\begin{equation*}
\boldsymbol{e}_x(x,y;p) = 
\frac{\boldsymbol{g}_x(x,y;p)}{\|\boldsymbol{g}_x(x,y;p)\|}, \quad
\boldsymbol{e}_y(x,y;p) = 
\frac{\boldsymbol{g}_y(x,y;p)}{\|\boldsymbol{g}_y(x,y;p)\|},
\end{equation*}
where $\|\cdot\|$ is the Euclidean norm. It can be shown
that the tangent vector of the geodesic from $\Psi_p(x,y)$
from $\Psi_p(x+\Delta x, y)$ is a multiple of $\boldsymbol{e}_x$
and  the tangent vector of the geodesic from $\Psi_p(x,y)$
from $\Psi_p(x, y+\Delta y)$ is a multiple of $\boldsymbol{e}_y$.
Thus, this process using the projection operator on the tangent space
allow us to compute the unit tangent vectors for any cubed-sphere mapping
gridlines at a given point.

\section{Conversions between latitude-longitude  
and contravariant coordinates}
\label{anexo-sph-ll}
We consider the latitude-longitude mapping 
$\Psi_{ll}: [0,2\pi] \times [-\frac{\pi}{2},\frac{\pi}{2}] \to \mathbb{S}^2_R$, given by:
\begin{align}
	\label{ll2sph}
	X(\lambda,\phi) &= R\cos \phi \cos \lambda,\\
	Y(\lambda,\phi) &= R\cos \phi \sin \lambda,\\
	Z(\lambda,\phi) &= R\sin \phi,
\end{align}
The derivative or Jacobian matrix of the mapping $\Psi_{ll}$ is given by:
\begin{equation}
	\label{dpsi}
	D\Psi_{ll} (\lambda,\phi) = 
	R \begin{bmatrix}
		  -\cos \phi \sin \lambda &  -\sin \phi \cos \lambda \\
		   \cos \phi \cos \lambda & \sin \phi \sin \lambda \\
		  0  &  \cos \phi
	\end{bmatrix}
\end{equation}
Using this matrix columns, we can define the tangent vectors:
\begin{equation}
	\boldsymbol{g}_{\lambda}(\lambda,\phi) = D\Psi_{ll}(\lambda,\phi)
	\begin{bmatrix}
		 1 \\
		 0
	\end{bmatrix}, \quad
	\boldsymbol{g}_{\phi}(\lambda,\phi) = D\Psi_{ll}(\lambda,\phi)
	\begin{bmatrix}
		 0 \\
		 1
	\end{bmatrix},
\end{equation}
We normalize the vectors $\boldsymbol{g}_\lambda$ and $\boldsymbol{g}_\phi$
and we obtain unit tangent vectors on the sphere at $\Phi_{ll}(\lambda, \phi)$:
\begin{equation}
	\label{latlon_tg_vectors}
	\boldsymbol{e}_{\lambda}(\lambda,\phi) = 
	\begin{bmatrix}
		 -\sin \lambda \\
		  \cos \lambda \\
		  0
	\end{bmatrix}, \quad
	\boldsymbol{e}_{\phi}(\lambda,\phi) =
	\begin{bmatrix}
		 -\sin \phi \cos \lambda \\
		 -\sin \phi \sin \lambda \\
			  \cos \phi
	\end{bmatrix},
\end{equation}

Let us consider a tangent vector field $\boldsymbol{u}: \mathbb{S}^2_R \to 
T\mathbb{S}_R^2$ on the sphere, \textit{i.e.}, 
$\boldsymbol{u}(P) \in T_P\mathbb{S}^2_R$, $\forall P \in \mathbb{S}^2_R$.
We may express this vector fields in latitude-longitude coordinates as:
\begin{equation}
	\label{latlon-wind}
	\boldsymbol{u}(\lambda, \phi) = 
        u_{\lambda} (\lambda, \phi) \boldsymbol{e}_{\lambda} (\lambda, \phi) + 
	v_{\phi} (\lambda, \phi) \boldsymbol{e}_{\phi} (\lambda, \phi). 
\end{equation}
Or, we may also represent this vector field using the basis 
obtained by cubed-sphere coordinates:
\begin{equation}
	\label{contravariant-wind}
	\boldsymbol{u}(x, y; p) = 
	\tilde{u}(x, y; p) \boldsymbol{e}_{x}(x, y; p) + 
	\tilde{v}(x, y; p) \boldsymbol{e}_{y}(x, y; p).
\end{equation}
This representation is known as contravariant representation.
In order to relate the latitude-longitude representation
with the contravariant representation, we notice that:
\begin{align}
	\label{basis-convertion1}
	\boldsymbol{e_{x}}(x, y; p) &= 
	\langle \boldsymbol{e}_{x} , \boldsymbol{e}_{\lambda}\rangle
	\boldsymbol{e}_{\lambda} (\lambda, \phi)  
	+ \langle \boldsymbol{e}_{x} , \boldsymbol{e}_{\phi}\rangle
	\boldsymbol{e}_{\phi} (\lambda, \phi), \\
	\label{basis-convertion2}
	\boldsymbol{e_{y}}(x, y; p) &=  
	\langle \boldsymbol{e}_{y} , \boldsymbol{e}_{\lambda}\rangle
	  \boldsymbol{e}_{\lambda} (\lambda, \phi) 
	+ \langle \boldsymbol{e}_{y} , \boldsymbol{e}_{\phi}\rangle
	\boldsymbol{e}_{\phi} (\lambda, \phi), 
\end{align}
which holds since the vectors $\boldsymbol{e}_{\lambda}(\lambda, \phi)$ and
$\boldsymbol{e}_{\phi}(\lambda, \phi)$ are orthogonal.
Replacing Equations \eqref{basis-convertion1} and \eqref{basis-convertion2}
in Equation \eqref{contravariant-wind}, we obtain the values $(u_\lambda, v_\phi)$
in terms of the contravariant components $(\tilde{u}, \tilde{v})$ 
as the following matrix equation:
\begin{equation}
	\label{ll-to-contravariant}
	\begin{bmatrix}
		 u_\lambda (\lambda, \phi) \\
		 v_\phi (\lambda, \phi) 
	\end{bmatrix}
	=
	\begin{bmatrix}
		\langle \boldsymbol{e}_x, \boldsymbol{e}_\lambda \rangle 
		& \langle \boldsymbol{e}_y, \boldsymbol{e}_\lambda \rangle \\
		\langle \boldsymbol{e}_x, \boldsymbol{e}_\phi \rangle 
		& \langle \boldsymbol{e}_y, \boldsymbol{e}_\phi \rangle \\
	\end{bmatrix}
	\begin{bmatrix}
		\tilde{u}(x,y;p) \\
		\tilde{v}(x,y;p)
	\end{bmatrix}.
\end{equation}
Conversely, we may express the contravariant components in terms of
latitude-longitude components by inverting Equation \eqref{ll-to-contravariant}:
\begin{equation}
	\label{contravariant-to-ll}
	\begin{bmatrix}
		\tilde{u}(x,y;p) \\
		\tilde{v}(x,y;p)
	\end{bmatrix}
	=
	\frac{1}{\langle \boldsymbol{e}_x, \boldsymbol{e}_\lambda\rangle
	\langle \boldsymbol{e}_y, \boldsymbol{e}_\phi \rangle
	-\langle \boldsymbol{e}_y, \boldsymbol{e}_\lambda \rangle
	\langle \boldsymbol{e}_x, \boldsymbol{e}_\phi \rangle}
	\begin{bmatrix}
		  \langle \boldsymbol{e}_y, \boldsymbol{e}_\phi \rangle 
		&-\langle \boldsymbol{e}_y, \boldsymbol{e}_\lambda \rangle \\
		 -\langle \boldsymbol{e}_x, \boldsymbol{e}_\phi \rangle 
		& \langle \boldsymbol{e}_x, \boldsymbol{e}_\lambda \rangle \\
	\end{bmatrix}
	\begin{bmatrix}
		 u_\lambda (\lambda, \phi) \\
		 v_\phi (\lambda, \phi) 
	\end{bmatrix}.
\end{equation}

\section{Covariant/contravariant conversion}
\label{anexo-cov-con}
Given Equation
Let us consider again a tangent vector field $\boldsymbol{u}: \mathbb{S}^2_R \to 
T\mathbb{S}_R^2$ on the sphere, the contravariant representation of 
$\boldsymbol{u}$ is given by Equation \eqref{contravariant-wind}.
The covariant components $(u,v)$ are given by:
\begin{align}
	\label{covariant-u}
	u(x,y;p) = \langle \boldsymbol{u}(x,y;p) , e_x(x,y;p)  \rangle, \\
	\label{covariant-v}
	v(x,y;p) = \langle \boldsymbol{u}(x,y;p) , e_y(x,y;p)  \rangle.
\end{align}
Replacing Equation \eqref{contravariant-wind} in 
Equations \eqref{covariant-u} and \eqref{covariant-v} we obtain
the relation covariant components in terms of the
contravariant terms:
\begin{equation}
	\label{contravariant-to-covariant}
	\begin{bmatrix}
		{u}(x,y;p) \\
		{v}(x,y;p)
	\end{bmatrix}
	=
	\begin{bmatrix}
		1  
		& \langle \boldsymbol{e}_x, \boldsymbol{e}_y \rangle \\
		  \langle \boldsymbol{e}_x, \boldsymbol{e}_y \rangle 
		& 1 \\
	\end{bmatrix}
	\begin{bmatrix}
		\tilde{u} (x,y;p) \\
		\tilde{v} (x,y;p) 
	\end{bmatrix}.
\end{equation}
Denoting the angle between $\boldsymbol{e}_x$ and  $\boldsymbol{e}_y$ by $\alpha$,
we have $\langle \boldsymbol{e}_x, \boldsymbol{e}_y \rangle = \cos \alpha$.
Thus, we may express the contravariant components in terms of 
the covariant terms inverting Equation \eqref{contravariant-to-covariant}:
\begin{equation}
	\label{convariant-to-contravariant}
	\begin{bmatrix}
		\tilde{u}(x,y;p) \\
		\tilde{v}(x,y;p)
	\end{bmatrix}
	= \frac{1}{\sin^2 \alpha}
	\begin{bmatrix}
		1  
		& -\cos \alpha \\
		  -\cos \alpha
		& 1\\
	\end{bmatrix}
	\begin{bmatrix}
		{u} (x,y;p) \\
		{v} (x,y;p) 
	\end{bmatrix}.
\end{equation}
Notice that combining Equations \eqref{contravariant-to-covariant}
and \eqref{convariant-to-contravariant} with Equations 
\eqref{ll-to-contravariant} and \eqref{contravariant-to-ll}
one may get relations between the latitude-longitude 
components and the covariant components.
