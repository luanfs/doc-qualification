\chapter{Cubed-sphere finite-volume methods}
\label{chp-cs-fv}

\section{Advection finite-volume scheme}

Consistent Treatment of Boundaries: Boundary conditions play a crucial role in maintaining mass conservation. 
The treatment of boundaries should ensure that there is no spurious mass exchange between the model domain and the exterior. 
Various approaches, such as using ghost cells or extrapolation methods, can be employed to enforce mass conservation at the boundaries.

In this Chapter, we show how we can use the dimension splitting method
presented in Chapter \ref{chp-2d-fv} to solve the advection
equation on the cubed-sphere with base on \citet{putman:2007}.

We denote by $\Psi_p:[-a,a] \times [-a,a] \to \mathbb{S}^2_R$,
$p=1, \cdots, 6,$ as a cubed-sphere mapping introduce in Chapter \ref{chp-cs-grids}.
We introduce the notations:

\begin{itemize}
\item $(x,y;p)$ represents a point on the cubed-sphere using a cubed-sphere mapping;
\item $[-a,a]^2 = \bigcup_{i,j=1}^N [x_{i-\frac{1}{2}}, x_{i+\frac{1}{2}}] \times [y_{j-\frac{1}{2}}, y_{j+\frac{1}{2}}]$;
\item $\Delta x = x_{i+\frac{1}{2}}- x_{i-\frac{1}{2}},
\Delta y = y_{j+\frac{1}{2}}- y_{j-\frac{1}{2}}$;
\item $\Omega_{ijp} = \Psi_p([x_{i-\frac{1}{2}}, x_{i+\frac{1}{2}}] \times [y_{j-\frac{1}{2}}, y_{j+\frac{1}{2}}])$
are the cubed-sphere control-volumes;
\item
	$\boldsymbol{g}_{1}(x,y;p) = D\Psi_{p}(x,y)
	\begin{bmatrix}
		1 \\
		0
	\end{bmatrix} \text{and }
	\boldsymbol{g}_{2}(x,y;p) = D\Psi_{p}(x,y)
	\begin{bmatrix}
		0 \\
		1
	\end{bmatrix}
	$ are the tangent vectors;
\item
	$	g_{\Psi}(x,y) =
	\begin{bmatrix}
		\langle \boldsymbol{g}_{1}(x,y;p), \boldsymbol{g}_{1}(x,y;p) \rangle &
		\langle \boldsymbol{g}_{1}(x,y;p), \boldsymbol{g}_{2}(x,y;p) \rangle \\
		\langle \boldsymbol{g}_{1}(x,y;p), \boldsymbol{g}_{2}(x,y;p) \rangle  &
		\langle \boldsymbol{g}_{2}(x,y;p), \boldsymbol{g}_{2}(x,y;p) \rangle
	\end{bmatrix}
	$ is the metric tensor;
\item $\sqrt{\det{g_{\Psi}(x,y) }}$ is the metric tensor Jacobian;
\item $$|\Omega_{ijp}| = \int_{x_{i-\frac{1}{2}}}^{x_{i+\frac{1}{2}}} \int_{y_{i-\frac{1}{2}}}^{y_{i+\frac{1}{2}}}\sqrt{\det{g_{\Psi}(x,y) }} \,dx \,dy$$ are the control-volume areas

\item $$Q_{ijp}(t) = \frac{1}{|\Omega_{ijp}|}\int_{x_{i-\frac{1}{2}}}^{x_{i+\frac{1}{2}}}
\int_{y_{j-\frac{1}{2}}}^{y_{j+\frac{1}{2}}}  q(x,y,t;p) \sqrt{\det{g_{\Psi}(x,y) }}\,dx \,dy$$
are the averages of $q$ on the control-volumes;
\item $u_{i+\frac{1}{2},j,p}^n = u(x_{i+\frac{1}{2}},y_j,t_n;p);$
\item $v_{i,j+\frac{1}{2},p}^n = v(x_i,y_{j+\frac{1}{2}},t_n;p).$
\end{itemize}


Given a tangent velocity field $\boldsymbol{u}$ on the sphere, we denote its
contravariant components by $\tilde{u}$ and $\tilde{v}$.
For a give a detailed discussion on contravariant representations in Appendix \ref{anexo-sph}.
The advection equation on panel the $p$ of the cubed-sphere is given by:
\begin{equation*}
	\frac{\partial}{\partial t}{q}+
	\frac{1}{\sqrt{\det{g_{\Psi}}}}\bigg(
	\frac{\partial}{\partial x} {(\tilde{u}\sqrt{\det{g_{\Psi}}}q)}+
	\frac{\partial}{\partial y} {(\tilde{v}\sqrt{\det{g_{\Psi}}}q)}
	\bigg)
	= 0,
\end{equation*}
$\forall (x,y,t) \in [-a,a]^2\times[0,T]$, $q = {q}(x,y,t;p)$.
Its integral form is given by:
\begin{align*}
	{Q}_{ijp}(t_{n+1})  = {Q}_{ijp}(t_{n})
	&- \frac{\Delta t}{|\Omega_{ijp}|}
	\delta _x \bigg( \frac{1}{\Delta t}
	\int_{t_1}^{t_2} \int_{y_{j-\frac{1}{2}}}^{y_{j+\frac{1}{2}}}
	{(\tilde{u}\sqrt{\det{g_{\Psi}}}q)}(x_{i}, y, t;p)
	\,dy \,dt \bigg) \\ \nonumber
	&- \frac{\Delta t}{|\Omega_{ijp}|}
	\delta _y \bigg( \frac{1}{\Delta t}
	\int_{t_1}^{t_2} \int_{x_{i-\frac{1}{2}}}^{x_{i+\frac{1}{2}}}
	{(\tilde{v}\sqrt{\det{g_{\Psi}}}q)}(x, y_{j}, t;p)
	\,dx \,dt \bigg),
\end{align*}
Hence, we can use the dimension splitting presented in Chapter \ref{chp-2d-fv}
to the variable $\sqrt{\det{g_{\Psi}}}q$.
However, when computing the stencils near to the cube edges,
we need to approximate the values of $q$ in the ghost cells in order to compute
the stencils.