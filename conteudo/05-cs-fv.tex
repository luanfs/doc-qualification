\chapter{Cubed-sphere finite-volume methods}
\label{chp-cs-fv}
In this section, we show how we can use the dimension splitting method
presented in Chapter \ref{chp-2d-fv} to solve the advection
equation on the cubed-sphere with base on \citet{putman:2007}.

\section{Cubed-sphere advection equation its integral form}
\label{chp-cs-adv}
Given a tangent velocity field $\boldsymbol{u}$ on the sphere, we denote its
contravariant components by ${u}$ and ${v}$.
We shall use all the notations introduced in Section \ref{cs-notation}.
The advection equation on panel the $p$ of the cubed-sphere with initial condition $q_0$ is given by:
\begin{equation}
	\begin{cases}
	\label{eq1-adv-cs}
	\bigg[{\partial}_t{q}+
	\frac{1}{\sigma}\bigg(
	{\partial}_x{({u} q \sigma)}+
	{\partial}_y{({v} q \sigma)}
	\bigg)\bigg](x,y,p,t)
	= 0,\\
	q(x,y,p,0) = q_0(x,y,p),
	\end{cases}
\end{equation}
$\forall (x,y) \in [-a,a]^2$, $t\in[0,T]$.
We denote by $\nabla \cdot (q\boldsymbol{u})$ the divergence operator:
\begin{equation}
	\label{advcs:eqdiv}
	\nabla \cdot (q\boldsymbol{u})(x, y, p, t) =  \frac{1}{\sigma}
	[{\partial_x (uq\sigma)} + {\partial_y (vq\sigma)}](x, y, p, t).
\end{equation}
We recall that we say the $\boldsymbol{u}$ is \textbf{non-divergent} if $\nabla \cdot \boldsymbol{u}=0$.
We define the $\mathcal{CS}_N$ grid function $\delta^n$ as
the exact divergence of $q\boldsymbol{u}$ at the cell centers, namely
\begin{equation}
	\label{cs-discrete-div}
	\delta^n_{ijp} = \nabla \cdot (\boldsymbol{u}q)(x_i,y_j,p,t^n).
\end{equation}
Since the metric tensor does not depend on $t$, we may rewrite Equation \eqref{eq1-adv-cs} as
\begin{equation}
	\label{eq2-adv-cs}
	\bigg[{\partial}_t{(q \sigma)}+
	{\partial}_x{({u}q \sigma)}+
	{\partial}_y{({v}q \sigma)}
	\bigg](x,y,p,t)
	= 0.
\end{equation}
Therefore, as in Problem \eqref{chp3-sec2-prob1}, the integral form of Equation \eqref{eq1-adv-cs}
is stated in \eqref{chp5-prob1}.
\begin{prob}
	\label{chp5-prob1}
	Given an initial condition ${q}_0$ and
	a velocity on the sphere $\boldsymbol{u}$, with contravariant components $(u,v)$ on the cubed-sphere coordinate system,
	we would like to find a weak solution ${q}$
	of the cubed-sphere advection equation in its integral form:
	\begin{align*}
		\int_{x_1}^{x_2} \int_{y_1}^{y_2}
		{(q\sigma)}(x, y, p, t) \,dx \,dy = &\int_{x_1}^{x_2} \int_{y_1}^{y_2}
		{(q\sigma)}(x, y, p, t) \,dx \,dy \\ \nonumber
		&-\int_{t_1}^{t_2} \int_{y_1}^{y_2} \bigg({(uq\sigma)}(x_2, y, t)
		-{(uq\sigma)}(x_1, y, t) \bigg) \,dy \,dt\\ \nonumber
		&-\int_{t_1}^{t_2} \int_{x_1}^{x_2} \bigg({(vq\sigma)}(x, y_2, t)
		-{(vq\sigma)}(x, y_1, t) \bigg) \,dx \,dt.
	\end{align*}
	$\forall [x_1, x_2]\times [y_1, y_2] \times[t_1, t_2] \subset \Omega \times[0,T]$, and
	$q(x,y,p,0)=q_0(x,y,p)$.
\end{prob}
Similarly to Section \ref{chp2-sec1}, Equation \eqref{eq1-adv-cs} and Problem \eqref{chp5-prob1} are equivalent
when ${q}, \boldsymbol{u} \in \mathcal{C}^1(\mathbb{S}^2_R)$.
For Problem \ref{chp5-prob1}, the total mass in $\mathbb{S}^2_R$ is defined by: 
\begin{equation}
	{M}_{\mathbb{S}^2_R}(t) = \sum_{p=1}^6 \int_{\Omega} {(q\sigma)}(x,y,p,t) \,dx \,dy , \quad \forall t \in [0,T],
\end{equation}
and is conserved within time: 
\begin{equation}
	{M}_{\mathbb{S}^2_R}(t) = {M}_{\mathbb{S}^2_R}(0), \quad \forall t \in [0,T].
\end{equation}
We define a discretized version of Problem \ref{chp5-prob1} as Problem \ref{chp5-prob2}.
\begin{prob}
	\label{chp5-prob2}
	Assume the framework of Problem \ref{chp5-prob1}
	and consider a $(\Delta x, \Delta y, \Delta t, \lambda)$-discretization of $\Omega\times [0,T]$, with $\Delta x= \Delta y$.
	Since we are in the framework of Problem \ref{chp5-prob1}, it follows that:
	\begin{align*}
		{Q}_{ijp}(t_{n+1})  = {Q}_{ijp}(t_{n})
		&- \lambda \frac{\Delta x  \Delta y}{|\Omega_{ijp}|}
		\delta _x \bigg( \frac{1}{\Delta t \Delta y}
		\int_{t^n}^{t^{n+1}} \int_{y_{j-\frac{1}{2}}}^{y_{j+\frac{1}{2}}} 
		{(uq\sigma)}(x_{i}, y, p, t)
		\,dy \,dt \bigg) \\ \nonumber
		&- \lambda \frac{\Delta x  \Delta y}{|\Omega_{ijp}|}
		\delta _y \bigg( \frac{1}{\Delta t \Delta x}
		\int_{t^n}^{t^{n+1}} \int_{x_{i-\frac{1}{2}}}^{x_{i+\frac{1}{2}}} 
		{(vq\sigma)}(x, y_{j}, p, t)
		\,dx \,dt \bigg),
	\end{align*}
	where
	\begin{equation}
	 {Q}_{ijp}(t) = \frac{1}{|\Omega_{ijp}|}
	\int_{x_{i-\frac{1}{2}}}^{x_{i+\frac{1}{2}}} 
	\int_{y_{j-\frac{1}{2}}}^{y_{j+\frac{1}{2}}} {(q\sigma)}(x,y,p,t) \,dx \,dy.
	\end{equation}
	Our problem now consists of finding the values ${Q}_{ijp}(t_{n})$, 
	$\forall i = 1, \ldots, N$, $\forall j = 1, \ldots, M$, $\forall n = 0, \ldots, N_T-1$,
	given the initial values ${Q}_{ijp}(0)$, $\forall i = 1, \ldots N$, $\forall j = 1, \ldots, M$.
	In other words, we aim to find the average values of ${q}$ in each control volume $\Omega_{ijp}$ at the specified time instances.
\end{prob}
It is important to note that no approximations have been made in Problems \eqref{chp5-prob1} and \eqref{chp5-prob2}. 
In practice, the term $\frac{\Delta x  \Delta y}{|\Omega_{ijp}|}$ is estimated using the second-order
formula obtained by applying the midpoint rule in Equation \eqref{chp4-area}:
\begin{equation}
\frac{\Delta x  \Delta y}{|\Omega_{ijp}|}= \frac{1}{\sigma_{ijp}} + O(\Delta x^2).
\end{equation}

\section{Finite-volume on the cubed-sphere approach}
\label{chp-cs-fvcs}
We are ready to introduce the finite-volume scheme on the cubed-sphere (CS-FV).
A CS-FV scheme problem as follows in Problem \ref{chp5-prob3}.
\begin{prob}[CS-FV scheme]
	\label{chp5-prob3}
	Assume the framework defined in Problem \ref{chp5-prob2}.
	The finite-volume approach of Problem \ref{chp5-prob1}
	consists of a finding a scheme of the form:
	\begin{align}
		\label{chp5-csfv}
		{Q}_{ijp}^{n+1} =  {Q}_{ijp}^{n} - \frac{\lambda}{\sigma_{ijp}}\delta_i {F}_{ijp}^{n}
		- \frac{\lambda}{\sigma_{ijp}} \delta_j {G}_{ijp}^{n},
		\\ \nonumber \quad \forall i = 1, \ldots, N, \quad \forall j = 1, \ldots, M, \quad p =1, \ldots, 6,
		\quad \forall n = 0, \ldots, N_T-1,
	\end{align}
	where $ \delta_i F_{ijp}^n =
	{F}_{i+\frac{1}{2},j,p}^{n} 
	- {F}_{i-\frac{1}{2},j,p}^{n}$,
	$ \delta_j G_{ijp}^n =
	{G}_{i,j+\frac{1}{2},p}^{n} 
	- {G}_{i,j-\frac{1}{2},p}^{n}$ 
	and ${Q}^{n}\in \mathcal{CS}_N$ is intended to be an approximation
	of ${Q}(t_{n})\in \mathcal{CS}_N$ in some sense. We define ${Q}_{ijp}^{0} = {Q}_{ijp}(0)$ or
	${Q}_{ijp}^{0} = {q}^0_{ijp}$.
	
	The term ${F}_{i+\frac{1}{2}, j, p}^{n}$ is known as numerical flux in the 
	$x$ direction and it approximates
	$\frac{1}{\Delta t \Delta y}\int_{t_n}^{t_{n+1}} 
	\int_{y_{j-\frac{1}{2}}}^{y_{j+\frac{1}{2}}} 
	(uq\sigma)(x_{i+\frac{1}{2}}, y, p, t) \,dy \,dt $,
	$\forall i = 0, 1, \ldots, N$, and 
	${G}_{i, j+\frac{1}{2}, p}^{n}$ is known as numerical flux in the 
	$y$ direction and it approximates
	$\frac{1}{\Delta t \Delta x}\int_{t_n}^{t_{n+1}}  
	\int_{x_{i-\frac{1}{2}}}^{x_{i+\frac{1}{2}}}
	(vq\sigma)(x, y_{j+\frac{1}{2}}, p, t) \,dx \,dt $,
	$\forall j = 0, 1, \ldots, M$,
	or, in other words, they estimate the time-averaged
	fluxes at the control volume $\Omega_{ijp}$ boundaries.
\end{prob}
\begin{remark}
	For Problem \ref{chp5-prob3}, we define the CFL number in the $x$ and $y$ direction
	by $\max \{{|u_{i+\frac{1}{2},j}^n}|\}\frac{\Delta t}{\Delta x}$ and 
	$\max \{ {|v_{i,j+\frac{1}{2}}^n}|\}\frac{\Delta t}{\Delta y}$, respectively.
	The CFL number is maximum between these numbers and we say that the CFL condition is
	satisfied if the CFL number is less than one. 
\end{remark}
As in Section \ref{sec:fv-2d}  we introduce the notion of discrete divergence,
which allow us to check the consistency of CS-FV schemes.
\begin{definition}[Discrete divergence]
	\label{chp5-def-div}
	For Problem \ref{chp5-prob3}, we define the discrete divergence as a 
	$\mathcal{CS}_N$-grid function $\mathbb{D}^n(Q^n,u^n,v^n)$
	given by:
	\begin{equation}
		\label{chp5-def-div-eq}
		\mathbb{D}_{ijp}^n(Q^n,u^n,v^n)=  \frac{1}{\Delta t \sigma_{ijp}}
		\bigg(\frac{\delta_i {F}_{ijp}^{n}}{\Delta x} + \frac{\delta_j {G}_{ijp}^{n}}{\Delta y} \bigg), 
		\quad i = 1, \ldots, N, \quad j=1, \ldots,M.
	\end{equation}
\end{definition}
With the aid of the discrete divergence, Equation \eqref{chp5-csfv} becomes:
	\begin{equation}
	\label{chp5-def-div-eq2}
	Q^{n+1} = Q^n - \Delta t \mathbb{D}^n(Q^n,u^n,v^n).
\end{equation}
For a CS-FV scheme the discrete total mass at the time-step $n$ is given by
\begin{equation*}
	M^n =\sum_{p=1}^6 \sum_{i,j=1}^N Q_{ijp}^n \sigma_{ijp} \Delta x \Delta y 
\end{equation*}
It follows from Equation \eqref{chp5-def-div-eq2} that:
\begin{align*}
	M^{n+1} &= M^n  - \sum_{p=1}^6 \sum_{i,j=1}^N \mathbb{D}_{ijp}^{n} \sigma_{ijp} \Delta x \Delta y .
\end{align*}
Hence, to ensure mass conservation, we must ensure that
\begin{align*}
	\sum_{p=1}^6 \sum_{i,j=1}^N   \mathbb{D}_{ijp}^{n} \sigma_{ijp} \Delta x \Delta y = 0.
\end{align*}
This property is discrete version of
\begin{align*}
	\int_{\mathbb{S}^2_R} \nabla \cdot (\boldsymbol{u}q) \,dS = 0,
\end{align*}
which follows from the divergence theorem and the fact of the sphere has no boundary, where $\,dS$ is the surface measure of the sphere.

%Consistent Treatment of Boundaries: Boundary conditions play a crucial role in maintaining mass conservation. 
%The treatment of boundaries should ensure that there is no spurious mass exchange between the model domain and the exterior. 
%Various approaches, such as using ghost cells or extrapolation methods, can be employed to enforce mass conservation at the boundaries.
As we mentioned in Problem \ref{chp5-prob3}, the initial condition may be assumed as $q_{ijp}^0$ or $Q_{ijp}(0)$.
We are going to assume  $q_{ijp}^0$ as initial data to avoid the computation of integrals.
Furthermore, the errors will be calculated using the values $q_{ijp}^n$ instead of $Q_{ijp}(t_n)$.
As in Section \ref{sec:fv-2d} this approximation leads to a second-order error.

\subsection{Dimension splitting}
\label{sec-csdsplit}
In this section, we will utilize the operator splitting method described in Section \ref{sec-dsplit} 
to obtain a CS-FV scheme.
We will focus on 1D fluxes, specifically the PPM fluxes introduced in Section \ref{chp2-sec-flux},
denoted as ${F}_{i+\frac{1}{2},j,p}^n$ and ${G}_{i,j+\frac{1}{2},p}^n$.
To facilitate the scheme, we introduce the auxiliary grid functions $\mathbf{F}$ and $\mathbf{G}$ belonging to $\mathcal{CS}_{N}$, defined as follows:
\begin{align*}
	\mathbf{F}_{ijp}({Q^n,\tilde{u}^n}) = -{\lambda} \bigg({F}_{i+\frac{1}{2},j,p}^n(Q^n_{\times,j,p},\tilde{u}^n_{i+\frac{1}{2},j,p})-
	{F}_{i-\frac{1}{2},j,p}^n(Q^n_{\times,j,p},\tilde{u}^n_{i-\frac{1}{2},j,p}) \bigg),
\end{align*}
for $i=1, \ldots, N$, $j=-\nu+1, \ldots, M + \nu$, and
\begin{align*}
	\mathbf{G}_{ijp}({Q^n,\tilde{v}^n}) = -{\lambda} \bigg({G}_{i,j+\frac{1}{2},p}^n(Q^n_{i,\times,p},\tilde{v}^n_{i,j+\frac{1}{2},p})-
	{G}_{i,j-\frac{1}{2},p}^n(Q^n_{i,\times,p},\tilde{v}^n_{i,j-\frac{1}{2},p}) \bigg),
\end{align*}
for $i=-\nu+1, \ldots, N + \nu$  $j=1, \ldots, M$.
The terms $\tilde{u}^n_{i+\frac{1}{2},j,p}$ and $\tilde{v}^n_{i,j+\frac{1}{2},p}$ represent the 
time-averaged winds used in the computation of departure points in the $x$ and $y$ directions, respectively. 
These averages are calculated using either RK1 or RK2 from Section \ref{chp2-sec-dp}. 
To facilitate the description of the 1D fluxes ${F}_{i+\frac{1}{2},j,p}^n$ and ${G}_{i,j+\frac{1}{2},p}^n$, 
we introduce the time-averaged CFL numbers defined as follows:
\begin{align*}
	\tilde{c}_{i+\frac{1}{2},j,p}^{x,n} = \tilde{u}_{i+\frac{1}{2},j,p}^{x,n}\frac{\Delta t}{\Delta x},\
	\tilde{c}_{i,j+\frac{1}{2},p}^{y,n} = \tilde{v}_{i,j+\frac{1}{2},p}^{y,n}\frac{\Delta t}{\Delta y}.
\end{align*}
The discrete divergence is then obtained as:
\begin{equation}
	\mathbb{D}^n_{ijp} = -\frac{1}{\Delta t \sigma_{ijp}}
	\bigg[
	\mathbf{F}_{ijp}\bigg(Q^n + \frac{1}{2}\mathbf{g}(Q^n,\tilde{v}^n), \tilde{u}^n \bigg) 
	+\mathbf{G}_{ijp}\bigg(Q^n + \frac{1}{2}\mathbf{f}(Q^n,\tilde{u}^n), \tilde{v}^n \bigg) \bigg],
\end{equation}
where the inner advective operators $\mathbf{f}$ and $\mathbf{g}$ are given in Table \ref{chp3-tab1}.

Now, our objective is to describe the 1D fluxes ${F}_{i+\frac{1}{2},j,p}^n$ and ${G}_{i,j+\frac{1}{2},p}^n$. 
It is important to note that these fluxes depend on the edge treatment, as well as the computation of stencils, as we will see in the following sections.


\subsubsection{ET-ZA22}
We have a piecewise-parabolic approximation in the $x$ direction:
\begin{align}
	\label{chp5-ppmx-eq1}
	\begin{cases}
		q_{ij}^x(x;Q_{\times, j}^n) = q_{L,i,j}^x + \Delta q_{ij}^x z_i(x) + q_{6, i,j}^xz_i(x)(1-z_i(x)), \\
		z_i(x) = \frac{x-x_{i-\frac{1}{2}}}{\Delta x},
		\quad x \in X_i,\\
		q_{L, i,j}^x = q_{i-\frac{i}{2},j}^n+ O(\Delta x^2),\\
		q_{R, i,j}^x = q_{i+\frac{i}{2},j}^n+ O(\Delta x^2),\\
		\Delta q_{ij}^x = q_{R, i,j}^x - q_{L, i,j}^x,\\
		q_{6,i,j}^x = 6\bigg(Q_{ij}^n - \frac{(q_{L,i,j}^x + q_{R,i,j}^x)}{2}\bigg),
	\end{cases}
\end{align}
for $i=1, \ldots, N$, $j=-\nu+1, \ldots, M + \nu$, and we also construct a piecewise-parabolic
approximation in the $y$ direction:
\begin{align}
	\label{chp5-ppmy-eq2}
	\begin{cases}
		q_{ij}^y(y;Q_{i,\times}^n) = q_{L,i,j}^y + \Delta q_{ij}^y z_j(y) + q_{6, i,j}^yz_j(y)(1-z_j(y)),\\ 
		z_j(y) = \frac{y-y_{j-\frac{1}{2}}}{\Delta y},
		\quad y \in Y_j,\\
		q_{L, i,j}^y = q_{i,j-\frac{1}{2}}^n+ O(\Delta y^2),\\
		q_{R, i,j}^y = q_{i,j+\frac{1}{2}}^n+ O(\Delta y^2),\\
		\Delta q_{ij}^y = q_{R, i,j}^y - q_{L, i,j}^y,\\
		q_{6,i,j}^y = 6\bigg(Q_{ij}^n - \frac{(q_{L,i,j}^y + q_{R,i,j}^y)}{2}\bigg),
	\end{cases}
\end{align}
for $i=-\nu+1, \ldots, N + \nu$, $j=1, \ldots, M$.
The values $q_{L,i,j}^x$, $q_{R,i,j}^x$, $q_{L,i,j}^y$, and $q_{R,i,j}^y$,
which approximate the values of $q$ at C-grid wind positions, are computed
using one of the schemes PPM-0, PPM-PL07, PPM-CW84, or PPM-L04, as described
in Sections \ref{chp2-sec-ppm} and \ref{chp2-sec-mono}.
These approximations are expected to be
second-order accurate because the given average values are computed on the
2D control volume $\Omega_{ij}$ instead of the 1D control volumes $X_i$ or $Y_j$.
Then, we may express the fluxes as in Equation \eqref{chp-sec-flux:numerical-flux3}, namely:
\begin{align}
	\label{chp5-flux-xdir}
	F_{i+\frac{1}{2},j}^n ({Q^n_{\times,j},\tilde{u}^n_{i+\frac{1}{2},j}})= \tilde{u}^{n}_{i+\frac{1}{2},j}\times
	\begin{cases}
		q_{R,i,j}^x +\frac{1}{2}(q_{6,i,j}^x - \Delta q_{ij}^x){\tilde{c}_{i+\frac{1}{2},j}^{x,n}}
		+\frac{1}{3}{q_{6,i,j}^x}(\tilde{c}_{i+\frac{1}{2},j}^{x,n})^2,
		\quad &\text{if} \quad \tilde{u}_{i+\frac{1}{2},j}^n>0,\\
		q_{L,i+1,j}^x - \frac{1}{2}(q_{6,i+1,j}^x + \Delta q_{i+1,j}^x){\tilde{c}_{i+\frac{1}{2},j}^{x,n}}
		-\frac{1}{3}{q_{6,i+1,j}^x}(\tilde{c}_{i+\frac{1}{2},j}^{x,n})^2,
		\quad &\text{if} \quad \tilde{u}_{i+\frac{1}{2},j}^n\leq0,\\
	\end{cases}
\end{align}
for $i=0, \ldots, N$, $j=-\nu+1, \ldots, M + \nu$, and 
\begin{align}
	\label{chp5-flux-ydir}
	G_{i,j+\frac{1}{2}}^n ({Q^n_{i,\times},\tilde{v}^n_{i,j+\frac{1}{2}}})= \tilde{v}^{n}_{i,j+\frac{1}{2}}\times
	\begin{cases}
		q_{R,i,j}^y +\frac{1}{2}(q_{6,i,j}^y - \Delta q_{ij}^y){\tilde{c}_{i,j+\frac{1}{2}}^{y,n}}
		+\frac{1}{3}{q_{6,i,j}^y}(\tilde{c}_{i,j+\frac{1}{2}}^{y,n})^2,
		\quad &\text{if} \quad \tilde{v}_{i+\frac{1}{2},j}^n>0,\\
		q_{L,i,j+1}^y - \frac{1}{2}(q_{6,i,j+1}^y + \Delta q_{i,j+1}^y){\tilde{c}_{i,j+\frac{1}{2}}^{y,n}}
		-\frac{1}{3}{q_{6,i,j+1}^y}(\tilde{c}_{i,j+\frac{1}{2}}^{y,n})^2,
		\quad &\text{if} \quad \tilde{v}_{i,j+\frac{1}{2}}^n\leq0,\\
	\end{cases}
\end{align}
for $i=-\nu+1, \ldots, N + \nu$, $j=0, \ldots, M$.
\subsubsection{ET-PL07}
We have a piecewise-parabolic approximation in the $x$ direction:
\begin{align}
	\label{chp5-ppmx-eq1}
	\begin{cases}
		q_{ij}^x(x;Q_{\times, j}^n) = q_{L,i,j}^x + \Delta q_{ij}^x z_i(x) + q_{6, i,j}^xz_i(x)(1-z_i(x)), \\
		z_i(x) = \frac{x-x_{i-\frac{1}{2}}}{\Delta x},
		\quad x \in X_i,\\
		q_{L, i,j}^x = q_{i-\frac{i}{2},j}^n+ O(\Delta x^2),\\
		q_{R, i,j}^x = q_{i+\frac{i}{2},j}^n+ O(\Delta x^2),\\
		\Delta q_{ij}^x = q_{R, i,j}^x - q_{L, i,j}^x,\\
		q_{6,i,j}^x = 6\bigg(Q_{ij}^n - \frac{(q_{L,i,j}^x + q_{R,i,j}^x)}{2}\bigg),
	\end{cases}
\end{align}
for $i=1, \ldots, N$, $j=-\nu+1, \ldots, M + \nu$, and we also construct a piecewise-parabolic
approximation in the $y$ direction:
\begin{align}
	\label{chp5-ppmy-eq2}
	\begin{cases}
		q_{ij}^y(y;Q_{i,\times}^n) = q_{L,i,j}^y + \Delta q_{ij}^y z_j(y) + q_{6, i,j}^yz_j(y)(1-z_j(y)),\\ 
		z_j(y) = \frac{y-y_{j-\frac{1}{2}}}{\Delta y},
		\quad y \in Y_j,\\
		q_{L, i,j}^y = q_{i,j-\frac{1}{2}}^n+ O(\Delta y^2),\\
		q_{R, i,j}^y = q_{i,j+\frac{1}{2}}^n+ O(\Delta y^2),\\
		\Delta q_{ij}^y = q_{R, i,j}^y - q_{L, i,j}^y,\\
		q_{6,i,j}^y = 6\bigg(Q_{ij}^n - \frac{(q_{L,i,j}^y + q_{R,i,j}^y)}{2}\bigg),
	\end{cases}
\end{align}
for $i=-\nu+1, \ldots, N + \nu$, $j=1, \ldots, M$.
The values $q_{L,i,j}^x$, $q_{R,i,j}^x$, $q_{L,i,j}^y$, and $q_{R,i,j}^y$,
which approximate the values of $q$ at C-grid wind positions, are computed
using one of the schemes PPM-0, PPM-PL07, PPM-CW84, or PPM-L04, as described
in Sections \ref{chp2-sec-ppm} and \ref{chp2-sec-mono}.
These approximations are expected to be
second-order accurate because the given average values are computed on the
2D control volume $\Omega_{ij}$ instead of the 1D control volumes $X_i$ or $Y_j$.
Then, we may express the fluxes as in Equation \eqref{chp-sec-flux:numerical-flux3}, namely:
\begin{align}
	\label{chp5-flux-xdir}
	F_{i+\frac{1}{2},j}^n ({Q^n_{\times,j},\tilde{u}^n_{i+\frac{1}{2},j}})= \tilde{u}^{n}_{i+\frac{1}{2},j}\times
	\begin{cases}
		q_{R,i,j}^x +\frac{1}{2}(q_{6,i,j}^x - \Delta q_{ij}^x){\tilde{c}_{i+\frac{1}{2},j}^{x,n}}
		+\frac{1}{3}{q_{6,i,j}^x}(\tilde{c}_{i+\frac{1}{2},j}^{x,n})^2,
		\quad &\text{if} \quad \tilde{u}_{i+\frac{1}{2},j}^n>0,\\
		q_{L,i+1,j}^x - \frac{1}{2}(q_{6,i+1,j}^x + \Delta q_{i+1,j}^x){\tilde{c}_{i+\frac{1}{2},j}^{x,n}}
		-\frac{1}{3}{q_{6,i+1,j}^x}(\tilde{c}_{i+\frac{1}{2},j}^{x,n})^2,
		\quad &\text{if} \quad \tilde{u}_{i+\frac{1}{2},j}^n\leq0,\\
	\end{cases}
\end{align}
for $i=0, \ldots, N$, $j=-\nu+1, \ldots, M + \nu$, and 
\begin{align}
	\label{chp5-flux-ydir}
	G_{i,j+\frac{1}{2}}^n ({Q^n_{i,\times},\tilde{v}^n_{i,j+\frac{1}{2}}})= \tilde{v}^{n}_{i,j+\frac{1}{2}}\times
	\begin{cases}
		q_{R,i,j}^y +\frac{1}{2}(q_{6,i,j}^y - \Delta q_{ij}^y){\tilde{c}_{i,j+\frac{1}{2}}^{y,n}}
		+\frac{1}{3}{q_{6,i,j}^y}(\tilde{c}_{i,j+\frac{1}{2}}^{y,n})^2,
		\quad &\text{if} \quad \tilde{v}_{i+\frac{1}{2},j}^n>0,\\
		q_{L,i,j+1}^y - \frac{1}{2}(q_{6,i,j+1}^y + \Delta q_{i,j+1}^y){\tilde{c}_{i,j+\frac{1}{2}}^{y,n}}
		-\frac{1}{3}{q_{6,i,j+1}^y}(\tilde{c}_{i,j+\frac{1}{2}}^{y,n})^2,
		\quad &\text{if} \quad \tilde{v}_{i,j+\frac{1}{2}}^n\leq0,\\
	\end{cases}
\end{align}
for $i=-\nu+1, \ldots, N + \nu$, $j=0, \ldots, M$.

