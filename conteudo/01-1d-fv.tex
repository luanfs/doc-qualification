%!TeX root=../tese.tex
%("dica" para o editor de texto: este arquivo é parte de um documento maior)
% para saber mais: https://tex.stackexchange.com/q/78101

\chapter{One-dimensional finite-volume methods}
\label{chp1-1d-fv}
\section{One-dimensional conservation laws in integral form}
\label{chp1-sec1}
In this section, we are going to present the derivation of one-dimensional conservation
laws in the integral form. 
The derivation presented here follows \citet{leveque:1990} and \citet{leveque:2002} closely and will
be useful to fix some notation. \index{Conservation law}
Let us assume that $x$ and $t$ represent the spatial and time coordinate, respectively.
Given $[x_1, x_2] \subset \mathbb{R}$, $x_1 \leq x_2$, and a time 
interval $[t_1, t_2] \subset ]0, +\infty[$, $t_1 \leq t_2$, 
our aim is to describe how $m$ state variables given by functions 
$q_1, \cdots, q_m: \mathbb{R}\times[0, +\infty[ \to \mathbb{R}$ 
evolve within time in the considered time interval, assuming that we have neither sinks nor sources 
for the mass of each state variable and also assuming that the mass
flow rate is known for all the state variables.

To set the problem in more mathematical terms, let us denote by 
$\mathbf{q}: \mathbb{R}\times [0, \infty[\to \mathbb{R}^m$, 
$\mathbf{q} = \mathbf{q}(x,t)$, the vector of state variables,
\textit{i.e.}, $\mathbf{q}_k = q_k$ for $k=1, \cdots, m$.
The mass of $\mathbf{q}$ in $[x_1, x_2]$ at time $t$ is defined by:
\begin{equation}
	\label{chp1-sec1-eq1}
	\mathbf{M}_{[x_1, x_2]}(t) := \int_{x_1}^{x_2} \mathbf{q}(x,t) \,dx \in \mathbb{R}^m.
\end{equation}

Thus, the mass in $[x_1, x_2]$ of the $k$-th state variable $q_k$ is equal to
$(\mathbf{M}_{[x_1, x_2]}(t))_k$, $\forall k = 1, \cdots, m$.
We are going to assume the following physical constraints concerning the total mass of each state variable:
\begin{enumerate}
	\item No mass is created;
	\item No mass is destroyed.
\end{enumerate}

Also, let us assume that the mass flow rate in a point $x$ and at a time 
$t > 0$ is given by $\mathbf{f}(\mathbf{q}(x,t))$, where $\mathbf{f}:\mathbb{R}^m \to \mathbb{R}^m$ is 
a continuously differentiable ($\mathcal{C}^1$) function. This function $\mathbf{f}$ is known as flux function.
With the physical constraints that we imposed, the following equation must hold for the mass:
\begin{equation}
	\label{chp1-sec1-eq2}
	\frac{d}{dt} \bigg( \int_{x_1}^{x_2} \mathbf{q}(x,t) \,dx \bigg) = 
	\mathbf{f}(\mathbf{q}(x_1,t)) - \mathbf{f}(\mathbf{q}(x_2,t)) .
\end{equation}

Equation \eqref{chp1-sec1-eq2} is known as a conservation law written in integral form and tell us how the mass 
$\mathbf{M}_{[x_1, x_2]}(t)$ varies with time. Another integral form of the conservation law may be obtained integrating
Equation \eqref{chp1-sec1-eq2} with respect to time in $[t_1, t_2]$ leading to: \index{Conservation law !in integral form}
\begin{equation}
	\label{chp1-sec1-eq3}
	\int_{x_1}^{x_2} \mathbf{q}(x, t_2) \,dx = 
	\int_{x_1}^{x_2} \mathbf{q}(x, t_1) \,dx + 
	\int_{t_1}^{t_2} \mathbf{f}(\mathbf{q}(x_1, t)) \,dt -
	\int_{t_1}^{t_2}\mathbf{f}(\mathbf{q}(x_2, t)) \,dt .
\end{equation}

Assuming that $\mathbf{q}$ is a $\mathcal{C}^1$ function, we may write:
\begin{equation}
	\label{chp1-sec1-eq4}
	\int_{t_1}^{t_2} 
	\frac{\partial}{\partial t} \mathbf{q}(x,t) \,dt
	= \mathbf{q}(x, t_2) - \mathbf{q}(x, t_1) ,
\end{equation}
and
\begin{equation}
	\label{chp1-sec1-eq5}
	\int_{x_1}^{x_2} \frac{\partial}{\partial x}\mathbf{f}(\mathbf{q}(x,t)) \,dx 
	= \mathbf{f}(\mathbf{q}(x_2, t)) -
	\mathbf{f}( \mathbf{q}(x_1, t)) .
\end{equation}

Replacing Equations \eqref{chp1-sec1-eq4} and \eqref{chp1-sec1-eq5}
in \eqref{chp1-sec1-eq3} we get the differential form of the conservation law:
\index{Conservation law !in differential form}
\begin{equation}
	\label{chp1-sec1-eq6}
	\int_{t_1}^{t_2} \int_{x_1}^{x_2} 
	\bigg( \frac{\partial}{\partial t}\mathbf{q}(x, t) 
	+ \frac{\partial}{\partial x} \mathbf{f}(\mathbf{q}(x, t)) \bigg) 
	\,dx \,dt  = 0.
\end{equation}

Since Equation \eqref{chp1-sec1-eq6} must hold for all $x_1, x_2, t_1$ and $t_2$ such that
$[x_1, x_2] \times [t_1, t_2] \subset \mathbb{R}\times ]0, +\infty[$, we obtain the differential form of the conservation law:
\begin{equation}
	\label{chp1-sec1-eq7}
	\frac{\partial}{\partial t}\mathbf{q}(x, t) +
	\frac{\partial}{\partial x} \mathbf{f}(\mathbf{q}(x, t))
	= 0, \quad \forall (x,t) \in \mathbb{R}\times ]0, +\infty[. 
\end{equation}

We shall assume that the eigenvalues of the Jacobian matrix of the flux function
$D\mathbf{f}(q)$ are all real and that $D\mathbf{f}(q)$ is a diagonalizable matrix,
$\forall q \in \mathbb{R}^m$, so that Equation \eqref{chp1-sec1-eq7}
is a hyperbolic partial differential equation \citep{leveque:1990}. As we will 
specify latter, some initial condition will also be suposed to be known as well.

Many physical relevant equations may be written as Equation \eqref{chp1-sec1-eq7}.
Some examples are the Euler equations for gas dynamics, obtained when $m = 3$,
and the one-dimensional shallow-water equations, obtained $m = 2$.
Another relevant equations are the Burgers equation, which is obtained when
$m = 1$ and $f(q) = q^2$. The Burgers equation is well known for developing shocks,
even for smooth initial conditions and is a simple prototype to study shock formation.
At last, the linear advection equation is another interesting example, which is obtained
when $m = 1$ and $f(q(x,t)) = u(x,t)q(x,t)$, where $u(x,t)$ is a given velocity.
Strictly speaking, the linear advection is not in the form given by Equation
\eqref{chp1-sec1-eq7} since $f$ depends on $q$ but also on $(x,t)$.
But, one may check that Equation \eqref{chp1-sec1-eq7} is still hyperbolic
in this case. The linear advection equation will play a key role in this work due to its importance
to development of atmospheric dynamical cores.


We say that $\mathbf{q}$ is a strong or classical solution to the conservation law \eqref{chp1-sec1-eq7}
if it is $\mathcal{C}^1$ and satisfies the Equation \eqref{chp1-sec1-eq7}.
Applying the steps from Equation \eqref{chp1-sec1-eq3} to Equation \eqref{chp1-sec1-eq7}
in a reverse order, one may check that if $\mathbf{q}$ is a strong solution,
then it satisfies the integral form \eqref{chp1-sec1-eq3} for all $x_1, x_2, t_1$ and $t_2$ such that
$[x_1, x_2] \times [t_1, t_2] \subset \mathbb{R}\times ]0, +\infty[$. 
Therefore, Equations \eqref{chp1-sec1-eq3} and \eqref{chp1-sec1-eq7} are
equivalent when $\mathbf{q}$ is $\mathcal{C}^1$.
However, the problem \eqref{chp1-sec1-eq3} can be formulated
to functions that are not $\mathcal{C}^1$ and have discontinuities.
More generally speaking, we say that $\mathbf{q} \in L^{\infty}(\Omega, \mathbb{R}^m)$ 
\footnote{$L^{\infty}(\Omega, \mathbb{R}^m) = \{q: \Omega \to \mathbb{R}^m$
such that $q$ is bounded.$\}$}
if it satisfies the Equation 
\eqref{chp1-sec1-eq3} for all $x_1, x_2, t_1$ and $t_2$ such that
$[x_1, x_2] \times [t_1, t_2] \subset \mathbb{R}\times ]0, +\infty[$.
It can be shown that this notion of weak solution is equivalent to requiring that \citep{leveque:1990}:
\begin{equation}
	\label{chp1-sec1-eq8}
	\int_{-\infty}^{+\infty} \int_{0}^{+\infty} \bigg(
	\frac{\partial}{\partial t} \phi(x, t)\mathbf{q}(x, t) +
	\frac{\partial}{\partial x} \phi(x ,t)\mathbf{f}(\mathbf{q}(x, t)) 
	\bigg)\,dt \,dx = 
	\int_{-\infty}^{+\infty} \phi(x, 0)\mathbf{q}(x, 0) \,dx  , \quad
\end{equation}
$\forall \phi \in C_{0}^{1}(\mathbb{R}\times[0, +\infty[)$
where $C_{0}^{1}(\mathbb{R}\times[0, +\infty[)$ denotes the set
of all continuously differentiable functions with compact support 
in $\mathbb{R}\times[0, +\infty[$. This formulation of weak solution
is more common employed on the construction of Discontinous Galerkin
methods \citep{nair:2011}.

In order to develop finite-volume methods for conservation laws, it is useful to define the vector of
average values of the state variable vector $\mathbf{q}$ in the interval $[x_1, x_2]$ at a time $t$ by:
\begin{equation}
	\label{chp1-sec1-eq9}
	\mathbf{Q}(t) = \frac{1}{\Delta x}
	\int_{x_1}^{x_2} \mathbf{q}(x,t) \,dx
	\in \mathbb{R}^m,
\end{equation}
where $\Delta x = x_2 - x_1$. The Equation \eqref{chp1-sec1-eq2} may be  rewritten in terms of $\mathbf{Q}$ as:
\begin{equation}
        \label{chp1-sec1-eq10}
	\frac{d}{dt} \mathbf{Q}(t) = \frac{1}{\Delta x} 
	(\mathbf{f}(\mathbf{q}(x_1,t)) - \mathbf{f}(\mathbf{q}(x_2,t))) ,
\end{equation}
and so is Equation \eqref{chp1-sec1-eq3}:
\begin{equation}
        \label{chp1-sec1-eq11}
	\mathbf{Q}(t_2) =  \mathbf{Q}(t_1) + 
	\frac{1}{\Delta x}\bigg( \int_{t_1}^{t_2} 
	\mathbf{f}(\mathbf{q}(x_1, t)) \,dt - 
	\int_{t_1}^{t_2}\mathbf{f}(\mathbf{q}(x_2, t)) \,dt \bigg).
\end{equation}

To move towards finite volume schemes, we will restrict our attention
to a conservation law in a bounded domain of the form 
$\Omega = [a,b]\times[0,T]$, $a<b$, $T>0$. However, we must 
impose some boundary condition. One possible way and that we will adopted 
in text are the periodic boundary conditions:
\begin{equation}
        \label{chp1-sec1-eq12}
	\mathbf{q}(a, t) = \mathbf{q}(b, t),\quad \forall t \in [0, T].
\end{equation}

Also, we assume that an initial condition $q_0(x) = q(x,0)$, $q_0 \in L^{\infty}([a,b],\mathbb{R}^m)$, is given.
Thus, we have specified a Cauchy problem.
We notice that Equations \eqref{chp1-sec1-eq10} and \eqref{chp1-sec1-eq11}
hold for all $x_1, x_2, t_1$ and $t_2$ such that
$[x_1, x_2] \times [t_1, t_2] \subset \Omega$.
So, let us discretize the domain $\Omega$ and write 
Equations \eqref{chp1-sec1-eq10} and \eqref{chp1-sec1-eq11} in terms of this discretization.
Given a positive integer $N_T$, 
we define the time step $\Delta t = \frac{T}{N_T}$, $t_n = n \Delta t$, for $n = 0, 1 ,\cdots, N_T$.
For the spatial discretization, we consider a partition of $[a, b]$ given by: 
\begin{equation}
	\label{chp1-sec1-eq13}
	[a,b] = \bigcup_{i=1}^N X_i, 
	\text{ where } X_i= [x_{i-\frac{1}{2}}, x_{i+\frac{1}{2}}] \text{ and } 
	a = x_{\frac{1}{2}} < x_{\frac{3}{2}} < \cdots < x_{N-\frac{1}{2}} < x_{N+\frac{1}{2}} = b.
\end{equation}

Each interval $X_i$ is refered to as control volume. \index{Control volume}
We shall use the notations $|X_i| = x_{i+\frac{1}{2}} - x_{i-\frac{1}{2}}$ 
and $x_i = \frac{1}{2}(x_{i+\frac{1}{2}} + x_{i-\frac{1}{2}})$, $\forall i = 1, \cdots, N$, 
to define the control volume length and midpoint, respectively.
We also denote by $\mathbf{Q}_i(t) \in \mathbb{R}^m$ as the vector of 
average values of state variable vector at time $t$
in the control volume $X_i$, $\forall i = 1, \cdots, N$. Replacing $t_1, t_2, x_1$ and 
$x_2$ by $t_{n}, t_{n+1}, x_{i-\frac{1}{2}}$ and $x_{i+\frac{1}{2}}$,
respectively, in Equation \ref{chp1-sec1-eq10}, we get:
\begin{equation}
        \label{chp1-sec1-eq14}
	\frac{d}{dt} \mathbf{Q}_i(t) = \frac{1}{|X_i|}
	(\mathbf{f}(\mathbf{q}(x_{\frac{i-1}{2}},t)) -
	\mathbf{f}(\mathbf{q}(x_{\frac{i+1}{2}},t))) ,
	\quad \forall i = 1, \cdots, N.
\end{equation}

Similarly, Equation \eqref{chp1-sec1-eq11} becomes:
\begin{equation}
        \label{chp1-sec1-eq15}
	\begin{aligned}
		\mathbf{Q}_i(t_{n+1}) =  \mathbf{Q}_i(t_n) +
		\frac{1}{|X_i|}\bigg( \int_{t_n}^{t_{n+1}}
        	\mathbf{f}(\mathbf{q}(x_{i-\frac{1}{2}}, t)) \,dt -
		\int_{t_n}^{t_{n+1}}\mathbf{f}(\mathbf{q}(x_{i+\frac{1}{2}}, t)) \,dt \bigg),
       		\\
		\quad \forall i = 1, \cdots, N,
		\quad \forall n = 1, \cdots, N_T.
	\end{aligned}
\end{equation}

In order to use a more compact notation, it is helpful to use the following centered difference notation:
\begin{equation}
	\label{chp1-sec1-eq16}
	\delta_x \mathbf{g}(x_i,t) = 
	\mathbf{g}(x_{i+\frac{1}{2}},t) - 
	\mathbf{g}(x_{i-\frac{1}{2}},t),
\end{equation}
for an arbitrary vector valued function $\mathbf{g}$. 
Using this notation, Equations \eqref{chp1-sec1-eq14}
and \eqref{chp1-sec1-eq15} lead to:
\begin{equation}
        \label{chp1-sec1-eq17}
        \frac{d}{dt} \mathbf{Q}_i(t) = -\frac{1}{|X_i|}
	\delta_x \mathbf{f}(\mathbf{q}(x_{i},t))
        \quad \forall i = 1, \cdots, N,
\end{equation}
and
\begin{equation}
        \label{chp1-sec1-eq18}
        \mathbf{Q}_i(t_{n+1}) =  \mathbf{Q}_i(t_n) -
	\frac{\Delta t}{|X_i|} \delta _x\bigg( \frac{1}{\Delta t}\int_{t_n}^{t_{n+1}}
        \mathbf{f}(\mathbf{q}(x_{i}, t)) \,dt \bigg),
        \quad \forall i = 1, \cdots, N,
        \quad \forall n = 1, \cdots, N_T,
\end{equation}
respectively.
It is worth pointing out that we have made no approximation in Equations
\eqref{chp1-sec1-eq17} and \eqref{chp1-sec1-eq18}. Indeed, if $\mathbf{q}$ satisfies Equation
$\eqref{chp1-sec1-eq2}$, $\forall [x_1, x_2] \subset [a,b]$ and $\forall t \in [0,T]$,
then Equation \eqref{chp1-sec1-eq17} is just Equation
\eqref{chp1-sec1-eq2} evaluated in the control volumes and written
in terms of the average values $\mathbf{Q}$. 
Similarly, if $\mathbf{q}$ satisfies Equation
$\eqref{chp1-sec1-eq3}$, $\forall [x_1, x_2] \times [t_1, t_2] \subset \Omega$,
then Equation \eqref{chp1-sec1-eq18} is just Equation
\eqref{chp1-sec1-eq3} evaluated in the control volumes,
at the time instants $t_n$, and written
in terms of the average values $\mathbf{Q}$.

Notice that in Equation \eqref{chp1-sec1-eq18} we divided and multiplied by $\Delta t$, so that 
we can interpret $\frac{1}{\Delta t}\int_{t_n}^{t_{n+1}}
\mathbf{f}(\mathbf{q}(x_{i}, t)) \,dt $ as a mean-time average flux.
This interpretation is very handy for the derivation of finite-volume schemes.

The formulations given by Equations \eqref{chp1-sec1-eq17} and \eqref{chp1-sec1-eq18} are the cornerstone 
of the development of finite volume methods for conservation laws. 
On the right-hand side of Equation \eqref{chp1-sec1-eq17}, the flux function $\mathbf{f}$ 
may be discretized leading to an ordinary differential equation (ODE)
that might be solved using classical ODE integrators. 
These methods are known as semi-discrete methods \citep{leveque:2002}, since only the spatial coordinate is discretized.
In this work we shall restrict our attention to methods based on Equation \eqref{chp1-sec1-eq18}.

\section{The finite-volume approach}
\label{chp1-sec2}
We summarize the problem of the conservation law in the integral form 
discussed in Section \ref{chp1-sec1} in Problem \ref{chp1-sec2-prob1}.

\newtheoremstyle{problem} % nome do estilo
{3pt} % espaço antes
{3pt} % espaço depois
{\itshape} % fonte do corpo
{} % Indentação
{\bfseries} % fonte do título
{:} % pontuação após o título
{.5em} % espaço após o título
{\thmname{#1} \thmnumber{ #2}\thmnote{ (#3)}} % formato do título
% vazio significa {\thmname{#1}\thmnumber{ #2}\thmnote{ (#3)}}

\theoremstyle{problem} % As próximas definições usam este estilo
\newtheorem{prob}{Problem}

\begin{prob}
	\label{chp1-sec2-prob1}
	Given $ \Omega = [a,b] \times [0,T]$, a $\mathcal{C}^1$ 
	flux function $\mathbf{f}: \mathbb{R}^m \to \mathbb{R}^m $,
	$m \geq 1$, we would like to find the weak solution
	$ \mathbf{q} \in L^{\infty}(\Omega, \mathbb{R}^m)$ 
	of the conservation law in the integral form:
	\begin{equation*}
	        \int_{x_1}^{x_2} \mathbf{q}(x, t_2) \,dx = 
       		\int_{x_1}^{x_2} \mathbf{q}(x, t_1) \,dx + 
        	\int_{t_1}^{t_2} \mathbf{f}(\mathbf{q}(x_1, t)) \,dt -
		\int_{t_1}^{t_2}\mathbf{f}(\mathbf{q}(x_2, t)) \,dt ,
	\end{equation*}
	$\forall [x_1, x_2]\times[t_1, t_2] \subset \Omega$, 
	given the initial condition 
	$\mathbf{q}(x,0) = \mathbf{q}_0(x)$, $\forall x \in [a,b]$, 
	and assuming periodic boundary conditions, 
	\textit{i.e.}, $\mathbf{q}(a,t) = \mathbf{q}(b,t)$, $\forall t \in [0,T]$.
\end{prob}

We point out that, for Problem \ref{chp1-sec2-prob1}, 
the total mass in $[a,b]$ satisfies: 
\begin{equation}
	\mathbf{M}_{[a,b]}(t) = \mathbf{M}_{[a,b]}(0), \quad \forall t \in [0,T].
\end{equation}
This is the conservation of total mass proprierty and is highly desirable
for any numerical scheme that intends to give a robust approximation of the 
conservation law solution.

In Section \ref{chp1-sec1} we introduced a version of Problem \ref{chp1-sec2-prob1}
considering a discretization of the domain $\Omega$. 
This idea is summarized in Problem \ref{chp1-sec2-prob2}.
\begin{prob}
        \label{chp1-sec2-prob2}
	Assume the framework of Problem \ref{chp1-sec2-prob1}.
        We consider positive integers $N$ and $N_T$, a spatial discretization of [a,b] given by
        $X_i = [x_{i-\frac{1}{2}}, x_{i+\frac{1}{2}}]$,
        $\forall i = 1, \cdots, N,$ 
	$a = x_{\frac{1}{2}} < x_{\frac{3}{2}} < \cdots < x_{N-\frac{1}{2}} < x_{N+\frac{1}{2}} = b$,
        a time discretization
        $t_n = n\Delta t$, $\Delta t = \frac{T}{N_T}$, $\forall n = 1, \cdots, N_T$.
	Since we are in the framework of Problem \ref{chp1-sec2-prob1}, it follows that:
        \begin{equation*}
                \mathbf{Q}_i(t_{n+1}) =  \mathbf{Q}_i(t_n) -
                \frac{\Delta t}{|X_i|} \delta _x\bigg( \frac{1}{\Delta t}\int_{t_n}^{t_{n+1}}
                \mathbf{f}(\mathbf{q}(x_{i}, t)) \,dt \bigg),
                \quad \forall i = 1, \cdots, N,
                \quad \forall n = 1, \cdots, N_T,
        \end{equation*}
        where $\mathbf{Q}_i(t) = \frac{1}{|X_i|}
        \int_{x_{i-\frac{1}{2}}}^{x_{i+\frac{1}{2}}} \mathbf{q}(x,t) \,dx$.
	
	Our problem now consists of finding the values $\mathbf{Q}_i(t_{n})$, 
	$\forall i = 1, \cdots, N$, $\forall n = 1, \cdots, N_T$,
	given the initial values $\mathbf{Q}_i(0)$, $\forall i = 1, \cdots N$.
	In other words, we would like to find the average values of $\mathbf{q}$
	in each control volume $X_i$ at the considered time instants.
\end{prob}

Finally, we define the one-dimensional (1D) finite-volume (FV)
scheme problem as follows in Problem \ref{chp1-sec2-prob3}.
\begin{prob}[1D FV scheme]
	\label{chp1-sec2-prob3}
	Assume the framework defined in Problem \ref{chp1-sec2-prob2}.
	The finite-volume approach of Problem \ref{chp1-sec2-prob2}
	consists of a finding a scheme of the form:
        \begin{equation*}
		\mathbf{Q}_{i}^{n+1} =  \mathbf{Q}_{i}^{n} -
		\frac{\Delta t}{|X_i|} (\mathbf{F}_{i+\frac{1}{2}}^{n} - \mathbf{F}_{i-\frac{1}{2}}^{n}),
                \quad \forall i = 1, \cdots, N,
                \quad \forall n = 1, \cdots, N_T,
        \end{equation*}
	where $\mathbf{Q}_{i}^{n} \in \mathbb{R}^m$ is intended to be an approximation
	of $\mathbf{Q}_i(t_{n})$ in some sense.
	The term $\mathbf{F}_{i+\frac{1}{2}}^{n}$ approximates
	$\frac{1}{\Delta t}\int_{t_n}^{t_{n+1}} \mathbf{f}(\mathbf{q}(x_{i+\frac{1}{2}}, t)) \,dt $
	and the term
        $\mathbf{F}_{i-\frac{1}{2}}^{n}$ approximates
	$\frac{1}{\Delta t}\int_{t_n}^{t_{n+1}} \mathbf{f}(\mathbf{q}(x_{i-\frac{1}{2}}, t)) \,dt $,
	or, in other words, they estimate the time-averaged fluxes at the control volume $X_i$ boundaries.
\end{prob}
\section{The piecewise parabolic method}


